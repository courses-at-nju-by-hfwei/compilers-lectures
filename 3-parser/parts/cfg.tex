% cfg.tex

%%%%%%%%%%%%%%%%%%%%
\begin{frame}{}
  \fig{width = 0.50\textwidth}{figs/cfg}

  \begin{center}
    上下文无关文法
  \end{center}
\end{frame}
%%%%%%%%%%%%%%%%%%%%

%%%%%%%%%%%%%%%%%%%%
\begin{frame}{}
  \begin{definition}[Context-Free Grammar (CFG); 上下文无关文法]
    上下文无关文法 $G$ 是一个四元组 $G = (T, N, P, S)$:
    \vspace{0.30cm}

    \begin{itemize}
      \setlength{\itemsep}{8pt}
      \item $T$ 是\red{\bf 终结符号} (\teal{Terminal}) 集合, 对应于词法分析器产生的词法单元;
      \item $N$ 是\red{\bf 非终结符号} (\teal{Non-terminal}) 集合;
      \item $P$ 是\red{\bf 产生式} (\teal{Production}) 集合;
        \[
          \boxed{A \in N \longrightarrow \alpha \in (T \cup N)^{\ast}}
        \]
        \vspace{-0.60cm}
        \begin{description}
          \setlength{\itemsep}{5pt}
          \item[头部/左部 (Head) $A$:] \purple{\bf 单个}非终结符
          \item[体部/右部 (Body) $\alpha$:] 终结符与非终结符构成的串, 也可以是空串 $\epsilon$
        \end{description}
      \item $S$ 为\red{\bf 开始} (\teal{Start}) 符号。要求 $S \in N$且唯一。
    \end{itemize}
  \end{definition}
\end{frame}
%%%%%%%%%%%%%%%%%%%%

%%%%%%%%%%%%%%%%%%%%
\begin{frame}{}
  \[
    G = (\set{S}, \set{(,)}, P, S)
  \]

  \fig{width = 0.25\textwidth}{figs/cfg-bp}
\end{frame}
%%%%%%%%%%%%%%%%%%%%

%%%%%%%%%%%%%%%%%%%%
\begin{frame}{}
  \[
    G = (\set{S}, \set{a,b}, P, S)
  \]

  \fig{width = 0.25\textwidth}{figs/cfg-anbn}
\end{frame}
%%%%%%%%%%%%%%%%%%%%

%%%%%%%%%%%%%%%%%%%%
\begin{frame}{}
  \begin{center}
    \fig{width = 0.70\textwidth}{figs/if-statement-list}

    \vspace{0.20cm}
    关于语句块与条件语句的文法

    \vspace{0.50cm}
    \blue{\bf 约定:} 如果没有明确指定, 第一个产生式的头部就是开始符号
  \end{center}
\end{frame}
%%%%%%%%%%%%%%%%%%%%

%%%%%%%%%%%%%%%%%%%%
\begin{frame}{}
  \begin{center}
    关于\blue{\bf 终结符号}的约定

    \vspace{0.30cm}
    \fig{width = 1.00\textwidth}{figs/convention-terminal}
  \end{center}
\end{frame}
%%%%%%%%%%%%%%%%%%%%

%%%%%%%%%%%%%%%%%%%%
\begin{frame}{}
  \begin{center}
    关于\blue{\bf 非终结符号}的约定

    \vspace{0.30cm}
    \fig{width = 0.90\textwidth}{figs/convention-nonterminal}
  \end{center}
\end{frame}
%%%%%%%%%%%%%%%%%%%%

%%%%%%%%%%%%%%%%%%%%
% \begin{frame}{}
%   \begin{center}
%     \fig{width = 1.00\textwidth}{figs/convention-symbols}
%   \end{center}
% \end{frame}
%%%%%%%%%%%%%%%%%%%%

%%%%%%%%%%%%%%%%%%%%
\begin{frame}{}
  \begin{center}
    {\large \red{\bf 推导}} (Derivation)

    \vspace{0.50cm}
    \fig{width = 0.60\textwidth}{figs/cfg-expression}

    \vspace{0.50cm}
    推导即是将某个产生式的左边\blue{\bf 替换}成它的右边

    \vspace{1.00cm}
    每一步推导需要选择替换哪个非终结符号, 以及使用哪个产生式
  \end{center}
\end{frame}
%%%%%%%%%%%%%%%%%%%%

%%%%%%%%%%%%%%%%%%%%
\begin{frame}{}
  \begin{center}
    {\large \red{\bf 推导}} (Derivation)
  \end{center}

  \fig{width = 0.60\textwidth}{figs/cfg-expression}

  \vspace{-0.50cm}
  \[
    E \implies -E \implies -(E) \implies -(E + E) \implies -(\id + E) \implies -(\id + \id)
  \]

  \pause
  \vspace{-0.30cm}
  \begin{align*}
    &E \implies -E: \text{经过一步推导得出} \\
    &E \dplus -(\id + E): \text{经过一步或多步推导得出} \\
    &E \dstar -(\id + E): \text{经过零步或多步推导得出}
  \end{align*}

  \pause
  \vspace{-0.50cm}
  \[
    E \implies -E \implies -(E) \implies -(E + E) \implies \blue{-(E + \id)} \implies -(\id + \id)
  \]
\end{frame}
%%%%%%%%%%%%%%%%%%%%

%%%%%%%%%%%%%%%%%%%%
\begin{frame}{}
  \begin{definition}[Sentential Form; 句型]
    如果 $S \dstar \alpha$, 且 $\alpha \in (T \cup N)^{\ast}$,
    则称 $\alpha$ 是文法 $G$ 的一个\red{\bf 句型}。 
  \end{definition}

  \vspace{0.30cm}
  \fig{width = 0.60\textwidth}{figs/cfg-expression}

  \vspace{-0.80cm}
  \[
    E \implies -E \implies -(E) \implies -(E + E) 
      \implies \red{-(\id + E)} \implies \blue{-(\id + \id)}
  \]

  \pause
  \begin{definition}[Sentence; 句子]
    如果 $S \dstar w$, 且 $w \in T^{\ast}$,
    则称 $w$ 是文法 $G$ 的一个\blue{\bf 句子}。
  \end{definition}
\end{frame}
%%%%%%%%%%%%%%%%%%%%

%%%%%%%%%%%%%%%%%%%%
\begin{frame}{}
  \begin{definition}[文法$G$生成的语言 $L(G)$]
    文法 $G$ 的\red{\bf 语言} $L(G)$ 是它能推导出的\blue{所有句子}构成的集合。

    \[
      w \in L(G) \iff S \dstar w
    \]
  \end{definition}
\end{frame}
%%%%%%%%%%%%%%%%%%%%

%%%%%%%%%%%%%%%%%%%%
\begin{frame}{}
  \begin{columns}
    \column{0.50\textwidth}
      \fig{width = 0.40\textwidth}{figs/cfg-bp}

      \vspace{-0.80cm}
      \[
        L(G) = \pause \set{\text{长度$\ge 2$的已匹配括号串}}
      \]
    \column{0.50\textwidth}
      \pause
      \fig{width = 0.40\textwidth}{figs/cfg-anbn}

      \[
        L(G) = \pause \set{a^{n}b^{n} \mid n \ge 0}
      \]
  \end{columns}
\end{frame}
%%%%%%%%%%%%%%%%%%%%

%%%%%%%%%%%%%%%%%%%%
\begin{frame}{}
  \begin{center}
    字母表 $\Sigma = \set{a, b}$ 上的所有\blue{\bf 回文串} (Palindrome) 构成的语言

    \pause
    \vspace{0.30cm}
    \fig{width = 0.30\textwidth}{figs/cfg-palindrome}

    \pause
    \vspace{-0.60cm}
    \[
      P \to \epsilon \;|\; 0 \;|\; 1 \;|\; 0P0 \;|\; 1P1
    \]
  \end{center}
\end{frame}
%%%%%%%%%%%%%%%%%%%%

%%%%%%%%%%%%%%%%%%%%
\begin{frame}{}
  \begin{center}
    \red{\bf 最左 (leftmost) 推导与最右 (rightmost) 推导}

    \fig{width = 0.60\textwidth}{figs/cfg-expression}

    \vspace{-0.50cm}
    \[
      E \lm -E \lm -(E) \lm -(E + E) \lm -(\red{\id} + E) \lm -(\id + \id)
    \]

    \pause
    \vspace{-0.30cm}
    \begin{align*}
      &E \lm -E: \text{经过一步最左推导得出} \\
      &E \dpluslm -(\id + E): \text{经过一步或多步最左推导得出} \\
      &E \dstarlm -(\id + E): \text{经过零步或多步最左推导得出}
    \end{align*}

    \pause
    \vspace{-0.50cm}
    \[
        E \rm -E \rm -(E) \rm -(E + E) \rm -(E + \red{\id}) \rm -(\id + \id)
    \]
  \end{center}
\end{frame}
%%%%%%%%%%%%%%%%%%%%

%%%%%%%%%%%%%%%%%%%%
\begin{frame}{}
  \begin{definition}[Left-sentential Form; 最左句型]
    如果 $S \;\red{\dstarlm}\; \alpha$, 且 $\alpha \in (T \cup N)^{\ast}$,
    则称 $\alpha$ 是文法 $G$ 的一个\red{\bf 最左句型}。 
  \end{definition}

  \[
    E \lm -E \lm -(E) \lm -(E + E) \lm -(\red{\id} + E) \lm -(\id + \id)
  \]

  \begin{definition}[Right-sentential Form; 最右句型]
    如果 $S \;\red{\dstarrm}\; \alpha$, 且 $\alpha \in (T \cup N)^{\ast}$,
    则称 $\alpha$ 是文法 $G$ 的一个\red{\bf 最右句型}。 
  \end{definition}

  \[
    E \rm -E \rm -(E) \rm -(E + E) \rm -(E + \red{\id}) \rm -(\id + \id)
  \]
\end{frame}
%%%%%%%%%%%%%%%%%%%%

%%%%%%%%%%%%%%%%%%%%
\begin{frame}{}
  \begin{center}
    \red{\bf 语法分析树}

    \vspace{0.30cm}
    语法分析树是静态的, 它不关心动态的推导顺序

    \fig{width = 0.35\textwidth}{figs/tree-(id+id)}

    一棵语法分析树对应多个推导

    \pause
    \vspace{0.30cm}
    但是, 一棵语法分析树与\blue{\bf 最左(最右)推导}一一对应
  \end{center}
\end{frame}
%%%%%%%%%%%%%%%%%%%%