% bottom-up.tex

%%%%%%%%%%%%%%%%%%%%
\begin{frame}{}
  \begin{center}
    只考虑\red{\bf 无二义性}的文法 \\[4pt]
    这意味着, 每个句子对应唯一的一棵语法分析树

    \fig{width = 0.60\textwidth}{figs/cfg-hierarchy}

    今日份主题: \red{\bf $LR$ 语法分析器}
  \end{center}
\end{frame}
%%%%%%%%%%%%%%%%%%%%

%%%%%%%%%%%%%%%%%%%%
\begin{frame}{}
  \begin{center}
    自底向上的、\\[15pt]
    不断归约的、\\[15pt]
    基于句柄识别自动机的、\\[15pt]
    适用于\red{\bf $LR$ 文法}的、\\[15pt]
    $LR$ 语法分析器
  \end{center}
\end{frame}
%%%%%%%%%%%%%%%%%%%%

%%%%%%%%%%%%%%%%%%%%
\begin{frame}{}
  \begin{center}
    {\large \red{\bf 自底向上}构建语法分析树}

    \vspace{0.60cm}
    \blue{\bf 根节点}是文法的起始符号 $S$

    \vspace{1.00cm}
    \uncover<2->{
      每个\blue{\bf 中间非终结符节点}表示\purple{\bf 使用它的某条产生式进行归约}
    }

    \vspace{1.00cm}
    \blue{\bf 叶节点}是词法单元流 $w\$$ \\[8pt]
    仅包含终结符号与特殊的\teal{\bf 文件结束符$\$$}
  \end{center}
\end{frame}
%%%%%%%%%%%%%%%%%%%%

%%%%%%%%%%%%%%%%%%%%
\begin{frame}{}
  \begin{center}
    \blue{\bf 自顶向下的``推导''} 与 \red{\bf 自底向上的``归约''}

    \vspace{-0.30cm}
    \[
      \blue{E \rm T \rm T \ast F \rm T \ast \id \rm F \ast \id \rm \id \ast \id}
    \]

    \vspace{-0.50cm}
    \begin{columns}
      \column{0.50\textwidth}
        % cfg-expr-add-mul-mul-first-numbering.tex

\begin{empheq}[box=\widefbox]{align*}
  E &\to E + T \\[8pt]
  E &\to T \\[8pt]
  T &\to T \ast F \\[8pt]
  T &\to F \\[8pt]
  F &\to (E) \\[8pt]
  F &\to \id
\end{empheq}
      \column{0.50\textwidth}
        \fig{width = 0.50\textwidth}{figs/tree-expr-add-mul}

        \vspace{-0.80cm}
        \[
          w = \blue{\id \ast \id}
        \]
    \end{columns}

    \[
      \red{E \impliedby T \impliedby T \ast F \impliedby T \ast \id \impliedby F \ast \id
        \impliedby \id \ast \id}
    \]
  \end{center}
\end{frame}
%%%%%%%%%%%%%%%%%%%%

%%%%%%%%%%%%%%%%%%%%
\begin{frame}{}
  \begin{center}
    \blue{\bf ``推导'' ($A \to \alpha$)} 与 \red{\bf ``归约'' ($A \gets \alpha$)}

    \[
      S \triangleq \gamma_{0} \implies \dots
        \blue{\gamma_{i-1} \implies \gamma_{i} \implies \gamma_{r+1}}
        \implies \dots \implies r_{n} = w
    \]
    \[
      S \triangleq \gamma_{0} \impliedby \dots
        \red{\gamma_{i-1} \impliedby \gamma_{i} \impliedby \gamma_{r+1}}
        \impliedby \dots \impliedby r_{n} = w
    \]

    \vspace{0.80cm}
    自底向上语法分析器为输入构造\red{\bf 反向推导}
  \end{center}
\end{frame}
%%%%%%%%%%%%%%%%%%%%

%%%%%%%%%%%%%%%%%%%%
\begin{frame}{}
  \begin{center}
    \red{\bf $LR$ 语法分析器}

    \vspace{0.80cm}
    \begin{columns}
      \column{0.10\textwidth}
      \column{0.80\textwidth}
        \begin{description}
          \setlength{\itemsep}{15pt}
          \item[$L:$] \purple{\bf 从左向右} (Left-to-right) 扫描输入
          \item[$R:$] 构建\purple{\bf 反向} (Reverse) \purple{\bf 最右推导}
        \end{description}
      \column{0.10\textwidth}
    \end{columns}

    \vspace{0.80cm}
    ``反向最右推导''与``从左到右扫描''相一致
  \end{center}
\end{frame}
%%%%%%%%%%%%%%%%%%%%

%%%%%%%%%%%%%%%%%%%%
\begin{frame}{}
  \begin{center}
    \red{\bf $LR$ 语法分析器的状态}

    \vspace{0.80cm}
    在任意时刻, 语法分析树的\red{\bf 上边缘}与\blue{\bf 剩余的输入}构成当前句型

    \vspace{0.60cm}
    \fig{width = 0.90\textwidth}{figs/lr-tree-expr-add-mul}
    \[
      E \impliedby T \impliedby T \ast F \impliedby T \ast \id \impliedby F \ast \id
        \impliedby \id \ast \id
    \]

    \vspace{0.60cm}
    $LR$ 语法分析器使用\red{\bf 栈}存储语法分析树的\red{\bf 上边缘}

    \vspace{0.30cm}
    它包含了语法分析器目前所知的所有信息
  \end{center}
\end{frame}
%%%%%%%%%%%%%%%%%%%%

%%%%%%%%%%%%%%%%%%%%
\begin{frame}{}
  \begin{center}
    板书演示``\red{\bf 栈}''上操作

    \begin{columns}
      \column{0.50\textwidth}
        % cfg-expr-add-mul-mul-first-numbering.tex

\begin{empheq}[box=\widefbox]{align*}
  E &\to E + T \\[8pt]
  E &\to T \\[8pt]
  T &\to T \ast F \\[8pt]
  T &\to F \\[8pt]
  F &\to (E) \\[8pt]
  F &\to \id
\end{empheq}
      \column{0.50\textwidth}
        \fig{width = 0.50\textwidth}{figs/tree-expr-add-mul}

        \vspace{-0.80cm}
        \[
          w = \blue{\id \ast \id}
        \]
    \end{columns}

    \vspace{0.60cm}
    两大操作: \blue{\bf 移入输入符号} 与 \red{\bf 按产生式归约}

    \vspace{0.30cm}
    直到栈中仅剩开始符号 $S$, 且输入已结束, 则成功停止
  \end{center}
\end{frame}
%%%%%%%%%%%%%%%%%%%%

%%%%%%%%%%%%%%%%%%%%
\begin{frame}{}
  \begin{center}
    \red{\bf 基于栈的 $LR$ 语法分析器}

    \vspace{0.80cm}
    \red{$Q_{1}:$} 何时归约? \gray{(何时移入?)}

    \vspace{0.80cm}
    \red{$Q_{2}:$} 按哪条产生式进行归约?
  \end{center}
\end{frame}
%%%%%%%%%%%%%%%%%%%%

%%%%%%%%%%%%%%%%%%%%
\begin{frame}{}
  \begin{center}
    \red{\bf 基于栈的 $LR$ 语法分析器}

    \begin{columns}
      \column{0.50\textwidth}
        % cfg-expr-add-mul-mul-first-numbering.tex

\begin{empheq}[box=\widefbox]{align*}
  E &\to E + T \\[8pt]
  E &\to T \\[8pt]
  T &\to T \ast F \\[8pt]
  T &\to F \\[8pt]
  F &\to (E) \\[8pt]
  F &\to \id
\end{empheq}
      \column{0.50\textwidth}
        \fig{width = 0.60\textwidth}{figs/tree-expr-add-mul}
    \end{columns}

    \vspace{0.50cm}
    为什么第二个 $F$ 以 $T \ast F$ 整体被归约为 $T$?

    \vspace{0.30cm}
    这与\red{\bf 栈}的当前状态 ``$T \ast F$'' 相关
  \end{center}
\end{frame}
%%%%%%%%%%%%%%%%%%%%

%%%%%%%%%%%%%%%%%%%%
\begin{frame}{}
  \begin{center}
    \purple{\bf $LR$ 分析表}指导$LR$语法分析器

    \vspace{0.30cm}
    \fig{width = 0.60\textwidth}{figs/lr0-table-expr-add-mul}

    \vspace{0.10cm}
    在\red{\bf 当前状态(编号)}下, 面对\blue{\bf 当前文法符号}时, 该采取什么\brown{\bf 动作}

    \vspace{0.30cm}
    \purple{\action{}} 表指明动作, \purple{\goto{}} 表仅用于归约时的状态转换
  \end{center}
\end{frame}
%%%%%%%%%%%%%%%%%%%%

%%%%%%%%%%%%%%%%%%%%
\begin{frame}{}
  \begin{center}
    \fig{width = 0.60\textwidth}{figs/lr0-table-expr-add-mul}

    \vspace{0.30cm}
    % lr-actions.tex

% \usepackage{graphicx}
\begin{table}[]
  \centering
  \resizebox{0.50\textwidth}{!}{
    \renewcommand{\arraystretch}{1.2}
    \begin{tabular}{c||c}
      \hline
      $\brown{s}n$ & 移入输入符号, 并进入\teal{\bf 状态 $n$} \\ \hline
      $\brown{r}k$ & 使用\teal{\bf $k$ 号产生式}进行归约 \\ \hline
      $\brown{g}n$ & 转换到\teal{\bf 状态 $n$} \\ \hline
      $\brown{acc}$  & 成功接受, 结束 \\ \hline
      空白        & 错误  \\ \hline
    \end{tabular}}
\end{table}
  \end{center}
\end{frame}
%%%%%%%%%%%%%%%%%%%%

%%%%%%%%%%%%%%%%%%%%
\begin{frame}{}
  \begin{center}
    再次板书演示``\red{\bf 栈}''上操作: \red{\bf 移入}与\red{\bf 归约}

    \begin{columns}
      \column{0.40\textwidth}
        % cfg-expr-add-mul-mul-first-numbering.tex

\begin{empheq}[box=\widefbox]{align*}
  E &\to E + T \\[8pt]
  E &\to T \\[8pt]
  T &\to T \ast F \\[8pt]
  T &\to F \\[8pt]
  F &\to (E) \\[8pt]
  F &\to \id
\end{empheq}
      \column{0.60\textwidth}
        \fig{width = 1.00\textwidth}{figs/lr0-table-expr-add-mul}
    \end{columns}

    \[
      w = \blue{\id \ast \id \$}
    \]

    \red{\bf 栈}中存储语法分析器的\purple{\bf 状态(编号)}, ``编码''了语法分析树的上边缘
  \end{center}
\end{frame}
%%%%%%%%%%%%%%%%%%%%

%%%%%%%%%%%%%%%%%%%%
\begin{frame}{}
  \begin{center}
    % \red{\bf $LR(\ast)$语法分析器框架}

    % lr-framework.tex

\begin{algorithm}[H]
% \caption{}
% \label{alg:S}
\begin{algorithmic}[1]
  \Procedure{\blue{$LR$}}{\null}
    \State $\Call{Push}{S, \$}$ \qquad $\Call{Push}{S, s_{0}}$

    \hStatex
    \State $t \gets \Call{\purple{next-token}}{\null}$
    \While{$1$}
      \State $s \gets \Call{Top}{S}$  \Comment{\blue{$s$一定是某个状态编号, 而不是文法符号}}

      \hStatex
      \If{\red{$\action[s, t] = s_{i}$}} \Comment{\brown{移入}}
        \State $\Call{Push}{S, t}$ \qquad $\Call{Push}{S, i}$
        \State $t \gets \Call{\purple{next-token}}{\null}$
      \ElsIf{\red{$\action[s, t] = r_{j}$}} \Comment{\brown{规约; $j: A \to \alpha$}}
        \State $2 \times |\alpha|$ 次 $\Call{Pop}{S}$
        \State $s \gets \Call{Top}{S}$  \Comment{\blue{$s$一定是某个状态编号, 而不是文法符号}}
        \State $\Call{Push}{S, A}$ \qquad $\Call{Push}{S, \cyan{\goto[s, A]}}$ \Comment{\brown{转换状态}}
      \ElsIf{\red{$\action[s, t] = a$}} \Comment{\brown{接受}}
        \State {\bf break}
      \Else
        \State $\Call{\teal{error}}{\dots}$
      \EndIf
    \EndWhile
  \EndProcedure
\end{algorithmic}
\end{algorithm}
  \end{center}
\end{frame}
%%%%%%%%%%%%%%%%%%%%

%%%%%%%%%%%%%%%%%%%%
\begin{frame}{}
  \fig{width = 0.80\textwidth}{figs/lr0-id-star-id}

  \begin{center}
    $w = \blue{\id \ast \id \$}$ 的分析过程
  \end{center}
\end{frame}
%%%%%%%%%%%%%%%%%%%%

%%%%%%%%%%%%%%%%%%%%
\begin{frame}{}
  \begin{center}
    \red{\bf 如何构造 $LR$ 分析表?}

    \vspace{0.30cm}
    \fig{width = 0.60\textwidth}{figs/lr0-table-expr-add-mul}

    \vspace{0.10cm}
    在\red{\bf 当前状态(编号)}下, 面对\blue{\bf 当前文法符号}时, 该采取什么\brown{\bf 动作}
  \end{center}
\end{frame}
%%%%%%%%%%%%%%%%%%%%

%%%%%%%%%%%%%%%%%%%%
\begin{frame}{}
  \begin{center}
    \red{\bf 状态是什么? 如何跟踪状态?}

    \fig{width = 0.60\textwidth}{figs/lr0-table-expr-add-mul}

    状态是语法分析树的上边缘, 存储在栈中

    \vspace{0.30cm}
    可以用\red{\bf 自动机}跟踪状态变化 (\blue{自动机中的路径 $\Leftrightarrow$ 栈中符号/状态编号})
  \end{center}
\end{frame}
%%%%%%%%%%%%%%%%%%%%

%%%%%%%%%%%%%%%%%%%%
\begin{frame}{}
  \begin{center}
    \red{\bf 何时归约? 使用哪条产生式进行归约?}

    \fig{width = 0.60\textwidth}{figs/lr0-table-expr-add-mul}

    \blue{\bf 必要条件:} 当前状态中, 已观察到\blue{某个产生式的完整右部}

    \vspace{0.30cm}
    对于 $LR$ 文法, 这是当前\purple{\bf 唯一}的选择
  \end{center}
\end{frame}
%%%%%%%%%%%%%%%%%%%%

%%%%%%%%%%%%%%%%%%%%
\begin{frame}{}
  \begin{center}
    \red{\bf 何时归约? 使用哪条产生式进行归约?}

    \begin{definition}[句柄 (Handle)]
      在输入串的(唯一)反向最右推导中, \purple{\bf 如果}下一步是逆用产生式 $A \to \alpha$
      将$\alpha$归约为$A$, 则称 $\alpha$ 是\blue{当前句型的}\red{\bf 句柄}。
    \end{definition}

    \vspace{0.50cm}
    \fig{width = 0.80\textwidth}{figs/lr-expr-handle}

    \vspace{0.30cm}
    $LR$语法分析器的关键就是高效\red{\bf 寻找每个归约步骤所使用的句柄}。
  \end{center}
\end{frame}
%%%%%%%%%%%%%%%%%%%%

%%%%%%%%%%%%%%%%%%%%
\begin{frame}{}
  \begin{center}
    \red{\bf 句柄可能在哪里?}

    \begin{theorem}
      \red{\bf 存在}一种$LR$语法分析方法, 保证\blue{\bf 句柄总是出现在栈顶}。
    \end{theorem}

    \pause
    \fig{width = 0.80\textwidth}{figs/rm-two-steps}

    \vspace{-0.30cm}
    \begin{columns}
      \column{0.50\textwidth}
        \[
          S \dstarrm \alpha Az \dstarrm \alpha\blue{\beta By}z
            \dstarrm \alpha\beta\blue{\gamma} yz
        \]
      \column{0.52\textwidth}
        \[
          S \dstarrm \alpha BxAz \dstarrm \alpha Bx\blue{y}z \dstarrm \alpha\blue{\gamma} xyz
        \]
    \end{columns}
  \end{center}
\end{frame}
%%%%%%%%%%%%%%%%%%%%

%%%%%%%%%%%%%%%%%%%%
\begin{frame}{}
  \begin{center}
    可以用\red{\bf 自动机}跟踪状态变化 \\[5pt]
    (\blue{自动机中的路径 $\Leftrightarrow$ 栈中符号/状态编号})

    \begin{theorem}
      \red{\bf 存在}一种$LR$语法分析方法, 保证\blue{\bf 句柄总是出现在栈顶}。
    \end{theorem}

    \pause
    \vspace{0.80cm}
    希望能够在自动机的当前状态识别可能的句柄
  \end{center}
\end{frame}
%%%%%%%%%%%%%%%%%%%%

%%%%%%%%%%%%%%%%%%%%
\begin{frame}{}
  \begin{center}
    $LR(0)$ \red{\bf 句柄识别有穷状态自动机} (Handle-Finding Automaton)
    \fig{width = 0.55\textwidth}{figs/lr0-automaton-expr}

    \vspace{0.10cm}
    \blue{\bf 状态是什么?}
  \end{center}
\end{frame}
%%%%%%%%%%%%%%%%%%%%

%%%%%%%%%%%%%%%%%%%%
% \begin{frame}{}
%   \begin{center}
%     $LR(0)$ \red{\bf 句柄识别自动机}
%
%     \vspace{0.80cm}
%     为给定的\red{\bf 文法$G$}构造相应的句柄识别自动机
%
%     \vspace{0.80cm}
%     该自动机用于识别该\blue{\bf 文法$G$所允许的所有可能的句柄}
%   \end{center}
% \end{frame}
%%%%%%%%%%%%%%%%%%%%

%%%%%%%%%%%%%%%%%%%%
% \begin{frame}{}
%   \begin{center}
%     $LR(0)$ \red{\bf 句柄识别自动机}
%     \fig{width = 0.55\textwidth}{figs/lr0-automaton-expr}
%
%     \blue{\bf 状态是什么?}
%   \end{center}
% \end{frame}
%%%%%%%%%%%%%%%%%%%%

%%%%%%%%%%%%%%%%%%%%
\begin{frame}{}
  \begin{center}
    \red{\bf 状态}刻画了``当前观察到的\purple{\bf 针对所有}\blue{\bf 产生式的右部的前缀}''
  \end{center}

  \begin{definition}[$LR(0)$项 (Item)]
    文法 $G$ 的一个 \blue{\bf $LR(0)$ 项}是 $G$ 的某个产生式加上一个位于体部的\blue{\bf 点}。
  \end{definition}

  \begin{center}
    \blue{\bf 项}指明了语法分析器已经观察到了某个产生式的某个前缀

    \pause
    \[
      A \to XYZ
    \]
    \begin{align*}
      [A &\to \cdot XYZ] \\[6pt]
      [A &\to X \cdot YZ] \\[6pt]
      [A &\to XY \cdot Z] \\[6pt]
      [A &\to XYZ \cdot]
    \end{align*}

    (产生式 $A \to \epsilon$ 只有一个项 \blue{$[A \to \cdot]$})

    % \pause
    % \vspace{0.20cm}
    % \fbox{\blue{\bf 点}指示了\purple{\bf 栈顶}, 左边(与路径)是栈中内容, 右边是期望看到的文法符号串}
  \end{center}
\end{frame}
%%%%%%%%%%%%%%%%%%%%

%%%%%%%%%%%%%%%%%%%%
\begin{frame}{}
  \begin{center}
    % \red{\bf 状态}刻画了``当前观察到的\purple{\bf 针对所有}\blue{\bf 产生式的右部的前缀}''
    \fig{width = 0.50\textwidth}{figs/lr0-automaton-expr-0-4-8}

    \vspace{0.30cm}
    \fbox{\blue{\bf 点}指示了\purple{\bf 栈顶}, 左边(与路径)是栈中内容, 右边是期望看到的文法符号串}
  \end{center}
\end{frame}
%%%%%%%%%%%%%%%%%%%%

%%%%%%%%%%%%%%%%%%%%
\begin{frame}{}
  \begin{center}
    \red{\bf 状态}刻画了``当前观察到的\purple{\bf 针对所有}\blue{\bf 产生式的右部的前缀}''

    \begin{definition}[项集]
      \purple{\bf 项集}就是若干\blue{\bf 项}构成的集合。
    \end{definition}

    \vspace{0.30cm}
    因此, 句柄识别自动机的一个\red{\bf 状态}可以表示为一个\purple{\bf 项集}

    \pause
    \vspace{0.60cm}
    \begin{definition}[项集族]
      \teal{\bf 项集族}就是若干\purple{\bf 项集}构成的集合。
    \end{definition}

    \vspace{0.30cm}
    因此, 句柄识别自动机的\red{\bf 状态集}可以表示为一个\teal{\bf 项集族}
  \end{center}
\end{frame}
%%%%%%%%%%%%%%%%%%%%

%%%%%%%%%%%%%%%%%%%%
\begin{frame}{}
  \begin{center}
    $LR(0)$ \red{\bf 句柄识别自动机}
    \fig{width = 0.55\textwidth}{figs/lr0-automaton-expr}

    \blue{\bf 项、项集、项集族}
  \end{center}
\end{frame}
%%%%%%%%%%%%%%%%%%%%

%%%%%%%%%%%%%%%%%%%%
\begin{frame}{}
  \begin{center}
    \begin{definition}[增广文法 (Augmented Grammar)]
      文法 $G$ 的\red{\bf 增广文法} $G'$ 是在 $G$ 中加入产生式 \blue{$S' \to S$} 得到的文法。
    \end{definition}

    \vspace{0.50cm}
    \purple{\bf 目的:} 告诉语法分析器何时停止分析并接受输入符号串

    \vspace{0.80cm}
    当语法分析器\red{\bf 面对$\$$}且\red{\bf 要使用 $S' \to S$ 进行归约}时, 输入符号串被接受
  \end{center}
\end{frame}
%%%%%%%%%%%%%%%%%%%%

%%%%%%%%%%%%%%%%%%%%
% \begin{frame}{}
%   \begin{center}
%     $LR(0)$ \red{\bf 句柄识别自动机}
%     \fig{width = 0.55\textwidth}{figs/lr0-automaton-expr}
%
%     \teal{注: 此``接受'' (输入串) 非彼``接受'' (句柄识别自动机)}
%   \end{center}
% \end{frame}
%%%%%%%%%%%%%%%%%%%%

%%%%%%%%%%%%%%%%%%%%
\begin{frame}{}
  \begin{center}
    $LR(0)$ \red{\bf 句柄识别自动机}

    \vspace{0.30cm}
    \fig{width = 0.55\textwidth}{figs/dao}

    \vspace{0.30cm}
    \blue{\bf 初始状态是什么?}
  \end{center}
\end{frame}
%%%%%%%%%%%%%%%%%%%%

%%%%%%%%%%%%%%%%%%%%
\begin{frame}{}
  \begin{center}
    \fbox{\blue{\bf 点}指示了\purple{\bf 栈顶}, 左边(与路径)是栈中内容, 右边是期望看到的文法符号串}

    \begin{columns}
      \column{0.50\textwidth}
        % cfg-expr-add-mul-mul-first-aug-numbering.tex

\begin{empheq}[box=\widefbox]{align*}
  (0)\; E' &\to E \\[8pt]
  (1)\; E &\to E + T \\[8pt]
  (2)\; E &\to T \\[8pt]
  (3)\; T &\to T \ast F \\[8pt]
  (4)\; T &\to F \\[8pt]
  (5)\; F &\to (E) \\[8pt]
  (6)\; F &\to \id
\end{empheq}

      \column{0.50\textwidth}
        \fig{width = 0.50\textwidth}{figs/lr-expr-init-state}
    \end{columns}

    \[
      \fbox{$\closure(\set{\red{[E' \to \cdot E]}})$}
    \]
  \end{center}
\end{frame}
%%%%%%%%%%%%%%%%%%%%

%%%%%%%%%%%%%%%%%%%%
\begin{frame}{}
  \begin{center}
    $LR(0)$ \red{\bf 句柄识别自动机}

    \vspace{0.30cm}
    \fig{width = 0.55\textwidth}{figs/dao}

    \vspace{0.30cm}
    \blue{\bf 状态之间如何转移?}
  \end{center}
\end{frame}
%%%%%%%%%%%%%%%%%%%%

%%%%%%%%%%%%%%%%%%%%
\begin{frame}{}
  \begin{center}
    板书演示 $LR(0)$ \red{\bf 句柄识别自动机}的构造过程
    \fig{width = 0.55\textwidth}{figs/lr0-automaton-expr}

    \vspace{0.20cm}
    状态编号约定
  \end{center}
\end{frame}
%%%%%%%%%%%%%%%%%%%%

%%%%%%%%%%%%%%%%%%%%
\begin{frame}{}
  \fig{width = 0.80\textwidth}{figs/lr0-closure-alg}
\end{frame}
%%%%%%%%%%%%%%%%%%%%

%%%%%%%%%%%%%%%%%%%%
\begin{frame}{}
  \begin{center}
    % \fig{width = 0.70\textwidth}{figs/lr-goto-alg}

    \begin{align*}
      J = \textsc{goto}(I, \red{X}) &= \closure\Big(\Big\{
            \blue{[A \to \alpha X \cdot \beta]}
          \Big\lvert \red{[A \to \alpha \cdot X \beta]} \in I \Big\}\Big)
    \end{align*}
    \[
      (X \in N \cup T)
    \]
  \end{center}
\end{frame}
%%%%%%%%%%%%%%%%%%%%

%%%%%%%%%%%%%%%%%%%%
\begin{frame}{}
  \fig{width = 0.80\textwidth}{figs/lr0-items}

  \pause
  \vspace{0.50cm}
  \begin{center}
    接受状态: $F = \set{I \in C \mid \exists k.\; [k: A \to \alpha \cdot] \in I}$
  \end{center}
\end{frame}
%%%%%%%%%%%%%%%%%%%%

%%%%%%%%%%%%%%%%%%%%
\begin{frame}{}
  \begin{center}
    \fig{width = 0.60\textwidth}{figs/lr0-automaton-expr}

    红色框中的状态为 \red{\bf 接受状态}
  \end{center}
\end{frame}
%%%%%%%%%%%%%%%%%%%%

%%%%%%%%%%%%%%%%%%%%
\begin{frame}{}
  \begin{center}
    \blue{\bf $LR(0)$ 分析表}

    \begin{columns}
      \column{0.50\textwidth}
        \fig{width = 0.90\textwidth}{figs/lr0-automaton-expr}
      \column{0.50\textwidth}
        % lr0-table-expr.tex

% \usepackage{multirow}
% \usepackage{graphicx}
\begin{table}[h]
  \centering
  \resizebox{\textwidth}{!}{%
  \renewcommand{\arraystretch}{1.3}
  \begin{tabular}{|c|cccccc|ccc|}
    \hline
    \multirow{2}{*}{}
    & \multicolumn{6}{c|}{\blue{\action}}
    & \multicolumn{3}{c|}{\blue{\goto}}
    \\ \cline{2-10}
        & $\id$ & $+$  & $\ast$   & $($  & $)$   & $\$$  & $E$     & $T$    & $F$     \\ \hline
    0   & $s5$  &      &          & $s4$ &       &       & $g1$    & $g2$   & $g3$    \\
    1   &       & $s6$ &          &      &       & \teal{$acc$} &         &        &         \\
    \blue{\bf 2}   & $r2$  & $r2$ & \purple{\bf $s7, r2$} & $r2$ & $r2$  & $r2$  &         &        &         \\
    3   & $r4$  & $r4$ & $r4$     & $r4$ & $r4$  & $r4$  &         &        &         \\
    \red{\bf 4}   & $s5$  &      &          & $s4$ &       &       & $g8$    & $g2$   & $g3$    \\
    5   & $r6$  & $r6$ & $r6$     & $r6$ & $r6$  & $r6$  &         &        &         \\
    6   & $s5$  &      &          & $s4$ &       &       &         & $g9$   & $g3$    \\
    7   & $s5$  &      &          & $s4$ &       &       &         &        & $g10$   \\
    8   &       & $s6$ &          &      & $s11$ &       &         &        &         \\
    9   & $r1$  & $r1$ & \purple{\bf $s7, r1$} & $r1$ & $r1$  & $r1$  &         &        &         \\
    \blue{\bf 10}  & $r3$  & $r3$ & $r3$     & $r3$ & $r3$  & $r3$  &         &        &         \\
    11  & $r5$  & $r5$ & $r5$     & $r5$ & $r5$  & $r5$  &         &        &         \\ \hline
  \end{tabular}}
\end{table}
    \end{columns}

    \vspace{0.20cm}
    \red{\goto{}函数}被拆分成 \blue{\action{}表} {\small (针对终结符)}
    与 \blue{\goto{}表} {\small (针对非终结符)}
  \end{center}
\end{frame}
%%%%%%%%%%%%%%%%%%%%

%%%%%%%%%%%%%%%%%%%%
\begin{frame}{}
  \begin{center}
    \begin{enumerate}[(1)]
      \centering
      \item $\goto(I_{i}, a) = I_{j} \land a \in T \implies \action[i, a] \gets sj$
    \end{enumerate}

    \begin{columns}
      \column{0.50\textwidth}
        \fig{width = 0.70\textwidth}{figs/lr0-automaton-expr-4}
      \column{0.50\textwidth}
        % lr0-table-expr.tex

% \usepackage{multirow}
% \usepackage{graphicx}
\begin{table}[h]
  \centering
  \resizebox{\textwidth}{!}{%
  \renewcommand{\arraystretch}{1.3}
  \begin{tabular}{|c|cccccc|ccc|}
    \hline
    \multirow{2}{*}{}
    & \multicolumn{6}{c|}{\blue{\action}}
    & \multicolumn{3}{c|}{\blue{\goto}}
    \\ \cline{2-10}
        & $\id$ & $+$  & $\ast$   & $($  & $)$   & $\$$  & $E$     & $T$    & $F$     \\ \hline
    0   & $s5$  &      &          & $s4$ &       &       & $g1$    & $g2$   & $g3$    \\
    1   &       & $s6$ &          &      &       & \teal{$acc$} &         &        &         \\
    \blue{\bf 2}   & $r2$  & $r2$ & \purple{\bf $s7, r2$} & $r2$ & $r2$  & $r2$  &         &        &         \\
    3   & $r4$  & $r4$ & $r4$     & $r4$ & $r4$  & $r4$  &         &        &         \\
    \red{\bf 4}   & $s5$  &      &          & $s4$ &       &       & $g8$    & $g2$   & $g3$    \\
    5   & $r6$  & $r6$ & $r6$     & $r6$ & $r6$  & $r6$  &         &        &         \\
    6   & $s5$  &      &          & $s4$ &       &       &         & $g9$   & $g3$    \\
    7   & $s5$  &      &          & $s4$ &       &       &         &        & $g10$   \\
    8   &       & $s6$ &          &      & $s11$ &       &         &        &         \\
    9   & $r1$  & $r1$ & \purple{\bf $s7, r1$} & $r1$ & $r1$  & $r1$  &         &        &         \\
    \blue{\bf 10}  & $r3$  & $r3$ & $r3$     & $r3$ & $r3$  & $r3$  &         &        &         \\
    11  & $r5$  & $r5$ & $r5$     & $r5$ & $r5$  & $r5$  &         &        &         \\ \hline
  \end{tabular}}
\end{table}
    \end{columns}

    \vspace{0.20cm}
    \begin{enumerate}[(2)]
      \centering
      \item $\goto(I_{i}, A) = I_{j} \land A \in N \implies \action[i, A] \gets gj$
    \end{enumerate}
  \end{center}
\end{frame}
%%%%%%%%%%%%%%%%%%%%

%%%%%%%%%%%%%%%%%%%%
\begin{frame}{}
  \begin{center}
    \begin{columns}
      \column{0.50\textwidth}
        \fig{width = 0.90\textwidth}{figs/lr0-automaton-expr-2-10}
      \column{0.50\textwidth}
        % lr0-table-expr.tex

% \usepackage{multirow}
% \usepackage{graphicx}
\begin{table}[h]
  \centering
  \resizebox{\textwidth}{!}{%
  \renewcommand{\arraystretch}{1.3}
  \begin{tabular}{|c|cccccc|ccc|}
    \hline
    \multirow{2}{*}{}
    & \multicolumn{6}{c|}{\blue{\action}}
    & \multicolumn{3}{c|}{\blue{\goto}}
    \\ \cline{2-10}
        & $\id$ & $+$  & $\ast$   & $($  & $)$   & $\$$  & $E$     & $T$    & $F$     \\ \hline
    0   & $s5$  &      &          & $s4$ &       &       & $g1$    & $g2$   & $g3$    \\
    1   &       & $s6$ &          &      &       & \teal{$acc$} &         &        &         \\
    \blue{\bf 2}   & $r2$  & $r2$ & \purple{\bf $s7, r2$} & $r2$ & $r2$  & $r2$  &         &        &         \\
    3   & $r4$  & $r4$ & $r4$     & $r4$ & $r4$  & $r4$  &         &        &         \\
    \red{\bf 4}   & $s5$  &      &          & $s4$ &       &       & $g8$    & $g2$   & $g3$    \\
    5   & $r6$  & $r6$ & $r6$     & $r6$ & $r6$  & $r6$  &         &        &         \\
    6   & $s5$  &      &          & $s4$ &       &       &         & $g9$   & $g3$    \\
    7   & $s5$  &      &          & $s4$ &       &       &         &        & $g10$   \\
    8   &       & $s6$ &          &      & $s11$ &       &         &        &         \\
    9   & $r1$  & $r1$ & \purple{\bf $s7, r1$} & $r1$ & $r1$  & $r1$  &         &        &         \\
    \blue{\bf 10}  & $r3$  & $r3$ & $r3$     & $r3$ & $r3$  & $r3$  &         &        &         \\
    11  & $r5$  & $r5$ & $r5$     & $r5$ & $r5$  & $r5$  &         &        &         \\ \hline
  \end{tabular}}
\end{table}
    \end{columns}

    \vspace{0.20cm}
    \begin{enumerate}[(3)]
      \centering
      \item $[k: A \to \alpha \cdot] \in I_{i} \land \blue{A \neq S'} \implies
        \forall t \in \red{T \cup \set{\$}}.\; \action[i, t] = rk$
    \end{enumerate}
  \end{center}
\end{frame}
%%%%%%%%%%%%%%%%%%%%

%%%%%%%%%%%%%%%%%%%%
\begin{frame}{}
  \begin{center}
    \begin{columns}
      \column{0.50\textwidth}
        \fig{width = 0.50\textwidth}{figs/lr0-automaton-expr-1}
      \column{0.50\textwidth}
        % lr0-table-expr.tex

% \usepackage{multirow}
% \usepackage{graphicx}
\begin{table}[h]
  \centering
  \resizebox{\textwidth}{!}{%
  \renewcommand{\arraystretch}{1.3}
  \begin{tabular}{|c|cccccc|ccc|}
    \hline
    \multirow{2}{*}{}
    & \multicolumn{6}{c|}{\blue{\action}}
    & \multicolumn{3}{c|}{\blue{\goto}}
    \\ \cline{2-10}
        & $\id$ & $+$  & $\ast$   & $($  & $)$   & $\$$  & $E$     & $T$    & $F$     \\ \hline
    0   & $s5$  &      &          & $s4$ &       &       & $g1$    & $g2$   & $g3$    \\
    1   &       & $s6$ &          &      &       & \teal{$acc$} &         &        &         \\
    \blue{\bf 2}   & $r2$  & $r2$ & \purple{\bf $s7, r2$} & $r2$ & $r2$  & $r2$  &         &        &         \\
    3   & $r4$  & $r4$ & $r4$     & $r4$ & $r4$  & $r4$  &         &        &         \\
    \red{\bf 4}   & $s5$  &      &          & $s4$ &       &       & $g8$    & $g2$   & $g3$    \\
    5   & $r6$  & $r6$ & $r6$     & $r6$ & $r6$  & $r6$  &         &        &         \\
    6   & $s5$  &      &          & $s4$ &       &       &         & $g9$   & $g3$    \\
    7   & $s5$  &      &          & $s4$ &       &       &         &        & $g10$   \\
    8   &       & $s6$ &          &      & $s11$ &       &         &        &         \\
    9   & $r1$  & $r1$ & \purple{\bf $s7, r1$} & $r1$ & $r1$  & $r1$  &         &        &         \\
    \blue{\bf 10}  & $r3$  & $r3$ & $r3$     & $r3$ & $r3$  & $r3$  &         &        &         \\
    11  & $r5$  & $r5$ & $r5$     & $r5$ & $r5$  & $r5$  &         &        &         \\ \hline
  \end{tabular}}
\end{table}
    \end{columns}

    \vspace{0.20cm}
    \begin{enumerate}[(4)]
      \centering
      \item $[S' \to S \cdot] \in I_{i} \implies \action[i, \$] \gets acc$
    \end{enumerate}
  \end{center}
\end{frame}
%%%%%%%%%%%%%%%%%%%%

%%%%%%%%%%%%%%%%%%%%
\begin{frame}{}
  \begin{center}
    \blue{\bf $LR(0)$ 分析表构建规则}

    \vspace{0.60cm}
    \begin{enumerate}[(1)]
      \setlength{\itemsep}{25pt}
      \item $\goto(I_{i}, a) = I_{j} \land a \in T \implies \action[i, a] \gets sj$
      \item $\goto(I_{i}, A) = I_{j} \land A \in N \implies \action[i, A] \gets gj$
      \item $[k: A \to \alpha \cdot] \in I_{i} \land \blue{A \neq S'} \implies
        \forall t \in \red{T \cup \set{\$}}.\; \action[i, t] = rk$
      \item $[S' \to S \cdot] \in I_{i} \implies \action[i, \$] \gets acc$
    \end{enumerate}
  \end{center}
\end{frame}
%%%%%%%%%%%%%%%%%%%%

%%%%%%%%%%%%%%%%%%%%
\begin{frame}{}
  \begin{definition}[$LR(0)$文法]
    如果文法 $G$ 的\red{\bf $LR(0)$分析表}是\blue{\bf 无冲突}的,
    则 $G$ 是 $LR(0)$ 文法。
  \end{definition}

  \begin{center}
    \begin{columns}
      \column{0.25\textwidth}
      \column{0.50\textwidth}
        % lr0-table-expr.tex

% \usepackage{multirow}
% \usepackage{graphicx}
\begin{table}[h]
  \centering
  \resizebox{\textwidth}{!}{%
  \renewcommand{\arraystretch}{1.3}
  \begin{tabular}{|c|cccccc|ccc|}
    \hline
    \multirow{2}{*}{}
    & \multicolumn{6}{c|}{\blue{\action}}
    & \multicolumn{3}{c|}{\blue{\goto}}
    \\ \cline{2-10}
        & $\id$ & $+$  & $\ast$   & $($  & $)$   & $\$$  & $E$     & $T$    & $F$     \\ \hline
    0   & $s5$  &      &          & $s4$ &       &       & $g1$    & $g2$   & $g3$    \\
    1   &       & $s6$ &          &      &       & \teal{$acc$} &         &        &         \\
    \blue{\bf 2}   & $r2$  & $r2$ & \purple{\bf $s7, r2$} & $r2$ & $r2$  & $r2$  &         &        &         \\
    3   & $r4$  & $r4$ & $r4$     & $r4$ & $r4$  & $r4$  &         &        &         \\
    \red{\bf 4}   & $s5$  &      &          & $s4$ &       &       & $g8$    & $g2$   & $g3$    \\
    5   & $r6$  & $r6$ & $r6$     & $r6$ & $r6$  & $r6$  &         &        &         \\
    6   & $s5$  &      &          & $s4$ &       &       &         & $g9$   & $g3$    \\
    7   & $s5$  &      &          & $s4$ &       &       &         &        & $g10$   \\
    8   &       & $s6$ &          &      & $s11$ &       &         &        &         \\
    9   & $r1$  & $r1$ & \purple{\bf $s7, r1$} & $r1$ & $r1$  & $r1$  &         &        &         \\
    \blue{\bf 10}  & $r3$  & $r3$ & $r3$     & $r3$ & $r3$  & $r3$  &         &        &         \\
    11  & $r5$  & $r5$ & $r5$     & $r5$ & $r5$  & $r5$  &         &        &         \\ \hline
  \end{tabular}}
\end{table}
      \column{0.25\textwidth}
    \end{columns}

    \vspace{0.20cm}
    非$LR(0)$分析表/文法
  \end{center}
\end{frame}
%%%%%%%%%%%%%%%%%%%%

%%%%%%%%%%%%%%%%%%%%
\begin{frame}{}
  \begin{center}
    $LR(0)$ 分析表每一行 (状态) \red{\bf 所选用的归约产生式是相同的}

    \begin{columns}
      \column{0.25\textwidth}
      \column{0.50\textwidth}
        % lr0-table-expr.tex

% \usepackage{multirow}
% \usepackage{graphicx}
\begin{table}[h]
  \centering
  \resizebox{\textwidth}{!}{%
  \renewcommand{\arraystretch}{1.3}
  \begin{tabular}{|c|cccccc|ccc|}
    \hline
    \multirow{2}{*}{}
    & \multicolumn{6}{c|}{\blue{\action}}
    & \multicolumn{3}{c|}{\blue{\goto}}
    \\ \cline{2-10}
        & $\id$ & $+$  & $\ast$   & $($  & $)$   & $\$$  & $E$     & $T$    & $F$     \\ \hline
    0   & $s5$  &      &          & $s4$ &       &       & $g1$    & $g2$   & $g3$    \\
    1   &       & $s6$ &          &      &       & \teal{$acc$} &         &        &         \\
    \blue{\bf 2}   & $r2$  & $r2$ & \purple{\bf $s7, r2$} & $r2$ & $r2$  & $r2$  &         &        &         \\
    3   & $r4$  & $r4$ & $r4$     & $r4$ & $r4$  & $r4$  &         &        &         \\
    \red{\bf 4}   & $s5$  &      &          & $s4$ &       &       & $g8$    & $g2$   & $g3$    \\
    5   & $r6$  & $r6$ & $r6$     & $r6$ & $r6$  & $r6$  &         &        &         \\
    6   & $s5$  &      &          & $s4$ &       &       &         & $g9$   & $g3$    \\
    7   & $s5$  &      &          & $s4$ &       &       &         &        & $g10$   \\
    8   &       & $s6$ &          &      & $s11$ &       &         &        &         \\
    9   & $r1$  & $r1$ & \purple{\bf $s7, r1$} & $r1$ & $r1$  & $r1$  &         &        &         \\
    \blue{\bf 10}  & $r3$  & $r3$ & $r3$     & $r3$ & $r3$  & $r3$  &         &        &         \\
    11  & $r5$  & $r5$ & $r5$     & $r5$ & $r5$  & $r5$  &         &        &         \\ \hline
  \end{tabular}}
\end{table}
      \column{0.25\textwidth}
    \end{columns}

    \vspace{0.20cm}
    \red{\bf 归约}时不需要向前看, 这就是 \red{\bf ``$0$''} 的含义
  \end{center}
\end{frame}
%%%%%%%%%%%%%%%%%%%%

%%%%%%%%%%%%%%%%%%%%
\begin{frame}{}
  \begin{center}
    \red{\bf $LR(0)$ 语法分析器}

    \vspace{0.80cm}
    \begin{columns}
      \column{0.10\textwidth}
      \column{0.80\textwidth}
        \begin{description}
          \setlength{\itemsep}{15pt}
          \item[$L:$] \purple{\bf 从左向右} (Left-to-right) 扫描输入
          \item[$R:$] 构建\purple{\bf 反向} (Reverse) \purple{\bf 最右推导}
          \item[$0:$] \purple{\bf 归约}时无需向前看
        \end{description}
      \column{0.10\textwidth}
    \end{columns}
  \end{center}
\end{frame}
%%%%%%%%%%%%%%%%%%%%

%%%%%%%%%%%%%%%%%%%%
\begin{frame}{}
  \begin{center}
    \red{\bf $LR(0)$ 自动机}与\blue{\bf 栈}之间的互动关系

    \vspace{0.80cm}
    向前走 $\Leftrightarrow$ 移入

    \vspace{0.50cm}
    回溯   $\Leftrightarrow$ 归约

    \vspace{1.00cm}
    \purple{\bf 自动机才是本质, 栈是实现方式}

    \vspace{0.20cm}
    (用栈记住``来时的路'', 以便回溯)
  \end{center}
\end{frame}
%%%%%%%%%%%%%%%%%%%%

%%%%%%%%%%%%%%%%%%%%
% \begin{frame}{}
%   \begin{center}
%     $LR(0)$ \red{\bf 句柄识别自动机}
%     \fig{width = 0.55\textwidth}{figs/lr0-automaton-expr}
%
%     \blue{\bf $Q:$ 为什么是个有穷状态自动机?}
%   \end{center}
% \end{frame}
%%%%%%%%%%%%%%%%%%%%

%%%%%%%%%%%%%%%%%%%%
\begin{frame}{}
  \begin{center}
    \red{$SLR(1)$ 分析表}

    \fig{width = 0.60\textwidth}{figs/lr0-table-expr-add-mul}

    \blue{\bf 归约:}
    \begin{enumerate}[(3)]
      \centering
      \item $[k: A \to \alpha \cdot] \in I_{i} \land \blue{A \neq S'} \implies
        \forall t \in \red{\follow(A)}.\; \action[i, t] = rk$
    \end{enumerate}
  \end{center}
\end{frame}
%%%%%%%%%%%%%%%%%%%%

%%%%%%%%%%%%%%%%%%%%
\begin{frame}{}
  \begin{definition}[$SLR(1)$文法]
    如果文法 $G$ 的\red{\bf $SLR(1)$分析表}是\blue{\bf 无冲突}的,
    则 $G$ 是 $SLR(1)$ 文法。
  \end{definition}

  \vspace{0.30cm}
  \begin{center}
    \blue{\bf 无冲突:} \action{}表中每个单元格最多只有一种动作 \\[8pt]

    \fig{width = 0.50\textwidth}{figs/lr0-table-expr-add-mul}

    \blue{\bf 两类可能的冲突:} ``移入/归约''冲突、``归约/归约''冲突
  \end{center}
\end{frame}
%%%%%%%%%%%%%%%%%%%%

%%%%%%%%%%%%%%%%%%%%
\begin{frame}{}
  \begin{center}
    非 $SLR(1)$ 文法举例
  \end{center}

  \begin{columns}
    \column{0.50\textwidth}
      \fig{width = 0.60\textwidth}{figs/cfg-LR}
    \column{0.50\textwidth}
      \fig{width = 1.00\textwidth}{figs/lr0-automaton-LR-2}
  \end{columns}

  \vspace{0.30cm}
  \[
    [S \to L \cdot = R] \in I_{2} \implies \action(I_{2}, =) \gets s6
  \]

  \[
    =\; \in \follow(R) \implies \action(I_{2}, =) \gets r5
  \]
\end{frame}
%%%%%%%%%%%%%%%%%%%%

%%%%%%%%%%%%%%%%%%%%
\begin{frame}{}
  \begin{center}
    即使考虑了\red{$=\; \in \follow(A)$}, 对该文法来说仍然不够

    \vspace{0.20cm}
    因为, 这仅仅说明在\red{\bf 某个}句型中, $a$可以跟在$A$后面

    \fig{width = 0.80\textwidth}{figs/lr0-automaton-SLR}

    该文法没有\blue{\bf 以 $R = \cdots$ 开头}的最右句型
    % \pause
    % \vspace{0.20cm}
    % 不仅要考虑\red{$a \in \follow(A)$}, 还要考虑``\blue{$Aa$}''所处的当前状态 ($\blue{\beta} Aa$)
  \end{center}
\end{frame}
%%%%%%%%%%%%%%%%%%%%

%%%%%%%%%%%%%%%%%%%%
\begin{frame}{}
  \begin{center}
    希望$LR$语法分析器的每个状态能\blue{\bf 尽可能精确}地

    \vspace{0.30cm}
    指明\red{\bf 哪些输入符号可以跟在句柄$A \to \alpha$的后面}

    \pause
    \vspace{1.00cm}
    在 $LR(0)$ 自动机中, 某个项集 \purple{$I_{j}$} 中包含 \purple{$[A \to \alpha \cdot]$}

    \vspace{0.50cm}
    则在之前的某个项集 \teal{$I_{i}$} 中包含 \teal{$[B \to \beta \cdot A \gamma]$}

    \vspace{0.50cm}
    \red{\fbox{这表明只有$a \in \first(\gamma)$时, 才可以进行 $A \to \alpha$ 归约}}

    \pause
    \vspace{0.80cm}
    但是, 对$I_{i}$求闭包时, 仅得到 $[A \to \cdot \alpha]$, 丢失了 $\first(\gamma)$ 信息
  \end{center}
\end{frame}
%%%%%%%%%%%%%%%%%%%%

%%%%%%%%%%%%%%%%%%%%
\begin{frame}{}
  \begin{center}
    \begin{definition}[$LR(1)$项 (Item)]
      \[
        [A \to \alpha \cdot \beta, \red{a}] \qquad (a \in T \cup \set{\$})
      \]

      此处, $a$ 是\red{\bf 向前看符号}, 数量为 \red{\bf 1}.
    \end{definition}

    \pause
    \vspace{0.80cm}
    思想: $\alpha$ 在栈顶, 且输入中开头的是可以从$\beta a$推导出的符号串
  \end{center}
\end{frame}
%%%%%%%%%%%%%%%%%%%%

%%%%%%%%%%%%%%%%%%%%
\begin{frame}{}
  \begin{center}
    $LR(1)$\red{\bf 句柄识别自动机}

    \[
      [A \to \alpha \cdot B \beta, \red{a}] \in I \qquad (a \in T \cup \set{\$})
    \]

    \fig{width = 0.80\textwidth}{figs/lr1-closure}
    \[
      \forall \blue{b \in \first(\beta a)}.\; [B \to \cdot \gamma, b] \in I
    \]
  \end{center}
\end{frame}
%%%%%%%%%%%%%%%%%%%%

%%%%%%%%%%%%%%%%%%%%
\begin{frame}{}
  \begin{center}
    $LR(1)$\red{\bf 句柄识别自动机}

    \vspace{0.60cm}
    \fig{width = 0.80\textwidth}{figs/lr1-goto}
    \begin{align*}
      J = \textsc{goto}(I, \red{X}) &= \closure\Big(\Big\{
            \blue{[A \to \alpha X \cdot \beta]}
            \Big\lvert \red{[A \to \alpha \cdot X \beta]} \in I \Big\}\Big)
    \end{align*}
    \[
      (X \in N \cup T)
    \]
  \end{center}
\end{frame}
%%%%%%%%%%%%%%%%%%%%

%%%%%%%%%%%%%%%%%%%%
\begin{frame}{}
  \begin{center}
    $LR(1)$\red{\bf 句柄识别自动机}

    \vspace{0.60cm}
    \fig{width = 0.80\textwidth}{figs/lr1-items}

    \vspace{0.60cm}
    初始状态: $\closure({[S' \to \cdot S, \red{\$}]})$
  \end{center}
\end{frame}
%%%%%%%%%%%%%%%%%%%%

%%%%%%%%%%%%%%%%%%%%
\begin{frame}{}
  \begin{center}
    板书演示: $LR(1)$自动机的构造过程

    \begin{columns}
      \column{0.30\textwidth}
        \fig{width = 0.70\textwidth}{figs/cfg-SCC}

        \uncover<2->{
        \begin{table}
          \begin{tabular}{c|c|c}
            \hline
              & \first & \follow \\ \hline \hline
            $S$ & $\set{c, d}$ & $\$$ \\ \hline
            $C$ & $\set{c, d}$ & $\set{c, d, \$}$ \\ \hline
          \end{tabular}}
    \end{table}
      \column{0.70\textwidth}
        \fig{width = 1.00\textwidth}{figs/lr1-automaton-SCC}
    \end{columns}
  \end{center}
\end{frame}
%%%%%%%%%%%%%%%%%%%%

%%%%%%%%%%%%%%%%%%%%
\begin{frame}{}
  \begin{center}
    \red{\bf $LR(1)$分析表构建规则}

    \vspace{0.60cm}
    \begin{enumerate}[(1)]
      \setlength{\itemsep}{25pt}
      \item $\goto(I_{i}, a) = I_{j} \land a \in T \implies \action[i, a] \gets sj$
      \item $\goto(I_{i}, A) = I_{j} \land A \in T \implies \action[i, A] \gets gj$
      \item $[k: A \to \alpha \cdot, \red{a}] \in I_{i} \land \blue{A \neq S'} \implies
        \action[i, \red{a}] = rk$
      \item $[S' \to S \cdot, \purple{\$}] \in I_{i} \implies \action[i, \$] \gets acc$
    \end{enumerate}

    \pause
    \vspace{0.20cm}
    \begin{definition}[$LR(1)$文法]
      如果文法 $G$ 的\red{\bf $LR(1)$分析表}是\blue{\bf 无冲突}的,
      则 $G$ 是 $LR(1)$ 文法。
    \end{definition}
  \end{center}
\end{frame}
%%%%%%%%%%%%%%%%%%%%

%%%%%%%%%%%%%%%%%%%%
\begin{frame}{}
  \begin{center}
    \begin{columns}
      \column{0.60\textwidth}
        \fig{width = 1.10\textwidth}{figs/lr1-automaton-SCC}
      \column{0.40\textwidth}
        \fig{width = 1.00\textwidth}{figs/lr1-table-SCC}
    \end{columns}

    \vspace{0.60cm}
    $LR(1)$ 通过\red{\bf 不同的向前看符号}, 区分了状态对 $(3, 6)$, $(4, 7)$ 与 $(8, 9)$
  \end{center}
\end{frame}
%%%%%%%%%%%%%%%%%%%%

%%%%%%%%%%%%%%%%%%%%
\begin{frame}{}
  \begin{center}
    总结: $LR(0)$、$SLR(1)$、$LR(1)$ 的\red{\bf 归约}条件

    \[
      [k: A \to \alpha \cdot] \in I_{i} \land A \neq S' \implies
        \forall t \in \red{T \cup \set{\$}}.\; \action[i, t] = rk
    \]

    \[
      [k: A \to \alpha \cdot] \in I_{i} \land A \neq S' \implies
        \forall t \in \red{\follow(A)}.\; \action[i, t] = rk
    \]

    \[
      [k: A \to \alpha \cdot, \red{a}] \in I_{i} \land A \neq S' \implies
        \action[i, \red{a}] = rk
    \]
  \end{center}
\end{frame}
%%%%%%%%%%%%%%%%%%%%

%%%%%%%%%%%%%%%%%%%%
\begin{frame}{}
  \begin{center}
    $LR(1)$虽然强大, 但是生成的$LR(1)$分析表可能过大, 状态过多

    \begin{columns}
      \column{0.60\textwidth}
        \fig{width = 1.10\textwidth}{figs/lr1-automaton-SCC}
      \column{0.40\textwidth}
        \fig{width = 1.00\textwidth}{figs/lr1-table-SCC}
    \end{columns}

    \pause
    \vspace{0.10cm}
    \red{$LALR(1):$} 合并具有相同\blue{\bf 核心 $LR(0)$}项的状态 (忽略不同的向前看符号)
  \end{center}
\end{frame}
%%%%%%%%%%%%%%%%%%%%

%%%%%%%%%%%%%%%%%%%%
\begin{frame}{}
  \begin{center}
    \begin{columns}
      \column{0.50\textwidth}
        \fig{width = 0.80\textwidth}{figs/lr1-table-SCC}
      \column{0.50\textwidth}
        \fig{width = 0.95\textwidth}{figs/lalr-table-SCC}
    \end{columns}

    \pause
    \vspace{0.60cm}
    \red{$Q:$ \goto{}函数怎么办?}

    \pause
    \vspace{0.50cm}
    \teal{$A:$ 可以合并的状态的\goto{}目标(状态)一定也是可以合并的}
  \end{center}
\end{frame}
%%%%%%%%%%%%%%%%%%%%

%%%%%%%%%%%%%%%%%%%%
\begin{frame}{}
  \begin{center}
    \red{$Q:$ 对于 $LR(1)$ 文法, 合并得到的$LALR(1)$ 分析表是否会引入冲突?}

    \pause
    \begin{theorem}
      $LALR(1)$ 分析表\blue{\bf 不会}引入移入/归约冲突。
    \end{theorem}

    \pause
    \vspace{0.50cm}
    \purple{\bf 反证法}

    \vspace{0.30cm}
    假设合并后出现 $[A \to \alpha \cdot, a]$ 与 $[B \to \beta \cdot a \gamma, b]$

    \vspace{0.60cm}
    则在$LR(1)$自动机中,

    \vspace{0.20cm}
    存在某状态同时包含 $[A \to \alpha \cdot, a]$ 与 $[B \to \beta \cdot a \gamma, c]$
  \end{center}
\end{frame}
%%%%%%%%%%%%%%%%%%%%

%%%%%%%%%%%%%%%%%%%%
\begin{frame}{}
  \begin{center}
    \red{$Q:$ 对于 $LR(1)$ 文法, 合并得到的$LALR(1)$ 分析表是否会引入冲突?}

    \begin{theorem}
      $LALR(1)$ 分析表\blue{\bf 可能会}引入归约/归约冲突。
    \end{theorem}

    \pause
    \[
      L(G) = \set{acd, ace, bcd, bce}
    \]
    \fig{width = 0.50\textwidth}{figs/lalr-rr-conflict}

    \pause
    \vspace{-0.30cm}
    \begin{gather*}
      \set{[A \to c \cdot, d], [B \to c \cdot, e]} \\
      \set{[A \to c \cdot, e], [B \to c \cdot, d]} \\[5pt]
      \set{[A \to c \cdot, \red{d/e}], [B \to c \cdot, \red{d/e}]} \\
    \end{gather*}
  \end{center}
\end{frame}
%%%%%%%%%%%%%%%%%%%%

%%%%%%%%%%%%%%%%%%%%
\begin{frame}{}
  \begin{center}
    \red{\bf 好消息:} 善用 $LR$ 语法分析器, 处理\blue{\bf 二义性}文法

    \fig{width = 0.50\textwidth}{figs/girl-woman}
  \end{center}
\end{frame}
%%%%%%%%%%%%%%%%%%%%

%%%%%%%%%%%%%%%%%%%%
\begin{frame}{}
  \begin{center}
    \red{\bf 表达式文法}

    \begin{columns}
      \column{0.50\textwidth}
        % cfg-expr-add-mul.tex

\begin{empheq}[box=\widefbox]{align*}
  E \to E + E \mid E \ast E \mid (E) \mid \id
\end{empheq}
      \column{0.50\textwidth}
        % cfg-expr-add-mul-mul-first.tex

\begin{empheq}[box=\widefbox]{align*}
  E &\to E + \red{T} \mid \red{T} \\[8pt]
  T &\to \red{T \ast F} \mid F \\[8pt]
  F &\to \purple{(E)} \mid \id
\end{empheq}
    \end{columns}

    % cfg-expr-add-mul-no-left-recursion.tex

\begin{empheq}[box=\widefbox]{align*}
  E &\to T E' \\[8pt]
  E' &\to +\; T E' \mid \epsilon \\[8pt]
  T &\to F T' \\[8pt]
  T' &\to \ast\; F T' \mid \epsilon \\[8pt]
  F &\to (E) \mid \id \mid \num
\end{empheq}
  \end{center}
\end{frame}
%%%%%%%%%%%%%%%%%%%%

%%%%%%%%%%%%%%%%%%%%
\begin{frame}{}
  \begin{center}
    \red{\bf 表达式文法:} 使用 $SLR(1)$ 语法分析方法

    \vspace{-0.50cm}
    % cfg-expr-add-mul.tex

\begin{empheq}[box=\widefbox]{align*}
  E \to E + E \mid E \ast E \mid (E) \mid \id
\end{empheq}

    \vspace{-0.60cm}
    \begin{columns}
      \column{0.60\textwidth}
        \fig{width = 0.60\textwidth}{figs/lr0-automaton-expr-E}
      \column{0.45\textwidth}
        \[
          \set{+, \ast} \subseteq \follow(E)
        \]

        \pause
        \begin{center}
          考虑到\blue{\bf 结合性与优先级}:
          \fig{width = 1.00\textwidth}{figs/slr-table-expr-E}
        \end{center}
    \end{columns}
  \end{center}
\end{frame}
%%%%%%%%%%%%%%%%%%%%

%%%%%%%%%%%%%%%%%%%%
\begin{frame}{}
  \begin{center}
    \red{\bf 条件语句文法}

    \begin{columns}
      \column{0.50\textwidth}
        \fig{width = 1.00\textwidth}{figs/cfg-if-then-else}
      \column{0.50\textwidth}
        \fig{width = 0.80\textwidth}{figs/cfg-if-then-else-short-2}
    \end{columns}
  \end{center}
\end{frame}
%%%%%%%%%%%%%%%%%%%%

%%%%%%%%%%%%%%%%%%%%
\begin{frame}{}
  \begin{center}
    \red{\bf 条件语句文法:} 使用$SLR(1)$语法分析方法
    \fig{width = 0.30\textwidth}{figs/cfg-if-then-else-short-2}

    \begin{columns}
      \column{0.50\textwidth}
        \fig{width = 1.00\textwidth}{figs/lr0-items-if-then-else}
        \[
          e \in \follow(S)
        \]
      \column{0.50\textwidth}
        \fig{width = 1.00\textwidth}{figs/slr-table-if-then-else-short-2}
        \[
          \action[4, e] = s5
        \]
    \end{columns}
  \end{center}
\end{frame}
%%%%%%%%%%%%%%%%%%%%

%%%%%%%%%%%%%%%%%%%%
\begin{frame}{}
  \fig{width = 0.60\textwidth}{figs/cfg-hierarchy}
\end{frame}
%%%%%%%%%%%%%%%%%%%%