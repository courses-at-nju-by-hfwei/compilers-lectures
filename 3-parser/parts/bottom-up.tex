% bottom-up.tex

%%%%%%%%%%%%%%%%%%%%
\begin{frame}{}
  \begin{center}
    只考虑\red{\bf 无二义性}的文法 \\[4pt]
    这意味着, 每个句子对应唯一的一棵语法分析树

    \fig{width = 0.60\textwidth}{figs/cfg-hierarchy}

    今日份主题: \red{\bf $LR(1)$ ($LR(0)$) 语法分析器}
  \end{center}
\end{frame}
%%%%%%%%%%%%%%%%%%%%

%%%%%%%%%%%%%%%%%%%%
\begin{frame}{}
  \begin{center}
    自顶向下的、\\[15pt]
    不断规约的、\\[15pt]
    基于句柄查找自动机的、\\[15pt]
    适用于\red{\bf $LR(1)$ ($LR(0)$) 文法}的、\\[15pt]
    $LR(1)$ ($LR(0)$) 语法分析器
  \end{center}
\end{frame}
%%%%%%%%%%%%%%%%%%%%

%%%%%%%%%%%%%%%%%%%%
\begin{frame}{}
  \begin{center}
    {\large \red{\bf 自底向上}构建语法分析树}

    \vspace{0.60cm}
    \blue{\bf 根节点}是文法的起始符号 $S$

    \vspace{1.00cm}
    \uncover<2->{
      每个\blue{\bf 中间非终结符节点}表示\purple{\bf 使用它的某条产生式进行归约}
    }

    \vspace{1.00cm}
    \blue{\bf 叶节点}是词法单元流 $w\$$ \\[8pt]
    仅包含终结符号与特殊的\teal{\bf 文件结束符$\$$}
  \end{center}
\end{frame}
%%%%%%%%%%%%%%%%%%%%

%%%%%%%%%%%%%%%%%%%%
\begin{frame}{}
  \begin{center}
    \red{\bf ``推导''与``归约''}

    \vspace{0.60cm}
    从\blue{\bf 产生式}的角度看, 是``推导''
    \[
      A \to \alpha
    \]

    从\blue{\bf 输入}的角度看, 是``归约''
    \[
      A \gets \alpha
    \]
    \[
      S \triangleq \gamma_{0} \implies \dots
        \red{\gamma_{i-1} \implies \gamma_{i} \implies \gamma_{r+1}}
        \implies \dots \implies r_{n} = w
    \]
    \[
      S \triangleq \gamma_{0} \impliedby \dots
        \red{\gamma_{i-1} \impliedby \gamma_{i} \impliedby \gamma_{r+1}}
        \impliedby \dots \impliedby r_{n} = w
    \]

    \vspace{0.50cm}
    自底向上语法分析器为输入构造\red{\bf 反向推导}
  \end{center}
\end{frame}
%%%%%%%%%%%%%%%%%%%%

%%%%%%%%%%%%%%%%%%%%
\begin{frame}{}
  \begin{center}
    {\large \red{\bf $LR(\ast)$ 语法分析器}}

    \vspace{0.80cm}
    \begin{columns}
      \column{0.10\textwidth}
      \column{0.80\textwidth}
        \begin{description}
          \setlength{\itemsep}{15pt}
          \item[$L:$] \purple{\bf 从左向右} (left-to-right) 扫描输入
          \item[$R:$] 构建\purple{\bf 反向 (reverse) 最右 (leftmost) 推导}
        \end{description}
      \column{0.10\textwidth}
    \end{columns}

    \vspace{0.80cm}
    在最右推导中, 最左叶节点最后才被处理

    \vspace{0.50cm}
    在反向最右推导中, 最左叶节点最先被处理 (与从左到右扫描一致)
  \end{center}
\end{frame}
%%%%%%%%%%%%%%%%%%%%

%%%%%%%%%%%%%%%%%%%%
% \begin{frame}{}
%   \begin{center}
%     % cfg-S-L-x.tex

\begin{empheq}[box=\widefbox]{align*}
  (0)\; S' &\to S\$ \\[8pt]
  (1)\; S &\to (L) \\[8pt]
  (2)\; S &\to x \\[8pt]
  (3)\; L &\to S \\[8pt]
  (4)\; L &\to L, S
\end{empheq}
%     \[
%       w = \blue{(x, x, x)}
%     \]
%   \end{center}
% \end{frame}
%%%%%%%%%%%%%%%%%%%%

%%%%%%%%%%%%%%%%%%%%
\begin{frame}{}
  \begin{center}
    % cfg-expr-add-mul-mul-first-numbering.tex

\begin{empheq}[box=\widefbox]{align*}
  E &\to E + T \\[8pt]
  E &\to T \\[8pt]
  T &\to T \ast F \\[8pt]
  T &\to F \\[8pt]
  F &\to (E) \\[8pt]
  F &\to \id
\end{empheq}
    \[
      w = \blue{\id \ast \id}
    \]
  \end{center}
\end{frame}
%%%%%%%%%%%%%%%%%%%%

%%%%%%%%%%%%%%%%%%%%
\begin{frame}{}
  \begin{center}
    板书演示``\red{\bf 栈}''上操作: \red{\bf 移入}与\red{\bf 规约}

    \fig{width = 0.30\textwidth}{figs/tree-expr-add-mul}
    \vspace{-0.30cm}
    \[
      w = \blue{\id \ast \id}
    \]
  \end{center}
\end{frame}
%%%%%%%%%%%%%%%%%%%%

% \fig{width = 0.80\textwidth}{figs/tree-S-L-x-partial.png}
% \fig{width = 0.70\textwidth}{figs/stack-S-L-x}

%%%%%%%%%%%%%%%%%%%%
\begin{frame}{}
  \begin{center}
    部分构建的语法分析树的\red{\bf 上边缘}与\blue{\bf 剩余的输入}构成当前句型

    \vspace{0.60cm}
    \fig{width = 0.90\textwidth}{figs/lr-tree-expr-add-mul}

    \vspace{0.60cm}
    $LR$ 语法分析器使用\purple{\bf 栈}存储语法分析树的\purple{\bf 上边缘}
  \end{center}
\end{frame}
%%%%%%%%%%%%%%%%%%%%

%%%%%%%%%%%%%%%%%%%%
\begin{frame}{}
  \begin{center}
    \begin{definition}[句柄 (Handle)]
      在输入串的(唯一)反向最右推导中, \purple{\bf 如果}下一步是逆用产生式 $A \to \alpha$
      将$\alpha$规约为$A$, 则称 $\alpha$ 是\blue{当前句型的}\red{\bf 句柄}。
    \end{definition}

    \vspace{0.30cm}
    \fig{width = 0.80\textwidth}{figs/lr-expr-handle}

    \vspace{0.30cm}
    $LR$语法分析器的关键就是高效\red{\bf 寻找每个归约步骤所使用的句柄}。
  \end{center}
\end{frame}
%%%%%%%%%%%%%%%%%%%%

%%%%%%%%%%%%%%%%%%%%
\begin{frame}{}
  \begin{center}
    \red{\bf 句柄可能在哪里?}

    \begin{theorem}
      \red{\bf 存在}一种$LR$语法分析方法, 保证\blue{\bf 句柄总是出现在栈顶}。
    \end{theorem}

    \pause
    \fig{width = 0.80\textwidth}{figs/rm-two-steps}

    \vspace{-0.30cm}
    \begin{columns}
      \column{0.50\textwidth}
        \[
          S \dstarrm \alpha Az \dstarrm \alpha\blue{\beta By}z
            \dstarrm \alpha\beta\blue{\gamma} yz
        \]
      \column{0.50\textwidth}
        \[
          S \dstarrm \alpha BxAz \dstarrm \alpha Bx\blue{y}z \dstarrm \alpha\blue{\gamma} xyz
        \]
    \end{columns}
  \end{center}
\end{frame}
%%%%%%%%%%%%%%%%%%%%

%%%%%%%%%%%%%%%%%%%%
\begin{frame}{}
  \begin{center}
    \begin{theorem}
      \red{\bf 存在}一种$LR$语法分析方法, 保证\blue{\bf 句柄总是出现在栈顶}。
    \end{theorem}

    \vspace{0.60cm}
    $LR$语法分析器在进行一次归约后, 接着移入零个或多个符号, 继续寻找下一个句柄
  \end{center}
\end{frame}
%%%%%%%%%%%%%%%%%%%%

%%%%%%%%%%%%%%%%%%%%
\begin{frame}{}
  \begin{center}
    为什么第一个 $F$ 被直接归约为 $T$, \\[6pt]
    而第二个 $F$ 则与 $T \ast F$ 一起被归约为 $T$?

    \fig{width = 0.25\textwidth}{figs/tree-expr-add-mul}

    \pause
    \vspace{0.30cm}
    \red{\bf 栈}刻画了语法分析器目前所知的所有信息, \\[6pt]
    \red{隐含}了语法分析器的当前状态
  \end{center}
\end{frame}
%%%%%%%%%%%%%%%%%%%%

%%%%%%%%%%%%%%%%%%%%
\begin{frame}{}
  \begin{center}
    \fig{width = 0.60\textwidth}{figs/lr0-table-expr-add-mul}

    \vspace{0.60cm}
    \purple{\bf $LR(0)$ 分析表}指导$LR(0)$语法分析器 \\[6pt]
    在\red{\bf 当前状态(编号)}下, 面对\blue{\bf 当前文法符号}时, 该采取什么\brown{\bf 动作}
  \end{center}
\end{frame}
%%%%%%%%%%%%%%%%%%%%

%%%%%%%%%%%%%%%%%%%%
\begin{frame}{}
  \begin{center}
    \fig{width = 0.60\textwidth}{figs/lr0-table-expr-add-mul}

    \vspace{0.30cm}
    % lr-actions.tex

% \usepackage{graphicx}
\begin{table}[]
  \centering
  \resizebox{0.50\textwidth}{!}{
    \renewcommand{\arraystretch}{1.2}
    \begin{tabular}{c||c}
      \hline
      $\brown{s}n$ & 移入输入符号, 并进入\teal{\bf 状态 $n$} \\ \hline
      $\brown{r}k$ & 使用\teal{\bf $k$ 号产生式}进行归约 \\ \hline
      $\brown{g}n$ & 转换到\teal{\bf 状态 $n$} \\ \hline
      $\brown{acc}$  & 成功接受, 结束 \\ \hline
      空白        & 错误  \\ \hline
    \end{tabular}}
\end{table}
  \end{center}
\end{frame}
%%%%%%%%%%%%%%%%%%%%

%%%%%%%%%%%%%%%%%%%%
\begin{frame}{}
  \begin{definition}[$LR(0)$文法]
    如果文法 $G$ 的\red{\bf $LR(0)$分析表}是\blue{\bf 无冲突}的,
    则 $G$ 是 $LR(0)$ 文法。
  \end{definition}

  \blue{\bf 无冲突:} \action{}表中每个单元格最多只有一种动作 \\[8pt]
\end{frame}
%%%%%%%%%%%%%%%%%%%%

%%%%%%%%%%%%%%%%%%%%
\begin{frame}{}
  \begin{center}
    再次板书演示``\red{\bf 栈}''上操作: \red{\bf 移入}与\red{\bf 规约}

    \begin{columns}
      \column{0.40\textwidth}
        % cfg-expr-add-mul-mul-first-numbering.tex

\begin{empheq}[box=\widefbox]{align*}
  E &\to E + T \\[8pt]
  E &\to T \\[8pt]
  T &\to T \ast F \\[8pt]
  T &\to F \\[8pt]
  F &\to (E) \\[8pt]
  F &\to \id
\end{empheq}
      \column{0.60\textwidth}
        \fig{width = 1.00\textwidth}{figs/lr0-table-expr-add-mul}
    \end{columns}

    \[
      w = \blue{(x, x, x)}
    \]

    \red{\bf 栈}中\red{明确}包含了语法分析器的状态
  \end{center}
\end{frame}
%%%%%%%%%%%%%%%%%%%%

%%%%%%%%%%%%%%%%%%%%
\begin{frame}{}
  \begin{center}
    % \red{\bf $LR(\ast)$语法分析器框架}

    % lr-framework.tex

\begin{algorithm}[H]
% \caption{}
% \label{alg:S}
\begin{algorithmic}[1]
  \Procedure{\blue{$LR$}}{\null}
    \State $\Call{Push}{S, \$}$ \qquad $\Call{Push}{S, s_{0}}$

    \hStatex
    \State $t \gets \Call{\purple{next-token}}{\null}$
    \While{$1$}
      \State $s \gets \Call{Top}{S}$  \Comment{\blue{$s$一定是某个状态编号, 而不是文法符号}}

      \hStatex
      \If{\red{$\action[s, t] = s_{i}$}} \Comment{\brown{移入}}
        \State $\Call{Push}{S, t}$ \qquad $\Call{Push}{S, i}$
        \State $t \gets \Call{\purple{next-token}}{\null}$
      \ElsIf{\red{$\action[s, t] = r_{j}$}} \Comment{\brown{规约; $j: A \to \alpha$}}
        \State $2 \times |\alpha|$ 次 $\Call{Pop}{S}$
        \State $s \gets \Call{Top}{S}$  \Comment{\blue{$s$一定是某个状态编号, 而不是文法符号}}
        \State $\Call{Push}{S, A}$ \qquad $\Call{Push}{S, \cyan{\goto[s, A]}}$ \Comment{\brown{转换状态}}
      \ElsIf{\red{$\action[s, t] = a$}} \Comment{\brown{接受}}
        \State {\bf break}
      \Else
        \State $\Call{\teal{error}}{\dots}$
      \EndIf
    \EndWhile
  \EndProcedure
\end{algorithmic}
\end{algorithm}
  \end{center}
\end{frame}
%%%%%%%%%%%%%%%%%%%%

%%%%%%%%%%%%%%%%%%%%
\begin{frame}{}
  \fig{width = 0.80\textwidth}{figs/lr0-id-star-id}

  \begin{center}
    $w = \id \ast \id \$$ 的分析过程
  \end{center}
\end{frame}
%%%%%%%%%%%%%%%%%%%%

%%%%%%%%%%%%%%%%%%%%
\begin{frame}{}
  \begin{center}
    \red{\bf 如何构造 $LR(0)$ 分析表?}

    \vspace{0.30cm}
    \fig{width = 0.60\textwidth}{figs/lr0-table-expr-add-mul}

    \vspace{0.30cm}
    \purple{\bf $LR(0)$ 分析表}指导$LR(0)$语法分析器 \\[6pt]
    在\red{\bf 当前状态(编号)}下, 面对\blue{\bf 当前文法符号}时, 该采取什么\brown{\bf 动作}
  \end{center}
\end{frame}
%%%%%%%%%%%%%%%%%%%%

%%%%%%%%%%%%%%%%%%%%
\begin{frame}{}
  \begin{center}
    $LR(0)$句柄寻找自动机
    \fig{width = 0.60\textwidth}{figs/lr0-automaton-expr}
  \end{center}
\end{frame}
%%%%%%%%%%%%%%%%%%%%

%%%%%%%%%%%%%%%%%%%%
\begin{frame}{}
  \begin{definition}[$LR(0)$项 (Item)]
    一个文法 $G$ 的一个 \red{\bf $LR(0)$} 项是 $G$ 的一个产生式再加上一个位于体部某处的点。
  \end{definition}

  \[
    A \to XYZ
  \]
  \begin{align*}
    A &\to \cdot XYZ \\[6pt]
    A &\to X \cdot YZ \\[6pt]
    A &\to XY \cdot Z \\[6pt]
    A &\to XYZ \cdot
  \end{align*}

  \begin{center}
    产生式 $A \epsilon$ 只有一个项 \blue{$A \to \cdot$}
  \end{center}
\end{frame}
%%%%%%%%%%%%%%%%%%%%

%%%%%%%%%%%%%%%%%%%%
\begin{frame}{}
  \begin{align*}
    A &\to \cdot XYZ \\[6pt]
    A &\to X \cdot YZ \\[6pt]
    A &\to XY \cdot Z \\[6pt]
    A &\to XYZ \cdot
  \end{align*}

  \vspace{0.30cm}
  \begin{center}
    项指明了语法分析器已经看到了一个产生式的哪些部分
  \end{center}
\end{frame}
%%%%%%%%%%%%%%%%%%%%

%%%%%%%%%%%%%%%%%%%%
\begin{frame}{}
  \begin{definition}[项集]
  \end{definition}

  \pause
  \begin{definition}[项集族]
  \end{definition}
\end{frame}
%%%%%%%%%%%%%%%%%%%%

%%%%%%%%%%%%%%%%%%%%
\begin{frame}{}
  \begin{center}
    \begin{definition}[增广文法 (Augmented Grammar)]
      文法 $G$ 的\red{\bf 增广文法}是在 $G$ 中加入产生式 \blue{$S' \to S$} 得到的文法。
    \end{definition}
  \end{center}
\end{frame}
%%%%%%%%%%%%%%%%%%%%

%%%%%%%%%%%%%%%%%%%%
\begin{frame}{}
  \begin{center}
    板书演示 $LR(0)$ 句柄寻找自动机
    \fig{width = 0.60\textwidth}{figs/lr0-automaton-expr}
  \end{center}
\end{frame}
%%%%%%%%%%%%%%%%%%%%

%%%%%%%%%%%%%%%%%%%%
\begin{frame}{}
  \begin{center}
    \[
      \textsc{closure}(I)
    \]

    \[
      \textsc{goto}(I, X)
    \]
  \end{center}
\end{frame}
%%%%%%%%%%%%%%%%%%%%

%%%%%%%%%%%%%%%%%%%%
\begin{frame}{}
\end{frame}
%%%%%%%%%%%%%%%%%%%%