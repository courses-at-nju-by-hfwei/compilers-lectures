% symtab.tex

%%%%%%%%%%%%%%%%%%%%
\begin{frame}{}
  \begin{columns}
    \column{0.50\textwidth}
      \begin{center}
        \fig{width = 0.95\textwidth}{figs/front-end}
      \end{center}
    \column{0.50\textwidth}
      \begin{center}
        \fig{width = 0.50\textwidth}{figs/back-end}
      \end{center}
  \end{columns}
\end{frame}
%%%%%%%%%%%%%%%%%%%%

%%%%%%%%%%%%%%%%%%%%
\begin{frame}{}
  \begin{center}
    \begin{definition}[符号表 (Symbol Table)]
      \red{\bf 符号表}是用于保存\blue{\bf 各种信息}的\purple{\bf 数据结构}。
    \end{definition}
  \end{center}
\end{frame}
%%%%%%%%%%%%%%%%%%%%

%%%%%%%%%%%%%%%%%%%%
\begin{frame}{}
  \begin{center}
    ``领域特定语言'' (DSL) 通常只有\blue{\bf 单作用域} (全局作用域)
  \end{center}
  \fig{width = 0.60\textwidth}{figs/config-file}
\end{frame}
%%%%%%%%%%%%%%%%%%%%

%%%%%%%%%%%%%%%%%%%%
\begin{frame}{}
  \begin{center}
    ``通用程序设计语言'' (GPL) 通常需要\blue{\bf 嵌套作用域}
  \end{center}
  \fig{width = 0.40\textwidth}{figs/vars}
\end{frame}
%%%%%%%%%%%%%%%%%%%%

%%%%%%%%%%%%%%%%%%%%
\begin{frame}{}
  \begin{columns}
    \column{0.50\textwidth}
      \fig{width = 0.65\textwidth}{figs/vars}
    \column{0.50\textwidth}
      \fig{width = 0.95\textwidth}{figs/vars-symtable}
  \end{columns}
\end{frame}
%%%%%%%%%%%%%%%%%%%%

%%%%%%%%%%%%%%%%%%%%
% \begin{frame}{}
%   \begin{center}
%     \red{\bf \texttt{struct}}: 类型作用域
%   \end{center}
%   \begin{columns}
%     \column{0.40\textwidth}
%       \fig{width = 0.90\textwidth}{figs/struct-code}
%     \column{0.60\textwidth}
%       \fig{width = 1.00\textwidth}{figs/struct-tree}
%   \end{columns}
%   \begin{center}
%     \blue{\bf $d.i$ \qquad $a.b.y$}
%   \end{center}
% \end{frame}
%%%%%%%%%%%%%%%%%%%%

%%%%%%%%%%%%%%%%%%%%
\begin{frame}{}
  \begin{center}
    声明: 添加符号与作用域 (``\blue{def}'')

    \vspace{1.00cm}
    使用: 解析符号 (``\blue{ref}'')

    \pause
    \vspace{1.50cm}
    \purple{先声明后使用} \emph{vs.} \purple{前向引用}
  \end{center}
\end{frame}
%%%%%%%%%%%%%%%%%%%%

%%%%%%%%%%%%%%%%%%%%
\begin{frame}{}
  \begin{center}
    \begin{definition}[类型表达式 (Type Expressions)]
      \begin{itemize}
        \setlength{\itemsep}{12pt}
        \item \blue{\bf 基本类型}是类型表达式;
          \begin{itemize}
            \item char, bool, int, float, double, void, $\dots$
          \end{itemize}
        \item \blue{\bf 类名}是类型表达式;
        \pause
        \item 如果 $t$ 是类型表达式, 则 \red{\bf array(num, t)} 是类型表达式;
        \item \red{\bf record($\langle \text{id : t}, \dots \rangle$)} 是类型表达式 (C 结构体);
        \pause
        \item 如果 $s$ 和 $t$ 是类型表达式, 则 \purple{$s \times t$} 是类型表达式;
        \item 如果 $s$ 和 $t$ 是类型表达式, 则 \purple{$s \to t$} 是类型表达式;
      \end{itemize}
    \end{definition}
  \end{center}
\end{frame}
%%%%%%%%%%%%%%%%%%%%

%%%%%%%%%%%%%%%%%%%%
\begin{frame}{}
  \begin{center}
    \red{\bf 类型声明}

    \fig{width = 0.80\textwidth}{figs/decl-code}

    \vspace{0.60cm}
    \blue{\bf \purple{\bf 符号表}中记录标识符的类型、宽度 (width)、偏移地址 (offset)}
  \end{center}
\end{frame}
%%%%%%%%%%%%%%%%%%%%

%%%%%%%%%%%%%%%%%%%%
\begin{frame}{}
  \begin{center}
    \fig{width = 0.80\textwidth}{figs/cfg-decl}

    \vspace{0.80cm}
    需要为每个 \red{\bf record} 生成单独的符号表
  \end{center}
\end{frame}
%%%%%%%%%%%%%%%%%%%%

%%%%%%%%%%%%%%%%%%%%
\begin{frame}{}
  \begin{center}
    \red{\bf 全局变量 $t$ 与 $w$} 将 $B$ 的\blue{\bf 类型与宽度}信息传递给产生式 \purple{\bf $C \to \epsilon$} \\[5pt]
    (在语法制导定义中, $t$ 与 $w$ 是 $C$ 的\green{\bf 继承属性})

    \vspace{0.30cm}
    \fig{width = 0.90\textwidth}{figs/SDT-type}

    \vspace{-0.80cm}
    \[
      \teal{\texttt{float x}}: \text{目前仅关心类型声明 (\teal{\texttt{float}})}
    \]
  \end{center}
\end{frame}
%%%%%%%%%%%%%%%%%%%%

%%%%%%%%%%%%%%%%%%%%
\begin{frame}{}
  \begin{center}
    \fig{width = 0.95\textwidth}{figs/anno-type-width}
    \vspace{-0.50cm}
    \[
      \teal{\texttt{int[2][3]}}
    \]
  \end{center}
\end{frame}
%%%%%%%%%%%%%%%%%%%%

%%%%%%%%%%%%%%%%%%%%
\begin{frame}{}
  \begin{center}
    \red{\bf 全局变量 offset 表示变量的相对地址}

    \vspace{0.20cm}
    \blue{\bf 全局变量 top 表示当前的符号表}

    \fig{width = 0.95\textwidth}{figs/SDT-offset}

    \vspace{-0.80cm}
    \[
      \teal{\texttt{float x; float y;}}
    \]
  \end{center}
\end{frame}
%%%%%%%%%%%%%%%%%%%%

%%%%%%%%%%%%%%%%%%%%
\begin{frame}{}
  \begin{center}
    \fig{width = 0.95\textwidth}{figs/SDT-record}

    \pause
    \vspace{0.60cm}
    \red{\bf record 类型表达式: \texttt{record(top)}} \\[6pt]
    字段名的偏移量是\red{相对}地址

    \pause
    \vspace{0.30cm}
    \blue{\bf 全局变量 top 表示当前的符号表}

    \pause
    \vspace{0.30cm}
    \green{\bf 全局变量 Env 表示符号表栈}

    \pause
    \vspace{-0.60cm}
    \[
      \teal{\texttt{record \{ float x; float y; \} p;}}
    \]
  \end{center}
\end{frame}
%%%%%%%%%%%%%%%%%%%%

%%%%%%%%%%%%%%%%%%%%
\begin{frame}{}
  \begin{center}
    \fig{width = 0.90\textwidth}{figs/decl-code}
  \end{center}
\end{frame}
%%%%%%%%%%%%%%%%%%%%

%%%%%%%%%%%%%%%%%%%%
\begin{frame}{}
  \begin{center}
    \href{https://github.com/antlr/symtab}{symtab @ antlr by parrt}

    \vspace{0.80cm}
    \href{https://github.com/parrt/cs652/tree/master/lectures/code/symtab/src}{symtab @ cs652 by parrt}
  \end{center}
\end{frame}
%%%%%%%%%%%%%%%%%%%%