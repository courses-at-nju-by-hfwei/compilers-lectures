% left-recursion-prec.tex

%%%%%%%%%%%%%%%%%%%%
\begin{frame}{}
  \begin{center}
    \Large{ANTLR 4 是如何处理\red{\bf 直接左递归与优先级}的?}
  \end{center}
\end{frame}
%%%%%%%%%%%%%%%%%%%%

%%%%%%%%%%%%%%%%%%%%
\begin{frame}{}
  \begin{center}
    \texttt{parser-allstar/LRExpr.g4}

    \vspace{0.30cm}
    \fig{width = 0.40\textwidth}{figs/LRExpr}
  \end{center}
\end{frame}
%%%%%%%%%%%%%%%%%%%%

%%%%%%%%%%%%%%%%%%%%
\begin{frame}{}
  \begin{center}
    \red{\bf 根本原因:} \\[10pt]
    究竟是在 \texttt{expr} 的\blue{\bf 当前调用}中匹配下一个运算符, \\[15pt]
    还是让 \texttt{expr} 的\blue{\bf 调用者}匹配下一个运算符。

    \pause
    \vspace{0.80cm}
    \texttt{antlr4 LRExpr -Xlog}
  \end{center}
\end{frame}
%%%%%%%%%%%%%%%%%%%%

%%%%%%%%%%%%%%%%%%%%
\begin{frame}{}
  \begin{center}
	\fig{width = 1.00\textwidth}{figs/LRExprXlog}

  \pause
	\fig{width = 0.30\textwidth}{figs/LRExpr}
  \end{center}
\end{frame}
%%%%%%%%%%%%%%%%%%%%

%%%%%%%%%%%%%%%%%%%%
\begin{frame}{}
  \fig{width = 0.40\textwidth}{figs/LRExprPrec}

  \begin{center}
	\texttt{expr[int \_p]}
  \end{center}

  \fig{width = 0.30\textwidth}{figs/LRExpr}
\end{frame}
%%%%%%%%%%%%%%%%%%%%

%%%%%%%%%%%%%%%%%%%%
\begin{frame}{}
  \fig{width = 0.60\textwidth}{figs/LRExprPrec}

  \[
	1 + 2 + 3 \qquad 1 + 2 \ast 3 \qquad 1 * 2 + 3
  \]
\end{frame}
%%%%%%%%%%%%%%%%%%%%

%%%%%%%%%%%%%%%%%%%%
\begin{frame}{}
  \begin{center}
    \texttt{parser-allstar/LRExprParen.g4}

    \vspace{0.30cm}
    \fig{width = 0.40\textwidth}{figs/LRExprParen}
  \end{center}
\end{frame}
%%%%%%%%%%%%%%%%%%%%

%%%%%%%%%%%%%%%%%%%%
\begin{frame}{}
  \begin{center}
    \texttt{parser-allstar/LRExprUS.g4}

    \vspace{0.30cm}
    \fig{width = 0.40\textwidth}{figs/LRExprUS}
  \end{center}
\end{frame}
%%%%%%%%%%%%%%%%%%%%

%%%%%%%%%%%%%%%%%%%%
\begin{frame}{}
  \begin{center}
    \fig{width = 0.60\textwidth}{figs/LRExprUSPrec}
  \end{center}

  \[
    \text{--}a!! \qquad \text{-}a+b!
  \]
\end{frame}
%%%%%%%%%%%%%%%%%%%%