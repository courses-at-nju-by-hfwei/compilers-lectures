% overview.tex

%%%%%%%%%%%%%%%%%%%%
% \begin{frame}{}
%   \begin{center}
%     \red{\bf Intermediate Representation (IR)}
%     \fig{width = 0.95\textwidth}{figs/ir-phase}
%   \end{center}
% \end{frame}
%%%%%%%%%%%%%%%%%%%%

%%%%%%%%%%%%%%%%%%%%
\begin{frame}{}
  \fig{width = 1.00\textwidth}{figs/ir}
  \begin{center}
    \href{https://dl.acm.org/doi/pdf/10.1145/2542661.2544374}{The Increasing Significance of Intermediate Representations in Compilers (Fred Chow; 2013)}
  \end{center}
\end{frame}
%%%%%%%%%%%%%%%%%%%%

%%%%%%%%%%%%%%%%%%%%
\begin{frame}
  \fig{width = 0.80\textwidth}{figs/daju}
\end{frame}
%%%%%%%%%%%%%%%%%%%%

%%%%%%%%%%%%%%%%%%%%
\begin{frame}{}
  \begin{center}
    {\Large 分工 \qquad 合作}

    \vspace{0.30cm}
    \fig{width = 0.45\textwidth}{figs/fengong}
  \end{center}

  \pause
  \begin{center}
    \red{父节点}为子节点准备跳转指令的目标标签 \\[5pt]
    子节点通过\blue{\bf 继承属性}确定跳转目标
  \end{center}
\end{frame}
%%%%%%%%%%%%%%%%%%%%

%%%%%%%%%%%%%%%%%%%%
\begin{frame}{}
  \begin{center}
    在自顶向下的分析过程中 \\[15pt]

    为右部的每个 $B$ 计算 $B.\text{true}$ 与 $B.\text{false}$ \\[10pt]
    为右部的每个 $S$ 计算 $S.\text{next}$
  \end{center}
\end{frame}
%%%%%%%%%%%%%%%%%%%%

%%%%%%%%%%%%%%%%%%%%
\begin{frame}{}
  \fig{width = 0.80\textwidth}{figs/grammars}
\end{frame}
%%%%%%%%%%%%%%%%%%%%