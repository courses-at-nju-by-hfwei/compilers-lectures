% boolexpr.tex

%%%%%%%%%%%%%%%%%%%%
\begin{frame}{}
  \begin{center}
    \red{\bf 布尔表达式的中间代码翻译}

    \fig{width = 0.75\textwidth}{figs/cfg-boolexpr}
  \end{center}
\end{frame}
%%%%%%%%%%%%%%%%%%%%

%%%%%%%%%%%%%%%%%%%%
\begin{frame}{}
  \begin{center}
    \fig{width = 0.85\textwidth}{figs/SDD-true-false}

    \pause
    \vspace{0.80cm}
    \teal{\texttt{if (true) {\bf assign}}}

    \vspace{0.20cm}
    \fig{width = 1.00\textwidth}{figs/SDD-if}

    \vspace{0.20cm}
    \teal{\texttt{if (false) {\bf assign}}}
  \end{center}
\end{frame}
%%%%%%%%%%%%%%%%%%%%

%%%%%%%%%%%%%%%%%%%%
\begin{frame}{}
  \begin{center}
    \fig{width = 0.80\textwidth}{figs/SDD-lnot}

    \pause
    \vspace{0.80cm}
    \teal{\texttt{if (!true) {\bf assign}}}

    \vspace{0.20cm}
    \fig{width = 1.00\textwidth}{figs/SDD-if}

    \vspace{0.20cm}
    \teal{\texttt{if (!false) {\bf assign}}}
  \end{center}
\end{frame}
%%%%%%%%%%%%%%%%%%%%

%%%%%%%%%%%%%%%%%%%%
\begin{frame}{}
  \begin{center}
    \red{\bf 短路求值}
    \fig{width = 0.95\textwidth}{figs/SDD-lor}

    \pause
    \vspace{0.80cm}
    \teal{\texttt{if (true || false) {\bf assign}}}

    \vspace{0.20cm}
    \fig{width = 0.95\textwidth}{figs/SDD-if}

    \vspace{0.20cm}
    \teal{\texttt{if (false || true) {\bf assign}}}
  \end{center}
\end{frame}
%%%%%%%%%%%%%%%%%%%%

%%%%%%%%%%%%%%%%%%%%
\begin{frame}{}
  \begin{center}
    \red{\bf 短路求值}
    \fig{width = 0.95\textwidth}{figs/SDD-land}

    \pause
    \vspace{0.80cm}
    \teal{\texttt{if (true \&\& false) {\bf assign}}}

    \vspace{0.20cm}
    \fig{width = 0.95\textwidth}{figs/SDD-if}

    \vspace{0.20cm}
    \teal{\texttt{if (false \&\& true) {\bf assign}}}
  \end{center}
\end{frame}
%%%%%%%%%%%%%%%%%%%%

%%%%%%%%%%%%%%%%%%%%
\begin{frame}{}
  \begin{center}
    \fig{width = 1.00\textwidth}{figs/SDD-rel}
  \end{center}
\end{frame}
%%%%%%%%%%%%%%%%%%%%

%%%%%%%%%%%%%%%%%%%%
\begin{frame}{}
  \begin{center}
    \teal{\texttt{\large if (x < 100 || x > 200 \&\& x != y) x = 0;}}

    \fig{width = 1.00\textwidth}{figs/if-boolexpr-ir}
  \end{center}
\end{frame}
%%%%%%%%%%%%%%%%%%%%

%%%%%%%%%%%%%%%%%%%%
\begin{frame}{}
  \begin{center}
    \teal{\texttt{\large if (x < 100 || x > 200 \&\& x != y) x = 0;}}

    \vspace{0.30cm}
    \fig{width = 0.50\textwidth}{figs/if-tac}
  \end{center}
\end{frame}
%%%%%%%%%%%%%%%%%%%%

%%%%%%%%%%%%%%%%%%%%
\begin{frame}{}
  \begin{center}
    \texttt{\teal{clang} -S -emit-llvm if-boolexpr.c -o if-boolexpr-opt0.ll}

    \vspace{0.30cm}
    \fig{width = 0.50\textwidth}{figs/if-boolexpr-c}
    \vspace{0.20cm}

    \texttt{\teal{opt} -dot-cfg if-boolexpr-opt0.ll} \\[5pt]
    \texttt{\teal{dot} -Tpdf .main.dot -o if-boolexpr-opt0-cfg.pdf}
  \end{center}
\end{frame}
%%%%%%%%%%%%%%%%%%%%

%%%%%%%%%%%%%%%%%%%%
\begin{frame}{}
  \fig{width = 0.50\textwidth}{figs/if-boolexpr-opt0-cfg}
\end{frame}
%%%%%%%%%%%%%%%%%%%%

%%%%%%%%%%%%%%%%%%%%
\begin{frame}{}
  \begin{center}
    {\Large 布尔表达式的作用: \red{\bf 布尔值} \emph{vs.} \blue{\bf 控制流跳转}}

    \vspace{0.50cm}
    \fig{width = 1.00\textwidth}{figs/boolexpr-value}

    \pause
    \vspace{0.50cm}
    根据 $E$ 所处的上下文判断 $E$ 所扮演的角色, 调用不同的代码生成函数 \\[15pt]

    函数 \blue{$jump(t, f)$}: 生成控制流代码 \\[6pt]
    函数 \red{$rvalue()$}: 生成计算布尔值的代码, 并将结果存储在临时变量中
  \end{center}
\end{frame}
%%%%%%%%%%%%%%%%%%%%

%%%%%%%%%%%%%%%%%%%%
\begin{frame}{}
  \begin{center}
    \teal{\texttt{\Large x = a < b \&\& c < d}}

    \fig{width = 0.70\textwidth}{figs/boolexpr-value-tac}
    \[
      \teal{E \to E_{1} \&\& E_{2}}
    \]

    为 $E$ 生成\blue{\bf 跳转代码},
    在\purple{\bf 真假出口处}将 {\bf true} 或 {\bf false} 存储到临时变量
  \end{center}
\end{frame}
%%%%%%%%%%%%%%%%%%%%

%%%%%%%%%%%%%%%%%%%%
% \begin{frame}{}
%   \begin{center}
%     {\Large \purple{\bf 布尔表达式}的作用: \red{\bf 布尔值} \emph{vs.} \blue{\bf 控制流跳转}}

%     \vspace{0.50cm}
%     \fig{width = 1.00\textwidth}{figs/boolexpr-value}

%     \pause
%     \vspace{0.80cm}
%     为什么要区分布尔表达式的这两种角色?

%     \pause
%     \vspace{0.80cm}
%     可以不区分布尔表达式的这两种角色吗?
%   \end{center}
% \end{frame}
%%%%%%%%%%%%%%%%%%%%