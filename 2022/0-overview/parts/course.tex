% course.tex

%%%%%%%%%%%%%%%%%%%%
\begin{frame}{}
  \begin{center}
    {\large $7$ 周 = $14$ 次课 \red{$<$} $8$ 周 = $16$ 次课}
  \end{center}

  \fig{width = 0.50\textwidth}{figs/12-24}
\end{frame}
%%%%%%%%%%%%%%%%%%%%

%%%%%%%%%%%%%%%%%%%%
\begin{frame}{}
  \begin{center}
    \red{\bf 作业 ($0$ 分):} $\approx 7$ 次作业, 每周 $1 \sim 2$ 题

    \pause
    \vspace{1.00cm}
    \red{\bf 实验 ($60$ 分):} $\approx 8$ 次必做实验 + $1$ 次选做实验 \blue{($\le 5$ 分)}

    \pause
    \vspace{1.00cm}
    \red{\bf 期末测试 ($40$ 分):} 考试周统一安排; $2$ 小时; \blue{开卷}

    \pause
    \vspace{1.00cm}
    \red{\bf 附加作业 ($\le 5$ 分):} 报告 + 录屏方式, 学习更现代的编译原理与技术
  \end{center}
\end{frame}
%%%%%%%%%%%%%%%%%%%%

%%%%%%%%%%%%%%%%%%%%
\begin{frame}
  \begin{center}
    \fig{width = 0.80\textwidth}{figs/no-plagiarism}
  \end{center}

  \vspace{0.30cm}
  \begin{columns}
    \column{0.20\textwidth}
    \column{0.60\textwidth}
      \begin{description}
        \item[附加项:] 不计分
        \item[期末测试:] 交与教务处
        \item[实验:] 当次实验计零分, 并扣 $5$ 分总评
      \end{description}
    \column{0.20\textwidth}
  \end{columns}
\end{frame}
%%%%%%%%%%%%%%%%%%%%

%%%%%%%%%%%%%%%%%%%%
\begin{frame}{}
  \begin{columns}
    \column{0.50\textwidth}
      \fig{width = 0.80\textwidth}{figs/square-logo}
    \column{0.50\textwidth}
      \fig{width = 0.60\textwidth}{figs/square-qrcode}
      \begin{center}
        邀请码: JCZ837HE
      \end{center}
  \end{columns}

  \vspace{1.0cm}
  \begin{center}
    每周三晚上布置作业 \qquad 下周三 \blue{$23:55$} 前\gray{提交}作业
  \end{center}
\end{frame}
%%%%%%%%%%%%%%%%%%%%

%%%%%%%%%%%%%%%%%%%%
% \begin{frame}{}
%   \[
%     45 = \red{0} + 5 + 15 + 15 + 10 + \red{5}
%   \]

%   \vspace{0.30cm}
%   \fig{width = 0.60\textwidth}{figs/labs}

%   \begin{center}
%     $L0:$ \red{\bf 环境配置}已经开放
%   \end{center}
% \end{frame}
%%%%%%%%%%%%%%%%%%%%

%%%%%%%%%%%%%%%%%%%%
\begin{frame}{}
  \begin{columns}
    \column{0.50\textwidth}
      \begin{center}
        QQ 群号: \blue{\bf 711805817}

        \fig{width = 0.60\textwidth}{figs/2021-qq}

        QQ 验证: \green{\bf 2021-编译原理}
      \end{center}
    \column{0.50\textwidth}
      \begin{center}
        {\bf \teal{助教:}} 夏宇、潘煜光、顾龙、$\dots$
      \end{center}
  \end{columns}
\end{frame}
%%%%%%%%%%%%%%%%%%%%

%%%%%%%%%%%%%%%%%%%%
\begin{frame}{}
  \begin{center}
    \fig{width = 0.50\textwidth}{figs/github-repos}

    \vspace{0.50cm}
    \teal{\url{https://github.com/courses-at-nju-by-hfwei/compilers-lectures/tree/master/2021}}
  \end{center}
\end{frame}
%%%%%%%%%%%%%%%%%%%%

%%%%%%%%%%%%%%%%%%%%
\begin{frame}{}
  \begin{columns}
    \column{0.50\textwidth}
      \fig{width = 0.60\textwidth}{figs/dragon-book}
      \begin{center}
        也可使用\blue{\bf ``本科教学版''}
      \end{center}
    \column{0.50\textwidth}
      \fig{width = 0.80\textwidth}{figs/lab-book}
      \begin{center}
        \teal{\url{https://cs.nju.edu.cn/changxu/2_compiler/index.html}}
      \end{center}
  \end{columns}
\end{frame}
%%%%%%%%%%%%%%%%%%%%

%%%%%%%%%%%%%%%%%%%%
\begin{frame}{}
  \begin{columns}
    \column{0.50\textwidth}
      \fig{width = 0.60\textwidth}{figs/flex}
      \begin{center}
        \href{https://en.wikipedia.org/wiki/Flex_(lexical_analyser_generator)}{Flex: 词法分析器生成器}
      \end{center}
    \column{0.50\textwidth}
      \fig{width = 0.60\textwidth}{figs/bison}
      \begin{center}
        \href{https://en.wikipedia.org/wiki/GNU_Bison}{Bison: 语法分析器生成器}
      \end{center}
  \end{columns}

  \pause
  \vspace{0.50cm}
  \begin{center}
    不够现代, 本课程不再支持
  \end{center}
\end{frame}
%%%%%%%%%%%%%%%%%%%%

%%%%%%%%%%%%%%%%%%%%
\begin{frame}{}
  \begin{columns}
    \column{0.50\textwidth}
      \fig{width = 0.80\textwidth}{figs/antlr-logo}
    \column{0.50\textwidth}
      \fig{width = 0.60\textwidth}{figs/parr.jpeg}
      \begin{center}
        Terence Parr
      \end{center}
  \end{columns}

  \vspace{0.80cm}
  \begin{center}
    \teal{\url{https://www.antlr.org/index.html}}
  \end{center}
\end{frame}
%%%%%%%%%%%%%%%%%%%%

%%%%%%%%%%%%%%%%%%%%
\begin{frame}{}
  \begin{columns}
    \column{0.50\textwidth}
      \fig{width = 0.70\textwidth}{figs/antlr4-book.jpg}
    \column{0.50\textwidth}
      % \fig{width = 0.70\textwidth}{figs/antlr4-book-ch.jpg}
  \end{columns}
\end{frame}
%%%%%%%%%%%%%%%%%%%%

%%%%%%%%%%%%%%%%%%%%
\begin{frame}{}
  \begin{columns}
    \column{0.50\textwidth}
      \fig{width = 0.70\textwidth}{figs/patterns-book}
    \column{0.50\textwidth}
      % \fig{width = 0.70\textwidth}{figs/patterns-book-ch}
  \end{columns}
\end{frame}
%%%%%%%%%%%%%%%%%%%%