% overview.tex

%%%%%%%%%%%%%%%%%%%%
\begin{frame}{}
  \begin{center}
    各个非终结符的\red{\bf 分工、合作}明确易懂

    \vspace{0.30cm}
    \fig{width = 0.40\textwidth}{figs/easy-mode}
    \vspace{0.30cm}

    只需要使用\blue{\bf 综合属性}
  \end{center}
\end{frame}
%%%%%%%%%%%%%%%%%%%%

%%%%%%%%%%%%%%%%%%%%
\begin{frame}{}
  \begin{center}
    {\Large 分工 \qquad 合作}

    \vspace{0.20cm}
    \fig{width = 0.60\textwidth}{figs/team-work-fly}
  \end{center}

  \begin{center}
    为布尔表达式 $B$ 计算逻辑值 (假设保存在临时变量 \texttt{t1} 中) \\[5pt]
    \texttt{\bf if}、\texttt{\bf while} 等语句
    根据 $B$ 的结果改变控制流
  \end{center}

  \vspace{-0.30cm}
  \[
    \texttt{\bf if}\; (B)\; S_{1}
  \]
\end{frame}
%%%%%%%%%%%%%%%%%%%%

%%%%%%%%%%%%%%%%%%%%
\begin{frame}{}
  \begin{center}
    \red{\bf 如何生成更短、\textcolor{lightgray}{更高效}的代码?}
  \end{center}

  \begin{columns}
    \column{0.50\textwidth}
      \fig{width = 0.60\textwidth}{figs/while-if-II}
    \column{0.50\textwidth}
      \fig{width = 0.85\textwidth}{figs/while-if-II-code}
  \end{columns}
\end{frame}
%%%%%%%%%%%%%%%%%%%%

%%%%%%%%%%%%%%%%%%%%
\begin{frame}{}
  \begin{center}
    \red{\bf 如何生成更短、\textcolor{lightgray}{更高效}的代码?}
  \end{center}

  \begin{columns}
    \column{0.50\textwidth}
      \fig{width = 0.80\textwidth}{figs/bool-short-circuit-II}
    \column{0.50\textwidth}
      \fig{width = 0.85\textwidth}{figs/bool-short-circuit-II-code}
  \end{columns}
\end{frame}
%%%%%%%%%%%%%%%%%%%%