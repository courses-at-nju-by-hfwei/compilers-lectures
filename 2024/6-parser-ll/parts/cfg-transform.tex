% cfg-transform.tex

%%%%%%%%%%%%%%%%%%%%
\begin{frame}{}
  \begin{center}
    不是 $LL(1)$ 文法怎么办?

    \pause
    \vspace{0.60cm}
    \red{\bf 改造它}

    \pause
    \vspace{0.80cm}
    \blue{消除左递归}

    \vspace{0.30cm}
    \blue{提取左公因子}
  \end{center}
\end{frame}
%%%%%%%%%%%%%%%%%%%%

%%%%%%%%%%%%%%%%%%%%
\begin{frame}{}
  \begin{center}
    \uncover<2->{
      $E$ 在\purple{\bf 不消耗任何词法单元}的情况下, 直接递归调用 $E$, 造成\green{\bf 死循环}
    }
    % cfg-expr-add-sub-mul-div.tex

\begin{empheq}[box=\widefbox]{align*}
  E \to E + E \mid E - E \mid E \ast E \mid E / E \mid (E) \mid \id \mid \num
\end{empheq}
    % cfg-expr-add-sub-mul-div-ETF.tex

\begin{empheq}[box=\widefbox]{align*}
  E &\to E + T \mid E - T \mid T \\[8pt]
  T &\to T \ast F \mid T / F \mid F \\[8pt]
  F &\to (E) \mid \id \mid \num
\end{empheq}

    \uncover<3->{
      \[
        \first(E + T) \cap \first(T) \neq \emptyset
      \]
      \green{\bf 不是 $LL(1)$ 文法}
    }
  \end{center}
\end{frame}
%%%%%%%%%%%%%%%%%%%%

%%%%%%%%%%%%%%%%%%%%
\begin{frame}{}
  \begin{center}
    改写成\red{\bf ``右递归''}文法
    \begin{columns}
      \column{0.26\textwidth}
        % cfg-expr-add-mul-no-left-recursion.tex

\begin{empheq}[box=\widefbox]{align*}
  E &\to T E' \\[8pt]
  E' &\to +\; T E' \mid \epsilon \\[8pt]
  T &\to F T' \\[8pt]
  T' &\to \ast\; F T' \mid \epsilon \\[8pt]
  F &\to (E) \mid \id
\end{empheq}
      \column{0.70\textwidth}
        \uncover<3->{\fig{width = 1.00\textwidth}{figs/ll1-table-expr}}
    \end{columns}

    \vspace{-0.30cm}
    \begin{columns}
      \column{0.50\textwidth}
        \pause
        \begin{align*}
          \first(F) &= \set{(, \id} \\
          \first(T) &= \set{(, \id} \\
          \red{\first(E)} &= \set{(, \id} \\
          \first(E') &= \set{+, \blue{\epsilon}} \\
          \first(T') &= \set{\ast, \epsilon}
        \end{align*}
      \column{0.50\textwidth}
        \begin{align*}
          \follow(E) &= \blue{\follow(E')} = \set{), \$} \\
          \follow(T) &= \follow(T') = \set{+, ), \$} \\
          \follow(F) &= \set{+, \ast, ), \$}
        \end{align*}
    \end{columns}
  \end{center}
\end{frame}
%%%%%%%%%%%%%%%%%%%%

%%%%%%%%%%%%%%%%%%%%
% \begin{frame}{}
%   \begin{center}
%     \blue{\bf 文件结束符 $\$$的用处}

%     \fig{width = 0.70\textwidth}{figs/ll1-stack-expr}
%   \end{center}
% \end{frame}
%%%%%%%%%%%%%%%%%%%%

%%%%%%%%%%%%%%%%%%%%
\begin{frame}{}
  \begin{center}
    \blue{\bf 直接左递归 (Direct Left Recursion)}
  \end{center}

  \begin{columns}
    \column{0.50\textwidth}
      \[
        A \to A \alpha \mid \beta
      \]
    \column{0.50\textwidth}
      \begin{align*}
        A &\to \beta A' \\[5pt]
        A' &\to \alpha A' \mid \epsilon
      \end{align*}
  \end{columns}
\end{frame}
%%%%%%%%%%%%%%%%%%%%

%%%%%%%%%%%%%%%%%%%%
\begin{frame}{}
  \begin{center}
    % cfg-A-left-recursion.tex

\begin{empheq}[box=\widefbox]{align*}
  A \to A \alpha_{1} \mid A \alpha_{2} \mid \dots A \alpha_{m}
    \mid \beta_{1} \mid \beta_{2} \mid \dots \beta_{n}
\end{empheq}

    其中, $\beta_{i}$ 都不以 $A$ 开头

    \vspace{0.30cm}
    % cfg-A-right-recursion.tex

\begin{empheq}[box=\widefbox]{align*}
  A &\to \beta_{1}A' \mid \beta_{2}A' \mid \dots \mid \beta_{n}A' \\[8pt]
  A' &\to \alpha_{1}A' \mid \alpha_{2}A' \mid \dots \mid \alpha_{m}A' \mid \epsilon
\end{empheq}
  \end{center}
\end{frame}
%%%%%%%%%%%%%%%%%%%%

%%%%%%%%%%%%%%%%%%%%
\begin{frame}{}
  \begin{center}
    \blue{\bf 间接左递归 (Indirect Left Recursion)}

    \vspace{-0.30cm}
    % cfg-indirect-left-recursion-in-book.tex

\begin{empheq}[box=\widefbox]{align*}
	S &\to Ac \mid c \\[8pt]
	A &\to Bb \mid b \\[8pt]
	B &\to Sa \mid a
\end{empheq}

    \vspace{-0.30cm}
    \[
      S \implies Ac \implies Bbc \implies \red{S}abc
    \]

    \pause
	\vspace{-0.30cm}
    \[
      \red{\boxed{A_{i} \to A_{j} \alpha \implies i < j}}
    \]

    \pause
    \fig{width = 0.90\textwidth}{figs/left-recursion-alg}
  \end{center}
\end{frame}
%%%%%%%%%%%%%%%%%%%%

%%%%%%%%%%%%%%%%%%%%
\begin{frame}{}
  \begin{columns}
	\column{0.50\textwidth}
	  % cfg-indirect-left-recursion-in-book.tex

\begin{empheq}[box=\widefbox]{align*}
	S &\to Ac \mid c \\[8pt]
	A &\to Bb \mid b \\[8pt]
	B &\to Sa \mid a
\end{empheq}
	\column{0.50\textwidth}
	  \pause
		\begin{align*}
			B &\to \red{S}a \mid a \\[5pt]
			  &\to (Ac \mid c) a \mid a \\[5pt]
			&\to \red{A}ca \mid ca \mid a \\[5pt]
			&\to (Bb \mid b) ca \mid ca \mid a \\[5pt]
			&\to \red{B}bca \mid bca \mid ca \mid a \\[5pt]
		\end{align*}
  \end{columns}

  \pause
  \vspace{-0.50cm}
  % cfg-indirect-left-recursion-in-book-transformed.tex

\begin{empheq}[box=\widefbox]{align*}
	S &\to Ac \mid c \\[8pt]
	A &\to Bb \mid b \\[8pt]
	B &\to (bca \mid ca \mid a) B' \\[8pt]
	B' &\to bcaB' \mid \epsilon
\end{empheq}

  \pause
  \[
    \red{\boxed{A_{i} \to A_{j} \alpha \implies i < j}}
  \]
\end{frame}
%%%%%%%%%%%%%%%%%%%%

%%%%%%%%%%%%%%%%%%%%
\begin{frame}{}
  \begin{center}
    \red{\bf 算法要求:}

    \vspace{0.50cm}
    文法中不存在环 (形如 $A \dstar A$ 的推导)

    \vspace{0.50cm}
	文法中不存在 $\epsilon$ 产生式 (形如 $A \to \epsilon$ 的产生式)
  \end{center}

  \pause
  % cfg-S-A.tex

\begin{empheq}[box=\widefbox]{align*}
  S &\to A a \mid b \\[8pt]
  A &\to A c \mid S b \mid \epsilon
\end{empheq}
\end{frame}
%%%%%%%%%%%%%%%%%%%%

%%%%%%%%%%%%%%%%%%%%
\begin{frame}{}
  \begin{center}
    \blue{\bf 提取左公因子 (Left-Factoring)}

    \vspace{0.30cm}
    \fig{width = 0.55\textwidth}{figs/optional-init-left-refactoring}

    \cyan{ANTLR 4 可以处理有左公因子的文法}
  \end{center}
\end{frame}
%%%%%%%%%%%%%%%%%%%%

%%%%%%%%%%%%%%%%%%%%
\begin{frame}{}
  \begin{center}
    \begin{columns}
      \column{0.50\textwidth}
        \fig{width = 0.90\textwidth}{figs/ifstat-g4}
      \column{0.50\textwidth}
        \fig{width = 0.90\textwidth}{figs/ifstat-factor-g4}
    \end{columns}

    \vspace{1.00cm}
    很明显, 提取左公因子无助于消除文法二义性
  \end{center}
\end{frame}
%%%%%%%%%%%%%%%%%%%%

%%%%%%%%%%%%%%%%%%%%
\begin{frame}{}
  \pause
  \fig{width = 1.00\textwidth}{figs/chatgpt-Q}

  \pause
  \fig{width = 1.00\textwidth}{figs/chatgpt-A}
\end{frame}
%%%%%%%%%%%%%%%%%%%%