% cfg.tex

%%%%%%%%%%%%%%%%%%%%
\begin{frame}{}
  \fig{width = 0.50\textwidth}{figs/cfg}

  \begin{center}
    上下文无关文法
  \end{center}
\end{frame}
%%%%%%%%%%%%%%%%%%%%

%%%%%%%%%%%%%%%%%%%%
\begin{frame}{}
  \begin{definition}[Context-Free Grammar (CFG); 上下文无关文法]
    上下文无关文法 $G$ 是一个四元组 $G = (T, N, S, P)$:
    \vspace{0.30cm}

    \begin{itemize}
      \setlength{\itemsep}{8pt}
      \item $T$ 是\red{\bf 终结符号} (\teal{Terminal}) 集合, 对应于词法分析器产生的词法单元
      \item $N$ 是\red{\bf 非终结符号} (\teal{Non-terminal}) 集合
      \item $S$ 是\red{\bf 开始} (\teal{Start}) 符号 ($S \in N$且唯一)
      \item $P$ 是\red{\bf 产生式} (\teal{Production}) 集合
        \[
          \boxed{A \in N \longrightarrow \alpha \in (T \cup N)^{\ast}}
        \]
        \vspace{-0.60cm}
        \begin{description}
          \setlength{\itemsep}{5pt}
          \item[头部/左部 (Head) $A$:] \purple{\bf 单个}非终结符
          \item[体部/右部 (Body) $\alpha$:] 终结符与非终结符构成的串, 也可以是空串 $\epsilon$
        \end{description}
    \end{itemize}
  \end{definition}
\end{frame}
%%%%%%%%%%%%%%%%%%%%

%%%%%%%%%%%%%%%%%%%%
\begin{frame}{}
  \begin{center}
    Context-Sensitive Grammar (CSG)
  \end{center}
  % csg-anbncn.tex

\begin{empheq}[box=\widefbox]{align*}
  S &\to aBC \\[4pt]
  S &\to aSBC \\[4pt]
  CB &\to CZ \\[4pt]
  CZ &\to WZ \\[4pt]
  WZ &\to WC \\[4pt]
  WC &\to BC \\[4pt]
  aB &\to ab \\[4pt]
  bB &\to bb \\[4pt]
  bC &\to bc \\[4pt]
  cC &\to cc
\end{empheq}
\end{frame}
%%%%%%%%%%%%%%%%%%%%

%%%%%%%%%%%%%%%%%%%%
\begin{frame}{}
  \begin{center}
    \red{[Extended] Backus-Naur form ([E]BNF)}
    \[
      \blue{+ \quad \ast \quad ?}
    \]
  \end{center}

  \begin{columns}
    \column{0.33\textwidth}
      \fig{width = 0.70\textwidth}{figs/Backus}
      \begin{center}
        \href{https://en.wikipedia.org/wiki/John_Backus}{John Backus ($1924 \sim 2007$)}
      \end{center}
    \column{0.34\textwidth}
      \fig{width = 0.75\textwidth}{figs/Naur}
      \begin{center}
        \href{https://en.wikipedia.org/wiki/Peter_Naur}{Peter Naur ($1928 \sim 2016$)}
      \end{center}
    \column{0.34\textwidth}
      \fig{width = 0.85\textwidth}{figs/Wirth}
      \begin{center}
        \href{https://en.wikipedia.org/wiki/Niklaus_Wirth}{Niklaus Wirth ($1934 \sim$)}
      \end{center}
  \end{columns}

  \pause
  \begin{center}
    1977 (FORTRAN) \qquad 2005 (ALGOL60) \qquad 1984 (PLs; PASCAL)
  \end{center}
\end{frame}
%%%%%%%%%%%%%%%%%%%%

%%%%%%%%%%%%%%%%%%%%
\begin{frame}{}
  \fig{width = 0.60\textwidth}{figs/syntax-semantics}
  \begin{center}
    语义: 上下文无关文法 $G$ 定义了一个\red{\bf 语言} $L(G)$

    \pause
    \vspace{0.50cm}
    语言是\blue{\bf 串}的集合

    \vspace{0.30cm}
    串从何来?
  \end{center}
\end{frame}
%%%%%%%%%%%%%%%%%%%%

%%%%%%%%%%%%%%%%%%%%
\begin{frame}{}
  \begin{center}
    {\large \red{\bf 推导}} (Derivation)

    % cfg-expr-add-mul.tex

\begin{empheq}[box=\widefbox]{align*}
  E \to E + E \mid E \ast E \mid (E) \mid -E \mid \id
\end{empheq}

    \vspace{0.50cm}
    推导即是将某个产生式的左边\blue{\bf 替换}成它的右边

    \vspace{1.00cm}
    每一步推导需要选择\purple{\bf 替换哪个非终结符号}, 以及\purple{\bf 使用哪个产生式}
  \end{center}
\end{frame}
%%%%%%%%%%%%%%%%%%%%

%%%%%%%%%%%%%%%%%%%%
\begin{frame}{}
  \begin{center}
    {\large \red{\bf 推导}} (Derivation)
  \end{center}

  % cfg-expr-add-mul.tex

\begin{empheq}[box=\widefbox]{align*}
  E \to E + E \mid E \ast E \mid (E) \mid -E \mid \id
\end{empheq}

  \vspace{-0.50cm}
  \[
    E \implies -E \implies -(E) \implies -(E + E)
      \implies \purple{-(\id + E)} \implies -(\id + \id)
  \]

  \pause
  \vspace{-0.30cm}
  \begin{align*}
    &E \implies -E: \text{经过一步推导得出} \\
    &E \dplus -(\id + E): \text{经过一步或多步推导得出} \\
    &E \dstar -(\id + E): \text{经过零步或多步推导得出}
  \end{align*}

  \pause
  \vspace{-0.50cm}
  \[
    E \implies -E \implies -(E) \implies -(E + E) \implies \blue{-(E + \id)} \implies -(\id + \id)
  \]

  \pause
  \begin{center}
    \purple{Leftmost (最左) Derivation} \quad \blue{Rightmost (最右) Derivation}
  \end{center}
\end{frame}
%%%%%%%%%%%%%%%%%%%%

%%%%%%%%%%%%%%%%%%%%
\begin{frame}{}
  \begin{definition}[Sentential Form; 句型]
    如果 $S \dstar \alpha$, 且 $\alpha \in (T \cup N)^{\ast}$,
    则称 $\alpha$ 是文法 $G$ 的一个\red{\bf 句型}。
  \end{definition}

  \vspace{0.30cm}
  % cfg-expr-add-mul.tex

\begin{empheq}[box=\widefbox]{align*}
  E \to E + E \mid E \ast E \mid (E) \mid -E \mid \id
\end{empheq}

  \vspace{-0.80cm}
  \[
    E \implies -E \implies -(E) \implies -(E + E)
      \implies \red{-(\id + E)} \implies \blue{-(\id + \id)}
  \]

  \pause
  \begin{definition}[Sentence; 句子]
    如果 $S \dstar w$, 且 $w \in T^{\ast}$,
    则称 $w$ 是文法 $G$ 的一个\blue{\bf 句子}。
  \end{definition}
\end{frame}
%%%%%%%%%%%%%%%%%%%%

%%%%%%%%%%%%%%%%%%%%
\begin{frame}{}
  \begin{definition}[文法$G$生成的语言 $L(G)$]
    文法 $G$ 的\red{\bf 语言} $L(G)$ 是它能推导出的\blue{\bf 所有句子}构成的集合。

    \[
      w \in L(G) \iff S \dstar w
    \]
  \end{definition}
\end{frame}
%%%%%%%%%%%%%%%%%%%%

%%%%%%%%%%%%%%%%%%%%
\begin{frame}{}
  \begin{center}
    关于文法 $G$ 的\red{\bf 两个基本问题:}
  \end{center}

  \vspace{0.60cm}
  \begin{columns}
    \column{0.10\textwidth}
    \column{0.80\textwidth}
      \begin{itemize}
        \setlength{\itemsep}{15pt}
        \item \blue{\bf Membership 问题:}
          给定字符串 $x \in \red{T^{\ast}}$, $x \in L(G)$?
        \item \blue{$L(G)$} 究竟是什么?
      \end{itemize}
    \column{0.10\textwidth}
  \end{columns}
\end{frame}
%%%%%%%%%%%%%%%%%%%%

%%%%%%%%%%%%%%%%%%%%
\begin{frame}{}
  \begin{center}
    \blue{给定字符串 $x \in T^{\ast}$, $x \in L(G)$?}

    \vspace{0.20cm}
    (即, 检查$x$是否符合文法$G$)

    \pause
    \vspace{1.00cm}
    这就是\red{\bf 语法分析器}的任务:

    \vspace{0.30cm}
    为输入的词法单元流寻找推导、\purple{\bf 构建语法分析树}, 或者报错
  \end{center}
\end{frame}
%%%%%%%%%%%%%%%%%%%%

%%%%%%%%%%%%%%%%%%%%
\begin{frame}{}
  \begin{center}
    \blue{$L(G)$ 是什么?}

    \vspace{0.60cm}
    这是\red{\bf 程序设计语言设计者}需要考虑的问题
  \end{center}
\end{frame}
%%%%%%%%%%%%%%%%%%%%

%%%%%%%%%%%%%%%%%%%%
\begin{frame}{}
  \begin{columns}
    \column{0.50\textwidth}
      % cfg-bp.tex

\begin{empheq}[box=\widefbox]{align*}
  S &\to SS \\[8pt]
  S &\to (S) \\[8pt]
  S &\to () \\[8pt]
  S &\to \epsilon
\end{empheq}

      \[
        L(G) = \pause \set{\text{良匹配括号串}}
      \]
    \column{0.50\textwidth}
      \pause
      % cfg-anbn.tex

\begin{empheq}[box=\widefbox]{align*}
  S &\to aSb \\[8pt]
  S &\to \epsilon
\end{empheq}

      \[
        L(G) = \pause \set{a^{n}b^{n} \mid n \ge 0}
      \]
  \end{columns}
\end{frame}
%%%%%%%%%%%%%%%%%%%%

%%%%%%%%%%%%%%%%%%%%
\begin{frame}{}
  \begin{center}
    字母表 $\Sigma = \set{a, b}$ 上的所有\blue{\bf 回文串} (Palindrome) 构成的语言

    \pause
    \vspace{0.30cm}
    % cfg-palindrome.tex

\begin{empheq}[box=\widefbox]{align*}
  S &\to aSa \\[8pt]
  S &\to bSb \\[8pt]
  S &\to a  \\[8pt]
  S &\to b \\[8pt]
  S &\to \epsilon
\end{empheq}
  \end{center}
\end{frame}
%%%%%%%%%%%%%%%%%%%%

%%%%%%%%%%%%%%%%%%%%
\begin{frame}{}
  \[
    \set{b^{n}a^{m}b^{2n} \mid n \ge 0, m \ge 0}
  \]

  \pause
  \vspace{0.50cm}
  % cfg-bn-am-b2n.tex

\begin{empheq}[box=\widefbox]{align*}
  S &\to bSbb \mid A \\[8pt]
  A &\to aA \mid \epsilon
\end{empheq}
\end{frame}
%%%%%%%%%%%%%%%%%%%%

%%%%%%%%%%%%%%%%%%%%
\begin{frame}{}
  \[
    \set{x \in \set{a, b}^{\ast} \mid x \text{ 中 }\; a, b \text{ 个数\red{相同}}}
  \]

  \pause
  \vspace{0.50cm}
  % cfg-equal-number-a-b.tex

\begin{empheq}[box=\widefbox]{align*}
  V \to aVbV \mid bVaV \mid \epsilon
\end{empheq}
\end{frame}
%%%%%%%%%%%%%%%%%%%%

%%%%%%%%%%%%%%%%%%%%
\begin{frame}{}
  \[
    \set{x \in \set{a, b}^{\ast} \mid x \text{ 中 }\; a, b \text{ 个数\red{不同}}}
  \]

  \pause
  \begin{columns}
    \column{0.25\textwidth}
    \column{0.50\textwidth}
      % cfg-unequal-number-a-b.tex

\begin{empheq}[box=\widefbox]{align*}
  S &\to T \mid U \\[8pt]
  T &\to VaT \mid VaV \\[8pt]
  U &\to VbU \mid VbV \\[8pt]
  V &\to aVbV \mid bVaV \mid \epsilon
\end{empheq}
    \column{0.25\textwidth}
  \end{columns}
\end{frame}
%%%%%%%%%%%%%%%%%%%%

%%%%%%%%%%%%%%%%%%%%
\begin{frame}{}
  % csg-anbncn.tex

\begin{empheq}[box=\widefbox]{align*}
  S &\to aBC \\[4pt]
  S &\to aSBC \\[4pt]
  CB &\to CZ \\[4pt]
  CZ &\to WZ \\[4pt]
  WZ &\to WC \\[4pt]
  WC &\to BC \\[4pt]
  aB &\to ab \\[4pt]
  bB &\to bb \\[4pt]
  bC &\to bc \\[4pt]
  cC &\to cc
\end{empheq}
  \pause
  \[
    L = \set{a^nb^nc^n \mid n \ge 1}
  \]
\end{frame}
%%%%%%%%%%%%%%%%%%%%