% cfg-re.tex

%%%%%%%%%%%%%%%%%%%%
\begin{frame}{}
  \begin{center}
    为什么不使用优雅、强大的\blue{\bf 正则表达式}描述程序设计语言的语法?

    \vspace{0.50cm}
    \fig{width = 0.40\textwidth}{figs/grammar-hierarchy}
    正则表达式的表达能力\red{\bf 严格弱于}上下文无关文法
  \end{center}
\end{frame}
%%%%%%%%%%%%%%%%%%%%

%%%%%%%%%%%%%%%%%%%%
\begin{frame}{}
  \begin{center}
    每个\blue{正则表达式} $r$ 对应的语言 $L(r)$ 都可以使用\blue{上下文无关文法}来描述

    \[
      r = (a | b)^{\ast} abb
    \]

    \pause
    \fig{width = 0.50\textwidth}{figs/nfa-abb}

    \pause
    \fig{width = 0.40\textwidth}{figs/nfa-abb-cfg}
    此外, 若 $\delta(A_i, \epsilon) = A_{j}$, 则添加 $A_{i} \to A_{j}$
  \end{center}
\end{frame}
%%%%%%%%%%%%%%%%%%%%

%%%%%%%%%%%%%%%%%%%%
\begin{frame}{}
  \begin{center}
    % cfg-anbn.tex

\begin{empheq}[box=\widefbox]{align*}
  S &\to aSb \\[8pt]
  S &\to \epsilon
\end{empheq}

    \[
      L = \set{a^{n} b^{n} \mid n \ge 0}
    \]
    该语言\red{\bf 无法}使用正则表达式来描述
  \end{center}
\end{frame}
%%%%%%%%%%%%%%%%%%%%

%%%%%%%%%%%%%%%%%%%%
\begin{frame}{}
  \begin{theorem}
    $L = \set{a^{n} b^{n} \mid n \ge 0}$ 无法使用正则表达式描述。
  \end{theorem}

  \pause
  \begin{center}
    \red{\bf 反证法}

    \pause
    \vspace{0.30cm}
    假设存在正则表达式 $r$: $L(r) = L = \set{a^{n} b^{n} \mid n \ge 0}$

    \pause
    \vspace{0.30cm}
    则存在\blue{\bf 有限}状态自动机 $D(r)$: $L(D(r)) = L$; 设其\purple{\bf 状态数}为 $k \ge 1$

    \pause
    \vspace{0.30cm}
    \fbox{\red{考虑输入 $a^{m}\; (m \ge k)$}}
    \fig{width = 0.90\textwidth}{figs/pumping-lemma-anbn}

    \pause
    \vspace{0.30cm}
    $D(r)$ 也能接受 $a^{i+j} b^{i}$; \red{\bf 矛盾!}
  \end{center}
\end{frame}
%%%%%%%%%%%%%%%%%%%%

%%%%%%%%%%%%%%%%%%%%
\begin{frame}{}
  \begin{center}
    \href{https://en.wikipedia.org/wiki/Pumping\_lemma\_for\_regular\_languages}{
      \teal{Pumping Lemma for \red{Regular Languages} (@ wiki)}}
  \end{center}

  % \fig{width = 0.80\textwidth}{figs/Pumping-Lemma-xyz}
  \fig{width = 0.50\textwidth}{figs/wiki-pumping-lemma}
  \vspace{-0.30cm}

  \pause
  \begin{theorem}
    \red{If} $L$ is a regular language,
    \red{then} there exists a number \red{$p \ge 1$}
    (pumping length) such that any string $s$ in $L$ of length $\ge p$
    can be divided into three pieces, \green{$s = xyz$}, satisfying the following conditions:
    \begin{enumerate}[(i)]
      \item \purple{$|y| \ge 1$}
      \item \violet{$|xy| \le p$}
      \item \blue{$\forall i \ge 0: xy^{i}z \in L$}
    \end{enumerate}
  \end{theorem}
\end{frame}
%%%%%%%%%%%%%%%%%%%%

%%%%%%%%%%%%%%%%%%%%
\begin{frame}{}
  \begin{example}
    $D = \set{1^{n^{2}} \mid n \ge 0}$ is not regular.
  \end{example}

  \pause
  \vspace{0.50cm}
  \begin{center}
    考虑 $s = 1^{p^{2}}$ \qquad ($p$ is the pumping length)
  \end{center}

  \pause
  \[
    |s| \ge p
  \]

  \pause
  \[
    s = xyz
  \]

  \pause
  \[
    \blue{xy^{2}z \in D}
  \]

  \pause
  \[
    p^2 < |xy^{2}z| = |xyz| + |y| \le p^2 + p < p^2 + 2p + 1 = (p + 1)^2
  \]

  \pause
  \[
    \red{xy^{2}z \notin D}
  \]
\end{frame}
%%%%%%%%%%%%%%%%%%%%

%%%%%%%%%%%%%%%%%%%%
\begin{frame}{}
  \begin{center}
    $L = \set{a^{n}b^{n} \mid n \ge 0}$ 无法使用正则表达式描述

    \vspace{0.50cm}
    \fig{width = 0.40\textwidth}{figs/grammar-hierarchy}
    \vspace{0.50cm}

    \red{$Q:$ 上下文无关文法描述能力的界限在那里?}
  \end{center}
\end{frame}
%%%%%%%%%%%%%%%%%%%%

%%%%%%%%%%%%%%%%%%%%
\begin{frame}{}
  % csg-anbncn.tex

\begin{empheq}[box=\widefbox]{align*}
  S &\to aBC \\[4pt]
  S &\to aSBC \\[4pt]
  CB &\to CZ \\[4pt]
  CZ &\to WZ \\[4pt]
  WZ &\to WC \\[4pt]
  WC &\to BC \\[4pt]
  aB &\to ab \\[4pt]
  bB &\to bb \\[4pt]
  bC &\to bc \\[4pt]
  cC &\to cc
\end{empheq}
  \[
    L = \set{a^nb^nc^n \mid n \ge 1}
  \]
\end{frame}
%%%%%%%%%%%%%%%%%%%%

%%%%%%%%%%%%%%%%%%%%
\begin{frame}{}
  \begin{center}
    \[
      L = \set{a^{n}b^{n}c^{n} \mid n \ge 0}
    \]
    \href{https://en.wikipedia.org/wiki/Pumping\_lemma\_for\_context-free\_languages}{
      \teal{Pumping Lemma for \red{Context-free Languages} (@ wiki)}}

    \vspace{0.30cm}
    \fig{width = 0.55\textwidth}{figs/Pumping-Lemma-uvwxy}
  \end{center}
\end{frame}
%%%%%%%%%%%%%%%%%%%%