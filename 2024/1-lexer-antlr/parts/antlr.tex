% antlr.tex

%%%%%%%%%%%%%%%%%%%%
\begin{frame}{}
  \fig{width = 0.40\textwidth}{figs/antlr-logo}
  \begin{center}
    词法分析器生成器
  \end{center}
\end{frame}
%%%%%%%%%%%%%%%%%%%%

%%%%%%%%%%%%%%%%%%%%
\begin{frame}{}
  \begin{center}
    \red{\bf 输入:} 词法单元的规约

    \vspace{0.50cm}
    \texttt{\blue{SimpleExpr.g4}}

    \vspace{1.00cm}
    \red{\bf 输出:} 词法分析器

    \begin{columns}
      \column{0.30\textwidth}
      \column{0.40\textwidth}
        \begin{itemize}
          \setlength{\itemsep}{8pt}
          \item \texttt{\teal{SimpleExprLexer.java}}
        \end{itemize}
      \column{0.30\textwidth}
    \end{columns}

    % \pause
    % \vspace{1.00cm}
    % \texttt{\purple{javac} SimpleExpr*.java}

    % \vspace{0.20cm}
    % \texttt{\purple{grun} simpleexpr.SimpleExpr prog -tokens}
  \end{center}
\end{frame}
%%%%%%%%%%%%%%%%%%%%

%%%%%%%%%%%%%%%%%%%%
\begin{frame}{}
  \fig{width = 0.60\textwidth}{figs/talk-cheap}
\end{frame}
%%%%%%%%%%%%%%%%%%%%

%%%%%%%%%%%%%%%%%%%%
\begin{frame}{}
  \begin{center}
    命令行式使用 ANTLR v4
    \fig{width = 0.75\textwidth}{figs/antlr-cmd}
    \url{https://www.antlr.org/}
  \end{center}
\end{frame}
%%%%%%%%%%%%%%%%%%%%

%%%%%%%%%%%%%%%%%%%%
\begin{frame}{}
  \begin{center}
    交互式使用 ANTLR v4
    \fig{width = 0.75\textwidth}{figs/antlr-plugin-intellij}
    \url{https://www.antlr.org/tools.html}
  \end{center}
\end{frame}
%%%%%%%%%%%%%%%%%%%%

%%%%%%%%%%%%%%%%%%%%
\begin{frame}{}
  \begin{center}
    编程式使用 ANTLR v4
    \fig{width = 0.80\textwidth}{figs/antlr-plugin-gradle}
    \url{https://docs.gradle.org/current/userguide/antlr_plugin.html}
  \end{center}
\end{frame}
%%%%%%%%%%%%%%%%%%%%

%%%%%%%%%%%%%%%%%%%%
\begin{frame}{}
  \begin{center}
    ANTLR v4 中的\red{\bf 冲突解决}规则

  \vspace{1.00cm}
	\begin{columns}[]
	  \column{0.15\textwidth}
	  \column{0.70\textwidth}
	  \begin{description}[最前优先匹配:]
		\setlength{\itemsep}{15pt}
      \item[最前优先匹配:] 关键字 \emph{vs.} 标识符 \\[3pt]
        \texttt{ML\_COMMENT} \emph{vs.} \texttt{DOC\_COMMENT}
      \item[最长优先匹配:] \texttt{1.23},\quad \texttt{>=},\quad \texttt{ifhappy}
      \item[非贪婪匹配:] \texttt{()??},\quad \texttt{()*?},\quad \texttt{()+?}
        % ,\quad \texttt{\{n, \}?}
	  \end{description}
	  \column{0.15\textwidth}
	\end{columns}
  \end{center}
\end{frame}
%%%%%%%%%%%%%%%%%%%%

%%%%%%%%%%%%%%%%%%%%
\begin{frame}{}
  \begin{columns}
    \column{0.40\textwidth}
      \fig{width = 0.80\textwidth}{figs/antlr4-book-ch}
    \column{0.60\textwidth}
      \begin{description}
        \setlength{\itemsep}{8pt}
        \item[5.5:] 识别常见的语法结构
        \item[15.5:] 词法规则
        \item[15.6:] 通配符与非贪婪子规则
        \item[\red{12:}] 掌握词法分析的``黑魔法''
      \end{description}
  \end{columns}
\end{frame}
%%%%%%%%%%%%%%%%%%%%

%%%%%%%%%%%%%%%%%%%%
\begin{frame}{}
  \begin{center}
    以\red{\bf 编程的方式}使用 ANTLR 4 生成的 \texttt{xxxLexer.java}
  \end{center}

  \begin{columns}
    \column{0.40\textwidth}
      \pause
      \fig{width = 0.60\textwidth}{figs/antlr-header-package}
    \column{0.60\textwidth}
      \pause
      \fig{width = 1.00\textwidth}{figs/antlr-lexer-java}
  \end{columns}
\end{frame}
%%%%%%%%%%%%%%%%%%%%

%%%%%%%%%%%%%%%%%%%%
\begin{frame}{}
  \begin{center}
    \red{lexer grammar} \\[10pt]

    Section 4.1 \blue{(1. 语法导入)} of《ANTLR 4 权威指南》
  \end{center}

  \begin{columns}
    \column{0.50\textwidth}
      \fig{width = 1.00\textwidth}{figs/simpleexpr-lexer-grammar}
    \column{0.50\textwidth}
      \fig{width = 0.70\textwidth}{figs/simpleexpr-lexer-grammar-import}
  \end{columns}
\end{frame}
%%%%%%%%%%%%%%%%%%%%

%%%%%%%%%%%%%%%%%%%%
\begin{frame}{}
  \begin{center}
    You can learn a lot from \href{https://github.com/antlr/grammars-v4/tree/master/c}{grammars-v4/c}.
  \end{center}
\end{frame}
%%%%%%%%%%%%%%%%%%%%

%%%%%%%%%%%%%%%%%%%%
\begin{frame}{}
  \begin{columns}
    \column{0.40\textwidth}
      \fig{width = 0.80\textwidth}{figs/antlr4-book-ch}
    \column{0.60\textwidth}
      \begin{description}
        \setlength{\itemsep}{8pt}
        \item[5.5:] 识别常见的语法结构
        \item[15.5:] 词法规则
        \item[15.6:] 通配符与非贪婪子规则
        \item[\red{12:}] 掌握词法分析的``黑魔法''
      \end{description}
  \end{columns}
\end{frame}
%%%%%%%%%%%%%%%%%%%%

%%%%%%%%%%%%%%%%%%%%
\begin{frame}{}
  \fig{width = 0.25\textwidth}{figs/ChatGPT}
  % \begin{center}
  %   {\large Let Us Ask \red{ChatGPT} to Write a Lexer Using ANTLR 4}
  % \end{center}
\end{frame}
%%%%%%%%%%%%%%%%%%%%