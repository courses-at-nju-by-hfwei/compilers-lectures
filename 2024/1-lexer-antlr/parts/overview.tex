% overview.tex

%%%%%%%%%%%%%%%%%%%%
\begin{frame}{}
  \begin{center}
    \red{\bf 输入:} 程序文本/字符串 $s$ (\teal{\texttt{CharStream}}) \uncover<2>{+ \blue{\bf 词法单元 (token) 的规约}}

    \vspace{0.50cm}
    \fig{width = 0.65\textwidth}{figs/lexer}

    \vspace{0.30cm}
    \red{\bf 输出:} 词法单元流 (\teal{\texttt{TokenStream}})
  \end{center}
\end{frame}
%%%%%%%%%%%%%%%%%%%%

%%%%%%%%%%%%%%%%%%%%
% \begin{frame}{}
%   \fig{width = 0.90\textwidth}{figs/antlr-lexer-parser}
%   \begin{center}
%     \texttt{SimpleExpr.g4}
%   \end{center}
% \end{frame}
%%%%%%%%%%%%%%%%%%%%

%%%%%%%%%%%%%%%%%%%%
% \begin{frame}{}
%   \[
%     \boxed{\text{token}: \langle \red{\text{token-class}}, \gray{\text{attribute-value}} \rangle}
%   \]

%   \fig{width = 0.80\textwidth}{figs/token-table}

%   \pause
%   \begin{center}
%     \intkw/\ifkw \qquad 关键词 \\[8pt]
%     \ws \qquad 空格、制表符、换行符  \\[8pt]
%     \comment \qquad ``//'' 开头的一行注释或者``/* */'' 包围的多行注释
%   \end{center}
%   % \input{tables/token}
% \end{frame}
%%%%%%%%%%%%%%%%%%%%

%%%%%%%%%%%%%%%%%%%%
\begin{frame}{}
  \begin{center}
    {\large 词法分析器的三种设计方法 (\green{由易到难})}
  \end{center}

  \vspace{0.30cm}
  \begin{columns}
    \column{0.33\textwidth}
      \pause
      \fig{width = 0.90\textwidth}{figs/antlr-logo}
      \begin{center}
        词法分析器生成器
      \end{center}
    \column{0.34\textwidth}
      \pause
      \fig{width = 0.85\textwidth}{figs/by-hand}
      \begin{center}
        手写词法分析器
      \end{center}
    \column{0.33\textwidth}
      \pause
      \fig{width = 0.85\textwidth}{figs/dfa-lexer}
      \begin{center}
        自动化词法分析器
      \end{center}
  \end{columns}

  \pause
  \vspace{0.80cm}
  \begin{center}
    很多生产环境下的编译器(如 gcc)仍选择{\bf 手写词法分析器}
  \end{center}
\end{frame}
%%%%%%%%%%%%%%%%%%%%

%%%%%%%%%%%%%%%%%%%%
\begin{frame}{}
  \fig{width = 0.20\textwidth}{figs/gcc-logo}

  \begin{columns}
    \column{0.50\textwidth}
      \fig{width = 0.90\textwidth}{figs/gcc-c-lex}
    \column{0.50\textwidth}
      \fig{width = 0.90\textwidth}{figs/gcc-cpp-lex}
  \end{columns}
\end{frame}
%%%%%%%%%%%%%%%%%%%%

%%%%%%%%%%%%%%%%%%%%
\begin{frame}{}
  \begin{columns}
    \column{0.50\textwidth}
      \fig{width = 0.80\textwidth}{figs/mysql}
    \column{0.50\textwidth}
      \fig{width = 0.95\textwidth}{figs/sql-lex}
  \end{columns}
\end{frame}
%%%%%%%%%%%%%%%%%%%%