% overview.tex

%%%%%%%%%%%%%%%%%%%%
\begin{frame}{}
  \fig{width = 0.50\textwidth}{figs/get-started}
\end{frame}
%%%%%%%%%%%%%%%%%%%%

%%%%%%%%%%%%%%%%%%%%
\begin{frame}{}
  \begin{center}
    \begin{columns}
      \column{0.50\textwidth}
        \fig{width = 0.85\textwidth}{figs/compiler-blackbox}
      \column{0.50\textwidth}
        \fig{width = 0.60\textwidth}{figs/keep-calm-open-box}
    \end{columns}
  \end{center}
\end{frame}
%%%%%%%%%%%%%%%%%%%%

%%%%%%%%%%%%%%%%%%%%
% \begin{frame}[fragile]{}
%   \begin{columns}
%     \column{0.60\textwidth}
%       \fig{width = 0.80\textwidth}{figs/lang-processing-system}
%     \column{0.40\textwidth}
%       \fig{width = 0.80\textwidth}{figs/gcc-v}
%   \end{columns}
% \end{frame}
%%%%%%%%%%%%%%%%%%%%

%%%%%%%%%%%%%%%%%%%%
\begin{frame}{}
  \begin{center}
    IR: Intermediate Representation (中间表示)
    \vspace{0.50cm}

    \fig{width = 0.80\textwidth}{figs/front-back}

    \vspace{0.50cm}
    前端 \red{\bf (分析阶段)}: 分析源语言程序, 收集所有必要的信息 \\[8pt]
    后端 \red{\bf (综合阶段)}: 利用收集到的信息, 生成目标语言程序
  \end{center}
\end{frame}
%%%%%%%%%%%%%%%%%%%%

%%%%%%%%%%%%%%%%%%%%
\begin{frame}{}
  \fig{width = 0.50\textwidth}{figs/talk-cheap}
\end{frame}
%%%%%%%%%%%%%%%%%%%%

%%%%%%%%%%%%%%%%%%%%
\begin{frame}{}
  \begin{center}
    \fig{width = 0.90\textwidth}{figs/clang-homepage}

    \vspace{0.30cm}
    \url{https://clang.llvm.org/}\\[5pt]
    (\texttt{factorial.c})
  \end{center}
\end{frame}
%%%%%%%%%%%%%%%%%%%%

%%%%%%%%%%%%%%%%%%%%
\begin{frame}{}
  \begin{center}
    \fig{width = 0.80\textwidth}{figs/optimizer}

    \vspace{0.50cm}
    机器无关的中间表示优化
  \end{center}
\end{frame}
%%%%%%%%%%%%%%%%%%%%

%%%%%%%%%%%%%%%%%%%%
\begin{frame}{}
  \begin{center}
    \fig{width = 0.90\textwidth}{figs/clang-homepage}

    \vspace{0.30cm}
    \url{https://clang.llvm.org/}\\[5pt]
    (\texttt{opt.c})
  \end{center}
\end{frame}
%%%%%%%%%%%%%%%%%%%%

%%%%%%%%%%%%%%%%%%%%
\begin{frame}{}
  \begin{center}
    在设计实际生产环境中的编译器时, {\bf 优化}通常占用了大多数时间
  \end{center}

  \begin{columns}
    \column{0.50\textwidth}
      \fig{width = 1.00\textwidth}{figs/FZ-Compiler}
    \column{0.50\textwidth}
      \fig{width = 0.80\textwidth}{figs/FZ-Compiler-book}
  \end{columns}

  \vspace{0.50cm}
  \begin{center}
    Maple IR 及其各种优化技术
  \end{center}
\end{frame}
%%%%%%%%%%%%%%%%%%%%

%%%%%%%%%%%%%%%%%%%%
\begin{frame}{}
  \begin{columns}
    \column{0.50\textwidth}
      \begin{center}
        编译器前端: 分析阶段
        \fig{width = 0.95\textwidth}{figs/front-end}
      \end{center}
    \column{0.50\textwidth}
      \begin{center}
        编译器后端: 综合阶段
        \fig{width = 0.50\textwidth}{figs/back-end}
      \end{center}
  \end{columns}
\end{frame}
%%%%%%%%%%%%%%%%%%%%

%%%%%%%%%%%%%%%%%%%%
\begin{frame}{}
  \begin{center}
    \fig{width = 0.90\textwidth}{figs/clang-homepage}

    \vspace{0.30cm}
    \url{https://clang.llvm.org/}\\[5pt]
    (\texttt{naming.c})
  \end{center}
\end{frame}
%%%%%%%%%%%%%%%%%%%%

% %%%%%%%%%%%%%%%%%%%%
% \begin{frame}{}
%   \fig{width = 0.50\textwidth}{figs/assignment}

%   \begin{center}
%     作为一名\red{\bf 程序员}, 你看到了什么?

%     \vspace{0.50cm}
%     \begin{columns}
%       \column{0.10\textwidth}
%       \column{0.80\textwidth}
%         \begin{description}
%           \setlength{\itemsep}{12pt}
%           \pause
%           \item[词法:] 标识符、数字、运算符
%           \pause
%           \item[语法:] 包含算术运算的赋值语句
%           \pause
%           \item[语义:] \texttt{position, initial, rate} 是数值类型
%           \pause
%           \item[物理定律:] $\text{当前位置 } = \text{ 初始位置 } + \text{ 速度 } \times \text{ 时间}$
%         \end{description}
%       \column{0.10\textwidth}
%     \end{columns}

%     \pause
%     \vspace{1.00cm}
%     但是, 作为\blue{\bf 编译器}, 它仅仅看到了一个\blue{\bf 字符串}
%   \end{center}
% \end{frame}
% %%%%%%%%%%%%%%%%%%%%

% %%%%%%%%%%%%%%%%%%%%
% \begin{frame}{}
%   \begin{center}
%     \red{\bf 词法分析器 (Lexer/Scanner):} 将{\bf 字符}流转化为{\bf 词法单元} (token) 流。

%     \[
%       \boxed{\text{token}: \langle \text{token-class}, \text{attribute-value} \rangle}
%     \]

%     \fig{width = 0.50\textwidth}{figs/assignment}

%     \begin{align*}
%       \langle \id, \red{1} \rangle \quad
%       \langle \ws \rangle \quad
%       \langle \assign \rangle \quad
%       \langle \ws \rangle \quad
%       \langle \id, \red{2} \rangle \quad
%       \langle \ws \rangle \quad \\
%       \langle + \rangle \quad
%       \langle \ws \rangle \quad
%       \langle \id, \red{3} \rangle \quad
%       \langle \ws \rangle \quad
%       \langle \ast \rangle \quad
%       \langle \ws \rangle \quad
%       \langle \num, \red{4} \rangle
%     \end{align*}
%     (此处, $1, 2, 3, 4$ 是指向\red{\bf 符号表}的指针)
%   \end{center}
% \end{frame}
% %%%%%%%%%%%%%%%%%%%%

% %%%%%%%%%%%%%%%%%%%%
% \begin{frame}{}
%   \begin{center}
%     \red{\bf 语法分析器 (Parser):} 构建{\bf 词法单元}之间的语法结构, 生成\blue{\bf 语法树}

%     \vspace{0.80cm}
%     \fig{width = 0.50\textwidth}{figs/assignment}
%     \fig{width = 0.50\textwidth}{figs/syntax-tree}
%   \end{center}
% \end{frame}
% %%%%%%%%%%%%%%%%%%%%

% %%%%%%%%%%%%%%%%%%%%
% \begin{frame}{}
%   \begin{center}
%     \red{\bf 语义分析器:} 语义检查, 如{\bf 类型检查}、{\bf ``先声明后使用''约束检查}

%     \vspace{0.80cm}
%     % \fig{width = 0.50\textwidth}{figs/assignment}
%     \fig{width = 0.50\textwidth}{figs/semantic-analysis}

%     \vspace{0.30cm}
%     通过语法树上的遍历来完成
%   \end{center}
% \end{frame}
% %%%%%%%%%%%%%%%%%%%%

% %%%%%%%%%%%%%%%%%%%%
% \begin{frame}{}
%   \begin{center}
%     \red{\bf 中间代码生成器:} 生成中间代码, 如 {\bf ``三地址代码''}

%     \vspace{0.80cm}
%     % \fig{width = 0.50\textwidth}{figs/assignment}
%     \fig{width = 0.50\textwidth}{figs/ICG}

%     \vspace{0.30cm}
%     中间代码类似目标代码, 但不含有机器相关信息 (如寄存器、指令格式)
%   \end{center}
% \end{frame}
% %%%%%%%%%%%%%%%%%%%%%%%%%%%%%%%%%%%%%%%

% %%%%%%%%%%%%%%%%%%%%
% \begin{frame}{}
%   \begin{center}
%     \red{\bf 中间代码优化器}

%     \vspace{0.80cm}
%     % \fig{width = 0.50\textwidth}{figs/assignment}
%     \fig{width = 0.50\textwidth}{figs/CO}

%     \vspace{0.30cm}
%     编译时计算、消除冗余临时变量
%   \end{center}
% \end{frame}
% %%%%%%%%%%%%%%%%%%%%%%%%%%%%%%%%%%%%%%%

% %%%%%%%%%%%%%%%%%%%%
% \begin{frame}{}
%   \begin{center}
%     \red{\bf 代码生成器:} 生成目标代码, 主要任务包括{\bf 指令选择、寄存器分配}

%     \vspace{0.80cm}
%     % \fig{width = 0.50\textwidth}{figs/assignment}
%     \fig{width = 0.50\textwidth}{figs/CG}
%   \end{center}
% \end{frame}
% %%%%%%%%%%%%%%%%%%%%%%%%%%%%%%%%%%%%%%%

% %%%%%%%%%%%%%%%%%%%%%%%%%%%%%%%%%%%%%%%
% \begin{frame}{}
%   \begin{center}
%     \red{\bf 符号表:} 收集并管理{\bf 变量名/函数名}相关的信息
%   \end{center}

%   \begin{columns}
%     \column{0.50\textwidth}
%       \begin{center}
%         \blue{\bf 变量名:} \\[3pt]
%         类型、寄存器、内存地址、行号

%         \vspace{0.50cm}
%         \blue{\bf 函数名:} \\[3pt]
%         参数个数、参数类型、返回值类型
%       \end{center}
%     \column{0.50\textwidth}
%       \fig{width = 0.80\textwidth}{figs/symbol-table}
%   \end{columns}
% \end{frame}
% %%%%%%%%%%%%%%%%%%%%%%%%%%%%%%%%%%%%%%%

% %%%%%%%%%%%%%%%%%%%%%%%%%%%%%%%%%%%%%%%
% \begin{frame}{}
%   \fig{width = 0.85\textwidth}{figs/st-api}

%   \vspace{0.50cm}
%   \begin{center}
%     红黑树 (RB-Tree)、哈希表 (Hashtable)
%   \end{center}
% \end{frame}
% %%%%%%%%%%%%%%%%%%%%%%%%%%%%%%%%%%%%%%%

% %%%%%%%%%%%%%%%%%%%%%%%%%%%%%%%%%%%%%%%
% \begin{frame}{}
%   \begin{center}
%     为了方便表达\red{\bf 嵌套结构与作用域}, 可能需要维护多个符号表
%   \end{center}

%   \fig{width = 0.50\textwidth}{figs/nested-symbol-tables}
% \end{frame}
% %%%%%%%%%%%%%%%%%%%%%%%%%%%%%%%%%%%%%%%