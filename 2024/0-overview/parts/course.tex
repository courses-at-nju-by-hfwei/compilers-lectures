% course.tex

%%%%%%%%%%%%%%%%%%%%
\begin{frame}{}
  \begin{center}
    \fig{width = 0.50\textwidth, angle = 90}{figs/duoduo}

    \vspace{0.50cm}
    {哦? 编译原理?}
  \end{center}
\end{frame}
%%%%%%%%%%%%%%%%%%%%

%%%%%%%%%%%%%%%%%%%%
\begin{frame}{}
  \begin{center}
    \fig{width = 0.45\textwidth}{figs/writing-a-c-compiler}
  \end{center}
\end{frame}
%%%%%%%%%%%%%%%%%%%%

%%%%%%%%%%%%%%%%%%%%
\begin{frame}{}
  \begin{center}
    \fig{width = 0.50\textwidth}{figs/push-the-button}

    \vspace{0.50cm}
    \blue{\text{gcc factorial.c -o factorial}}
  \end{center}
\end{frame}
%%%%%%%%%%%%%%%%%%%%

%%%%%%%%%%%%%%%%%%%%
\begin{frame}{}
  \begin{center}
    \[
      \text{``高级''语言 $\implies$ \gray{(通常)} ``低级''语言 (如, 汇编语言)}
    \]
    汇编语言经过{\bf 汇编器}生成机器语言

    \vspace{0.30cm}
    \fig{width = 0.40\textwidth}{figs/compiler-blackbox}

    \pause
    \begin{columns}
      \column{0.50\textwidth}
        \fig{width = 0.60\textwidth}{figs/gopherjs-logo}
      \column{0.50\textwidth}
        \fig{width = 0.95\textwidth}{figs/gopherjs}
    \end{columns}
  \end{center}
\end{frame}
%%%%%%%%%%%%%%%%%%%%

%%%%%%%%%%%%%%%%%%%%
\begin{frame}{}
  \begin{center}
    \fig{width = 0.40\textwidth}{figs/lang-processing-system}
  \end{center}
\end{frame}
%%%%%%%%%%%%%%%%%%%%

%%%%%%%%%%%%%%%%%%%%
\begin{frame}{}
  \begin{center}
    \red{$Q:$ 机器语言是如何跑起来的?}

    \pause
    \vspace{0.80cm}
    \red{\bf 作业 (P1 $\sim$ P9):} \teal{\url{https://www.bilibili.com/video/BV1EW411u7th}} \\[8pt]
    (计算机科学速成课 $40$ 集全 Crash Course Computer Science)

    \vspace{0.80cm}
    \begin{columns}
      \column{0.30\textwidth}
        \pause
        \fig{width = 0.80\textwidth}{figs/insight}
      \column{0.35\textwidth}
        \pause
        \fig{width = 0.85\textwidth}{figs/2000-years}
      \column{0.30\textwidth}
        \pause
        \fig{width = 0.80\textwidth}{figs/insight}
    \end{columns}
  \end{center}
\end{frame}
%%%%%%%%%%%%%%%%%%%%

%%%%%%%%%%%%%%%%%%%%
\begin{frame}{}
  \begin{center}
    ``我只想安静地做个码农, 你为什么要来强迫我?''
  \end{center}

  \begin{columns}
    \column{0.50\textwidth}
      \fig{width = 0.70\textwidth}{figs/code-maker}
    \column{0.50\textwidth}
      \fig{width = 0.80\textwidth}{figs/fool-me.png}
  \end{columns}
\end{frame}
%%%%%%%%%%%%%%%%%%%%

%%%%%%%%%%%%%%%%%%%%
\begin{frame}{}
  \fig{width = 0.50\textwidth}{figs/fun}
\end{frame}
%%%%%%%%%%%%%%%%%%%%

%%%%%%%%%%%%%%%%%%%%
\begin{frame}{}
  \begin{center}
    \href{https://alda.io/}{alda.io}

    \fig{width = 0.90\textwidth}{figs/alda}

    \texttt{alda repl}

    \vspace{0.20cm}
    \texttt{alda play -f .alda}
  \end{center}
\end{frame}
%%%%%%%%%%%%%%%%%%%%

%%%%%%%%%%%%%%%%%%%%
\begin{frame}{}
  \begin{center}
    \href{https://www.zhihu.com/zvideo/1467997225217908736}{alda @ 知乎}

    \fig{width = 0.50\textwidth}{figs/alda-zhihu}

    \href{https://youtu.be/c5pCFtwO4j8}{10 Demo: Alda @ youtube}
  \end{center}
\end{frame}
%%%%%%%%%%%%%%%%%%%%

%%%%%%%%%%%%%%%%%%%%
\begin{frame}{}
  \begin{center}
    \fig{width = 0.50\textwidth}{figs/Dragon-Trees}

    \vspace{0.30cm}
    \href{https://www.cs.unm.edu/~joel/PaperFoldingFractal/paper.html}{Fractal Grower (Try It!)}

    \vspace{0.30cm}
    \href{https://en.wikipedia.org/wiki/Sierpi\%C5\%84ski\_triangle}{Sierpinski Triangle @ wiki}
  \end{center}
\end{frame}
%%%%%%%%%%%%%%%%%%%%

%%%%%%%%%%%%%%%%%%%%
\begin{frame}{}
  \fig{width = 0.60\textwidth}{figs/useful}
\end{frame}
%%%%%%%%%%%%%%%%%%%%

%%%%%%%%%%%%%%%%%%%%
\begin{frame}{}
  \begin{center}
    {\Large 语言类应用程序}
  \end{center}

  \begin{columns}
    \column{0.15\textwidth}
    \column{0.70\textwidth}
      \begin{itemize}
        \setlength{\itemsep}{6pt}
        \item 配置文件解析 (\href{https://en.wikipedia.org/wiki/.properties}{.properties})
        \item CSV 文件 (\href{https://en.wikipedia.org/wiki/Comma-separated_values\#/media/File:CsvDelimited001.svg}{Comma-Separated Values})
        \item JSON 文件 (\href{https://en.wikipedia.org/wiki/JSON\#Syntax}{JavaScript Object Notation})
        \vspace{8pt}
        \pause
        \item SQL 引擎 (\href{https://en.wikipedia.org/wiki/SQL_syntax}{Structured Query Language})
        \item TLA$^{+}$/TLAPS (\href{https://github.com/Starydark/PaxosStore-tla/blob/master/theorem\%20proving/TPaxos.tla}{TPaxos.tla})
        \item (Java) 字节码解释器
        \item C/C++ 语言编译器
        \vspace{8pt}
        \pause
        \item 排版工具 (\href{https://github.com/courses-at-nju-by-hfwei/compilers-lectures/blob/master/2021/0-overview/parts/overview.tex}{\LaTeX})
        \item 绘图工具 (\href{https://www.overleaf.com/learn/latex/TikZ_package}{TikZ},
          \href{https://renenyffenegger.ch/notes/tools/Graphviz/examples/index}{Dot/Graphviz})
        \item L-System (\href{https://en.wikipedia.org/wiki/L-system\#Example_3:_Cantor_set}{Cantor Set})
      \end{itemize}
    \column{0.15\textwidth}
  \end{columns}
\end{frame}
%%%%%%%%%%%%%%%%%%%%

%%%%%%%%%%%%%%%%%%%%
\begin{frame}{}
  \fig{width = 0.60\textwidth}{figs/hard-to-learn}
\end{frame}
%%%%%%%%%%%%%%%%%%%%

%%%%%%%%%%%%%%%%%%%%
\begin{frame}{}
  \begin{center}
    \red{\bf 学习编译原理最大的难点: ``只见树木, 不见森林''}
    \fig{width = 0.50\textwidth}{figs/trees}
  \end{center}
\end{frame}
%%%%%%%%%%%%%%%%%%%%

%%%%%%%%%%%%%%%%%%%%
\begin{frame}{}
  \begin{columns}
    \column{0.30\textwidth}
      \fig{width = 0.60\textwidth}{figs/gongxing}
    \column{0.50\textwidth}
      \fig{width = 0.80\textwidth}{figs/diy}
  \end{columns}
\end{frame}
%%%%%%%%%%%%%%%%%%%%

%%%%%%%%%%%%%%%%%%%%
\begin{frame}{}
  \begin{columns}
    \column{0.10\textwidth}
    \column{0.80\textwidth}
      \begin{description}[<+->]
        \setlength{\itemsep}{25pt}
        \item[\blue{\bf 考勤 ($0\%$):}] 非必要不点名, 不需要请假
        \item[\cyan{\bf 平时作业 ($0\%$):}] $\approx 10$ 次作业, 每次 $\le 3$ 题
        \item[\red{\bf 课程实验 ($60\%$):}] $8 \sim 10$ 次实验
        \onslide<4>{\item[\red{\bf 期末测试 ($40\%$):}] 考试周统一安排; \purple{$3$ 小时; 上机考试}}
        \item[\red{\bf 期末测试 ($40\%$):}] 考试周统一安排; \purple{$2$ 小时; 开卷}
      \end{description}
    \column{0.10\textwidth}
  \end{columns}
\end{frame}
%%%%%%%%%%%%%%%%%%%%

%%%%%%%%%%%%%%%%%%%%
\begin{frame}{}
  \begin{center}
    每周五发布作业 \qquad 下周五 \blue{$23:55$} 前\teal{自愿提交}作业
  \end{center}

  \begin{columns}
    \column{0.50\textwidth}
      \fig{width = 0.80\textwidth}{figs/square-logo}
    \column{0.50\textwidth}
      \fig{width = 0.60\textwidth}{figs/2024-Compilers-square-qrcode}
    \vspace{-0.80cm}
    \begin{center}
      \purple{邀请码: 8G928EBJ}
    \end{center}
  \end{columns}
\end{frame}
%%%%%%%%%%%%%%%%%%%%

%%%%%%%%%%%%%%%%%%%%
\begin{frame}{}
  \begin{center}
    课程实验:开发 \href{https://compiler.educg.net/}{\textsf{SysY}} 语言编译器

    \fig{width = 0.80\textwidth}{figs/huawei-bisheng}

    \vspace{0.30cm}
    \red{\bf L0:} \red{\bf 环境配置} 本周五 18:00 发布
  \end{center}
\end{frame}
%%%%%%%%%%%%%%%%%%%%

%%%%%%%%%%%%%%%%%%%%
\begin{frame}
  \begin{center}
    鼓励讨论,但需独立编码完成课程实验
    \fig{width = 0.80\textwidth}{figs/no-plagiarism}
  \end{center}

  \begin{columns}
    \column{0.20\textwidth}
    \column{0.60\textwidth}
    \begin{description}
      \item[课程实验:] 抄袭者当次实验计 $0$ 分
    \end{description}
    \column{0.20\textwidth}
  \end{columns}
\end{frame}
%%%%%%%%%%%%%%%%%%%%

%%%%%%%%%%%%%%%%%%%%
\begin{frame}{}
  \begin{columns}
    \column{0.50\textwidth}
    \begin{center}
      QQ 群号: \blue{\bf 869910463}

      \fig{width = 0.60\textwidth}{figs/2024-Compilers-QQ-Class1}
    \end{center}
    \column{0.50\textwidth}
    \begin{center}
      {\bf \teal{助教:}} 顾龙、李和煦、钱品亦
    \end{center}
  \end{columns}
\end{frame}
%%%%%%%%%%%%%%%%%%%%

%%%%%%%%%%%%%%%%%%%%
\begin{frame}{}
  \begin{center}
    \url{http://docs.compilers.cpl.icu/} \\[5pt]

    \fig{width = 0.25\textwidth}{figs/compilers-docs-2024}

    编译原理课程网站,请\red{\bf 收藏}并及时关注网站更新
  \end{center}
\end{frame}
%%%%%%%%%%%%%%%%%%%%

%%%%%%%%%%%%%%%%%%%%
\begin{frame}{}
  \begin{center}
    \fig{width = 1.00\textwidth}{figs/compilers-lectures-2024}

    \vspace{0.50cm}
    \url{https://github.com/courses-at-nju-by-hfwei/compilers-lectures/tree/master/2024}

    \vspace{0.50cm}
    \texttt{overview.pdf \qquad overview-handout.pdf}
  \end{center}
\end{frame}
%%%%%%%%%%%%%%%%%%%%

%%%%%%%%%%%%%%%%%%%%
% \begin{frame}{}
%   \begin{center}
%     \fig{width = 0.60\textwidth}{figs/compilers-papers-we-love}
%   \end{center}
% \end{frame}
%%%%%%%%%%%%%%%%%%%%

%%%%%%%%%%%%%%%%%%%%
% \begin{frame}{}
%   \fig{width = 0.85\textwidth}{figs/FT-blog}

%   \begin{center}
%     \href{https://tomassetti.me/}{tomassetti.me}

%     \vspace{0.30cm}
%     专业人士,专业网站,推荐\red{\bf 订阅}
%   \end{center}
% \end{frame}
%%%%%%%%%%%%%%%%%%%%

%%%%%%%%%%%%%%%%%%%%
\begin{frame}{}
  \fig{width = 0.40\textwidth}{figs/dragon-book}
  \begin{center}
    也可使用\blue{\bf ``本科教学版''}
  \end{center}
\end{frame}
%%%%%%%%%%%%%%%%%%%%

%%%%%%%%%%%%%%%%%%%%
\begin{frame}{}
  \begin{columns}
    \column{0.33\textwidth}
    \fig{width = 0.80\textwidth}{figs/flex}
    \begin{center}
      \href{https://en.wikipedia.org/wiki/Flex_(lexical_analyser_generator)}{\footnotesize Flex: 词法分析器生成器}
    \end{center}
    \column{0.33\textwidth}
    \fig{width = 0.80\textwidth}{figs/lex-yacc-book}
    \column{0.33\textwidth}
    \fig{width = 0.80\textwidth}{figs/bison}
    \begin{center}
      \href{https://en.wikipedia.org/wiki/GNU_Bison}{\footnotesize Bison: 语法分析器生成器}
    \end{center}
  \end{columns}

  \vspace{0.50cm}
  \begin{center}
    不够现代, 本学期课程实验\red{\bf 不再支持}这些工具
  \end{center}
\end{frame}
%%%%%%%%%%%%%%%%%%%%

%%%%%%%%%%%%%%%%%%%%
\begin{frame}{}
  \begin{columns}
    \column{0.40\textwidth}
    \fig{width = 1.00\textwidth}{figs/antlr-logo}
    \begin{center}
      (Since 1988)
    \end{center}
    \column{0.60\textwidth}
    \fig{width = 0.50\textwidth}{figs/parr.jpeg}
    \begin{center}
      \href{https://parrt.cs.usfca.edu/}{\small Terence Parr (University of San Francisco)}
    \end{center}
  \end{columns}

  \vspace{0.80cm}
  \begin{center}
    \url{https://www.antlr.org/index.html} \\[5pt]
    \url{https://www.antlr.org/tools.html} (\red{IntelliJ Plugin}) \\[5pt]
    \url{http://lab.antlr.org/} (Online lab)
  \end{center}
\end{frame}
%%%%%%%%%%%%%%%%%%%%

%%%%%%%%%%%%%%%%%%%%
\begin{frame}{}
  \begin{columns}
    \column{0.50\textwidth}
    \fig{width = 0.70\textwidth}{figs/antlr4-book-en}
    \column{0.50\textwidth}
    \fig{width = 0.60\textwidth}{figs/antlr4-book-ch}
  \end{columns}

  \vspace{0.50cm}
  \begin{center}
    基于 ANTLR 4, 是课程实验指导的\red{\bf 重要}参考资料
  \end{center}
\end{frame}
%%%%%%%%%%%%%%%%%%%%

%%%%%%%%%%%%%%%%%%%%
\begin{frame}{}
  \begin{columns}
    \column{0.50\textwidth}
    \fig{width = 0.65\textwidth}{figs/patterns-book-en}
    \column{0.50\textwidth}
    \fig{width = 0.70\textwidth}{figs/patterns-book-ch}
  \end{columns}

  \vspace{0.50cm}
  \begin{center}
    基于 ANTLR 3, 与 ANTLR 4 相比有些过时,\\[3pt]
    但可以看作理解 ANTLR 4 的基础
  \end{center}
\end{frame}
%%%%%%%%%%%%%%%%%%%%

%%%%%%%%%%%%%%%%%%%%
\begin{frame}{}
  \fig{width = 0.40\textwidth}{figs/how-to-develop-a-compiler-book}
  \begin{center}
    ``从零开始制作真正的编译器'',对课程实验很有帮助,\red{\bf 强烈推荐}
  \end{center}
\end{frame}
%%%%%%%%%%%%%%%%%%%%

%%%%%%%%%%%%%%%%%%%%
\begin{frame}{}
  \fig{width = 0.50\textwidth}{figs/llvm-cookbook-book}

  \begin{center}
    从某次实验开始, 你就会开始接触 LLVM (\url{https://llvm.org/})
  \end{center}
\end{frame}
%%%%%%%%%%%%%%%%%%%%

%%%%%%%%%%%%%%%%%%%%
\begin{frame}{}
  \fig{width = 0.50\textwidth}{figs/llvm-logo}

  \vspace{0.20cm}
  \begin{center}
    \href{https://www.bilibili.com/video/BV1RF411K7F5/?vd_source=e3cbbf5ca80db268fa006d63626e267e}{LLVM 简介 @ Bilibili}
  \end{center}
\end{frame}
%%%%%%%%%%%%%%%%%%%%

%%%%%%%%%%%%%%%%%%%%
\begin{frame}{}
  \begin{center}
    更多参考书, 随课程进展陆续发布

    \fig{width = 0.40\textwidth}{figs/more-books}

    \pause
    \vspace{0.30cm}
    \url{http://docs.compilers.cpl.icu/\#/2024/resources}
    \fig{width = 0.40\textwidth}{figs/books}
  \end{center}
\end{frame}
%%%%%%%%%%%%%%%%%%%%