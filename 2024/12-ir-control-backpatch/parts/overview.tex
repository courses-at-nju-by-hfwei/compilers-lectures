% overview.tex

%%%%%%%%%%%%%%%%%%%%
\begin{frame}{}
	\begin{center}
		使用\blue{标签}标记跳转目标
	\end{center}

  \begin{columns}
    \column{0.50\textwidth}
      \fig{width = 0.80\textwidth}{figs/bool-short-circuit-II}
    \column{0.50\textwidth}
      \fig{width = 0.40\textwidth}{figs/bool-short-circuit-II-hard-code}
  \end{columns}
\end{frame}
%%%%%%%%%%%%%%%%%%%%

%%%%%%%%%%%%%%%%%%%%
\begin{frame}{}
	\begin{center}
		Java Bytecode: 使用\red{地址值}作为跳转目标
	\end{center}
  \begin{columns}
    \column{0.50\textwidth}
      \fig{width = 0.80\textwidth}{figs/java-code}
    \column{0.50\textwidth}
      \pause
      \fig{width = 0.45\textwidth}{figs/java-bytecode}
  \end{columns}

  \pause
	\begin{center}
		\red{$Q:$ 如何在一趟扫描中生成跳转目标的地址?}
	\end{center}
\end{frame}
%%%%%%%%%%%%%%%%%%%%

%%%%%%%%%%%%%%%%%%%%
\begin{frame}{}
  \begin{center}
    \teal{\texttt{\large if (x < 100 || x > 200 \&\& x != y) x = 0;}}

    \vspace{0.30cm}
    \fig{width = 0.50\textwidth}{figs/if-tac}

    $L_{1}$ 标签的位置是由 $P \to S \;(\to {\bf if} \;B\; S_{1})$ 确定的, \\[6pt]
    但是, 生成 $B$ 所对应的中间代码 {\bf goto} $L_{1}$ 时, 尚不知道 $L_{1}$ 的地址
  \end{center}
\end{frame}
%%%%%%%%%%%%%%%%%%%%

%%%%%%%%%%%%%%%%%%%%
\begin{frame}{}
  \begin{center}
    \teal{\texttt{\large if (x < 100 || x > 200 \&\& x != y) x = 0;}}

    \fig{width = 1.00\textwidth}{figs/if-boolexpr-ir}

    \vspace{-0.60cm}
    \pause
    \[
      B_{21}.\false \cgets B_{2}.\false \cgets B.\false \cgets S.\nextir = \teal{L_{1}}
    \]
    \[
      B_{22}.\false \cgets B_{2}.\false \cgets B.\false \cgets S.\nextir = \teal{L_{1}}
    \]
  \end{center}
\end{frame}
%%%%%%%%%%%%%%%%%%%%

%%%%%%%%%%%%%%%%%%%%
\begin{frame}{}
  \fig{width = 0.50\textwidth}{figs/big-idea}
\end{frame}
%%%%%%%%%%%%%%%%%%%%

%%%%%%%%%%%%%%%%%%%%
\begin{frame}{}
  \begin{center}
    \teal{\texttt{\large if (x < 100 || x > 200 \&\& x != y) x = 0;}}

    \fig{width = 1.00\textwidth}{figs/if-boolexpr-ir}

    \vspace{-0.60cm}
    \pause
    \[
      \set{\purple{103}} = B_{21}.\false \cto B_{2}.\false \cto B.\false \cto S.\nextir = \teal{107}
    \]
    \[
      \set{\purple{105}} = B_{22}.\false \cto B_{2}.\false \cto B.\false \cto S.\nextir = \teal{107}
    \]
  \end{center}
\end{frame}
%%%%%%%%%%%%%%%%%%%%

%%%%%%%%%%%%%%%%%%%%
% \begin{frame}{}
%   \begin{center}
%     \fig{width = 1.00\textwidth}{figs/SDD-if}

%     \vspace{0.80cm}
%     \uncover<2->{\blue{$B$ 可以自行计算 $B.\mathit{true}$ 对应的指令地址}}

%     \vspace{0.30cm}
%     \fig{width = 0.40\textwidth}{figs/if-code-block}
%     \vspace{0.30cm}

%     \uncover<3->{\red{但是, $B$ 计算不出 $B.\mathit{false}$ 对应的指令地址}}
%   \end{center}
% \end{frame}
%%%%%%%%%%%%%%%%%%%%

%%%%%%%%%%%%%%%%%%%%
\begin{frame}{}
  \begin{center}
    子节点暂时不指定跳转指令的目标地址
  \end{center}

  \begin{columns}
    \column{0.50\textwidth}
      \fig{width = 0.80\textwidth}{figs/if-tac-placeholder}
    \column{0.50\textwidth}
      \uncover<2->{\fig{width = 1.00\textwidth}{figs/is-it-a-big-idea}}
  \end{columns}

  \vspace{0.50cm}
  \begin{center}
    待祖先节点能够确定目标地址时回头填充
  \end{center}
\end{frame}
%%%%%%%%%%%%%%%%%%%%

%%%%%%%%%%%%%%%%%%%%
\begin{frame}{}
  \begin{center}
    \red{回填 (Backpatching) 技术}: {子节点挖坑、祖先节点填坑}

    \fig{width = 0.50\textwidth}{figs/wakeng}

    \pause
    \vspace{0.80cm}
    祖先节点通过\red{\bf 综合属性}\blue{收集}子节点中\purple{具有相同目标的跳转指令}
  \end{center}
\end{frame}
%%%%%%%%%%%%%%%%%%%%

%%%%%%%%%%%%%%%%%%%%
\begin{frame}{}
  \begin{center}
    南北爱情故事

    \fig{width = 0.65\textwidth}{figs/nju-pku}
  \end{center}
\end{frame}
%%%%%%%%%%%%%%%%%%%%

%%%%%%%%%%%%%%%%%%%%
\begin{frame}{}
	\fig{width = 0.65\textwidth}{figs/train-college-film}
\end{frame}
%%%%%%%%%%%%%%%%%%%%

%%%%%%%%%%%%%%%%%%%%
\begin{frame}{}
	\fig{width = 0.65\textwidth}{figs/train-college-cartoon}
\end{frame}
%%%%%%%%%%%%%%%%%%%%

%%%%%%%%%%%%%%%%%%%%
\begin{frame}{}
  \begin{center}
    {$B.\mathit{truelist}$ 保存{需要跳转到 $B.\mathit{true}$ 标签}的{指令}} \\[5pt]
    {$B.\mathit{falselist}$ 保存{需要跳转到 $B.\mathit{false}$ 标签}的{指令}}

    \fig{width = 0.50\textwidth}{figs/backpatch-impl}

    {$S/L.\mathit{nextlist}$} 保存{需要跳转到 $S/L.\nextir$ 标签的指令}
  \end{center}
\end{frame}
%%%%%%%%%%%%%%%%%%%%

%%%%%%%%%%%%%%%%%%%%
\begin{frame}{}
  \begin{center}
    为左部非终结符 \red{$B$} 计算综合属性 $B.\mathit{truelist}$ 与 $B.\mathit{falselist}$ \\[5pt]
    为左部非终结符 \blue{$S/L$} 计算综合属性 $S/L.\mathit{nextlist}$

    \fig{width = 0.50\textwidth}{figs/backpatch-impl}

    并为已能确定目标地址的跳转指令进行\purple{回填} (考虑每个综合属性)
  \end{center}
\end{frame}
%%%%%%%%%%%%%%%%%%%%