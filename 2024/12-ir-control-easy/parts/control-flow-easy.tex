% control-flow-easy.tex

%%%%%%%%%%%%%%%%%%%%
\begin{frame}{}
  \begin{center}
    各个非终结符的\red{\bf 分工、合作}明确易懂

    \vspace{0.30cm}
    \fig{width = 0.40\textwidth}{figs/easy-mode}
    \vspace{0.30cm}

    只需要使用\blue{\bf 综合属性}
  \end{center}
\end{frame}
%%%%%%%%%%%%%%%%%%%%

%%%%%%%%%%%%%%%%%%%%
\begin{frame}
  \fig{width = 0.80\textwidth}{figs/daju}
  \begin{center}
    心中有``树'' (语法分析树)
  \end{center}
\end{frame}
%%%%%%%%%%%%%%%%%%%%

%%%%%%%%%%%%%%%%%%%%
\begin{frame}{}
  \begin{center}
    {\Large 分工 \qquad 合作}

    \vspace{0.20cm}
    \fig{width = 0.60\textwidth}{figs/team-work-fly}
  \end{center}

  \begin{center}
    为布尔表达式 $B$ 计算逻辑值 (假设保存在临时变量 \texttt{t1} 中) \\[5pt]
    \texttt{\bf if}、\texttt{\bf while} 等语句
    根据 $B$ 的结果改变控制流
  \end{center}

  \vspace{-0.30cm}
  \[
    \texttt{\bf if}\; (B)\; S_{1}
  \]
\end{frame}
%%%%%%%%%%%%%%%%%%%%

%%%%%%%%%%%%%%%%%%%%
\begin{frame}{}
  \begin{center}
    \fig{width = 0.40\textwidth}{figs/talk-code}

    \vspace{0.30cm}
    \teal{\texttt{CodeGenVisitor.java}}
  \end{center}
\end{frame}
%%%%%%%%%%%%%%%%%%%%

%%%%%%%%%%%%%%%%%%%%
\begin{frame}
  \begin{center}
    实现顺序: \cyan{标识符 $E \to {\bf \texttt{id}}$},
      \blue{布尔表达式 $B$}、\violet{\texttt{if} 语句}、
      \red{\texttt{while} 语句}
  \end{center}

  \begin{columns}
    \column{0.50\textwidth}
      \fig{width = 0.40\textwidth}{figs/cfg-S-grammar}
    \column{0.50\textwidth}
      \fig{width = 0.35\textwidth}{figs/cfg-boolexpr-grammar}
  \end{columns}
\end{frame}
%%%%%%%%%%%%%%%%%%%%

%%%%%%%%%%%%%%%%%%%%
\begin{frame}
  \begin{center}
    实现方式: \textcolor{lightgray}{Listeners},
      \textcolor{blue}{Visitors},
      \textcolor{lightgray}{Attributed Grammar}
  \end{center}

  \begin{columns}
    \column{0.50\textwidth}
      \fig{width = 0.40\textwidth}{figs/cfg-S-grammar}
    \column{0.50\textwidth}
      \fig{width = 0.35\textwidth}{figs/cfg-boolexpr-grammar}
  \end{columns}

  \begin{center}
    \blue{及时输出生成的中间代码, 避免频繁的字符串拼接操作}
  \end{center}
\end{frame}
%%%%%%%%%%%%%%%%%%%%

%%%%%%%%%%%%%%%%%%%%
\begin{frame}
  \begin{center}
    \fig{width = 0.50\textwidth}{figs/break}

    \vspace{0.50cm}
    如何翻译 ``\texttt{break}'' 语句?
  \end{center}
\end{frame}
%%%%%%%%%%%%%%%%%%%%

%%%%%%%%%%%%%%%%%%%%
\begin{frame}
  \begin{center}
    \fig{width = 0.40\textwidth}{figs/ShortCircuit}

    \vspace{0.20cm}
    如何实现布尔表达式的 ``短路求值'' (Short-Circuit Evaluation)?
  \end{center}
\end{frame}
%%%%%%%%%%%%%%%%%%%%