% nfa-to-dfa.tex

%%%%%%%%%%%%%%%%%%%%
\begin{frame}{}
  \fig{width = 0.70\textwidth}{figs/re-dfa-lexer}

  \vspace{0.30cm}
  \begin{center}
    \blue{NFA $\implies$ DFA}
    \[
      \boxed{N \implies D}
    \]
    \[
      \text{要求}: \red{L(D) = L(N)}
    \]
  \end{center}
\end{frame}
%%%%%%%%%%%%%%%%%%%%

%%%%%%%%%%%%%%%%%%%%
\begin{frame}{}
  \begin{center}
    从 NFA 到 DFA 的转换: \red{\bf 子集构造法} (Subset/Powerset Construction)

    \vspace{0.50cm}
    \begin{columns}
      \column{0.50\textwidth}
        \fig{width = 0.45\textwidth}{figs/rabin}
        \begin{center}
          \teal{Michael O. Rabin (1931 $\sim$)}
        \end{center}
      \column{0.50\textwidth}
        \fig{width = 0.50\textwidth}{figs/scott}
        \begin{center}
          \teal{Dana Scott (1932 $\sim$)}
        \end{center}
    \end{columns}
  \end{center}

  \vspace{0.20cm}
  \begin{center}
    \blue{\bf 思想: 用 DFA 模拟 NFA}
  \end{center}
\end{frame}
%%%%%%%%%%%%%%%%%%%%

%%%%%%%%%%%%%%%%%%%%
% \begin{frame}{}
%   \begin{center}
%     \red{\bf 用 DFA 模拟 NFA}
%
%     \fig{width = 0.50\textwidth}{figs/nfa-abb}
%     \[
%       \blue{L(\mathcal{A}) = L((a|b)^{\ast}abb)}
%     \]
%     \fig{width = 0.50\textwidth}{figs/dfa-abb}
%   \end{center}
% \end{frame}
%%%%%%%%%%%%%%%%%%%%

%%%%%%%%%%%%%%%%%%%%
\begin{frame}{}
  \begin{center}
    \red{\bf 用 DFA 模拟 NFA}

    \fig{width = 0.75\textwidth}{figs/nfa-abb-epsilon}
    \vspace{-0.20cm}
    \[
      \blue{L(\mathcal{A}) = L((a|b)^{\ast}abb)}
    \]
    \vspace{-0.30cm}
    \fig{width = 0.50\textwidth}{figs/dfa-abb-subset}
  \end{center}
\end{frame}
%%%%%%%%%%%%%%%%%%%%

% %%%%%%%%%%%%%%%%%%%%
% \begin{frame}{}
%   \fig{width = 0.90\textwidth}{figs/re-nfa-abb-A}
%   \[
%     A \triangleq \epsilon\text{-closure}(0) = \set{0, 1, 2, 4, 7}
%   \]
%
%   \begin{center}
%     DFA $D$ 的\purple{\bf 初始状态} $d_{0} = \epsilon\text{-closure}(n_{0})$
%   \end{center}
% \end{frame}
% %%%%%%%%%%%%%%%%%%%%
%
% %%%%%%%%%%%%%%%%%%%%
% \begin{frame}{}
%   \fig{width = 0.90\textwidth}{figs/re-nfa-abb-B}
%
%   \[
%     A = \set{0, 1, 2, 4, 7}
%   \]
%   \[
%     B \triangleq \delta_{D}(A, a) = \epsilon\text{-closure}(\set{3, 8})
%     = \set{1, 2, 3, 4, 6, 7, 8}
%   \]
% \end{frame}
% %%%%%%%%%%%%%%%%%%%%
%
% %%%%%%%%%%%%%%%%%%%%
% \begin{frame}{}
%   \fig{width = 0.90\textwidth}{figs/re-nfa-abb-C}
%
%   \[
%     A = \set{0, 1, 2, 4, 7}
%   \]
%   \[
%     C \triangleq \delta_{D}(A, b) = \epsilon\text{-closure}(5)
%     = \set{1, 2, 4, 5, 6, 7}
%   \]
% \end{frame}
% %%%%%%%%%%%%%%%%%%%%
%
% %%%%%%%%%%%%%%%%%%%%
% \begin{frame}{}
% \end{frame}
% %%%%%%%%%%%%%%%%%%%%
%
% %%%%%%%%%%%%%%%%%%%%
% \begin{frame}{}
%   \fig{width = 0.90\textwidth}{figs/re-nfa-abb-D}
%
%   \[
%     B = \set{1, 2, 4, 5, 6, 7}
%   \]
%   \[
%     D \triangleq \delta_{D}(B, b) = \epsilon\text{-closure}(\set{5, 9})
%     = \set{1, 2, 4, 5, 6, 7, 9}
%   \]
% \end{frame}
% %%%%%%%%%%%%%%%%%%%%

%%%%%%%%%%%%%%%%%%%%
\begin{frame}{}
  \begin{center}
    \fig{width = 0.40\textwidth}{figs/blackboard}

    \vspace{0.30cm}
    板书演示算法过程
  \end{center}
\end{frame}
%%%%%%%%%%%%%%%%%%%%

%%%%%%%%%%%%%%%%%%%%
\begin{frame}{}
  \fig{width = 0.50\textwidth}{figs/nfa-dfa-table}
  \fig{width = 0.60\textwidth}{figs/dfa-abb-subset}
\end{frame}
%%%%%%%%%%%%%%%%%%%%

%%%%%%%%%%%%%%%%%%%%
\begin{frame}{}
  \begin{center}
    从状态 $s$ 开始, 只通过 $\epsilon$-转移可达的状态集合
    \[
      \epsilon\text{-closure}(s) = \set{t \in S_{N} | s \rel{\epsilon^{\ast}} t}
    \]

    \pause
    \vspace{0.30cm}
    \[
      \epsilon\text{-closure}(T) = \bigcup_{s \in T} \epsilon\text{-closure}(s)
    \]

    \pause
    \vspace{0.30cm}
    \[
      \text{move}(T, a) = \bigcup_{s \in T} \delta(s, a)
    \]
  \end{center}
\end{frame}
%%%%%%%%%%%%%%%%%%%%

%%%%%%%%%%%%%%%%%%%%
\begin{frame}{}
  \begin{center}
    子集构造法 ($N \implies D$) 的\red{\bf 原理}:
  \end{center}
  \[
    \blue{N}: (\Sigma_{N}, S_{N}, n_{0}, \delta_{N}, F_{N})
  \]
  \[
    \red{D}: (\Sigma_{D}, S_{D}, d_{0}, \delta_{D}, F_{D})
  \]

  \pause
  \[
    \Sigma_{D} = \Sigma_{N}
  \]

  \pause
  \[
    S_{D} \subseteq 2^{S_{N}} \qquad (\forall s_{D} \in S_{D}: \blue{s_{D} \subseteq S_{N}})
  \]

  \pause
  \[
    \text{\purple{\bf 初始状态}}\;
    {d_{0} = \epsilon\text{-closure}(n_{0})}
  \]
  \pause
  \vspace{-0.30cm}
  \[
    \text{\purple{\bf 转移函数}}\;
    \forall a \in \Sigma_{D}:
    {\delta_{D}(s_{D}, a) = \epsilon\text{-closure}(\text{move}(s_{D}, a))}
  \]
  \pause
  \vspace{-0.30cm}
  \[
    \text{\purple{\bf 接受状态集}}\;
    F_{\mathcal{D}} = \set{s_{D} \in S_{\mathcal{D}} \mid
      \exists f \in F_{N}.\; f \in s_{D}}
  \]
\end{frame}
%%%%%%%%%%%%%%%%%%%%

%%%%%%%%%%%%%%%%%%%%
% \begin{frame}{}
%   \begin{center}
%     子集构造法 ($N \implies D$) 的\red{\bf 实现}:
%     使用\blue{\bf 栈}实现\blue{$\epsilon\text{-closure}(T)$}
%
%     \fig{width = 0.90\textwidth}{figs/epsilon-closure-impl}
%   \end{center}
% \end{frame}
%%%%%%%%%%%%%%%%%%%%

%%%%%%%%%%%%%%%%%%%%
% \begin{frame}{}
%   \begin{center}
%     子集构造法 ($N \implies D$) 的\red{\bf 实现}:
%     使用\blue{\bf 标记搜索}过程构造\blue{\bf 状态集}
%
%     \fig{width = 0.90\textwidth}{figs/subset-construction}
%   \end{center}
% \end{frame}
%%%%%%%%%%%%%%%%%%%%

%%%%%%%%%%%%%%%%%%%%
\begin{frame}{}
  \begin{center}
    子集构造法的\red{\bf 复杂度分析}: \\
    ($|S_{N}| = n$)

    \[
      |S_{D}| = \Theta(2^n) = O(2^n) \cap \Omega(2^n)
    \]

    \vspace{0.50cm}
    \red{对于任何算法}, 最坏情况下, $|S_{D}| = \Omega(2^n)$
  \end{center}
\end{frame}
%%%%%%%%%%%%%%%%%%%%

%%%%%%%%%%%%%%%%%%%%
\begin{frame}{}
  \begin{center}
    ``长度为 $m \ge n$个字符的$a$, $b$串, 且倒数第$n$个字符是$a$''

    \pause
    \[
      L_{n} = (a | b)^{\ast} a (a | b)^{\red{n-1}}
    \]

    \pause
    \fig{width = 0.80\textwidth}{figs/nfa-Ln}

    \pause
    \vspace{0.30cm}
    $n = 4$
  \end{center}
\end{frame}
%%%%%%%%%%%%%%%%%%%%

%%%%%%%%%%%%%%%%%%%%
\begin{frame}{}
  \fig{width = 0.90\textwidth}{figs/nfa-dfa-blown-up}
\end{frame}
%%%%%%%%%%%%%%%%%%%%

%%%%%%%%%%%%%%%%%%%%
\begin{frame}{}
  \begin{center}
    \red{\bf 闭包 (Closure):} $\purple{f}\text{-closure}(\blue{T})$

    \pause
    \[
      \epsilon\text{-closure}(T)
	%   \qquad \pause R^{+}: \text{传递闭包}
    \]

    \pause
    \[
      T \implies f(T) \implies f(f(T)) \implies f(f(f(T))) \implies \dots
    \]

    \pause
    直到找到 $x$ 使得 $f(x) = x$ ($x$ 称为 $f$ 的\red{\bf 不动点})
  \end{center}
\end{frame}
%%%%%%%%%%%%%%%%%%%%