% re.tex

%%%%%%%%%%%%%%%%%%%%
\begin{frame}{}
  \[
    \id: \red{L(L \cup D)^{\ast}}
  \]

  \begin{center}
    该如何告诉 ANTLR v4 $:$ 这个集合就是 \id{} 呢?
  \end{center}

  \pause
  \fig{width = 0.50\textwidth}{figs/regex}
  \begin{center}
    下面向大家隆重介绍简洁、优雅、强大的\red{\bf 正则表达式}
  \end{center}
\end{frame}
%%%%%%%%%%%%%%%%%%%%

%%%%%%%%%%%%%%%%%%%%
\begin{frame}{}
  \begin{center}
    每个正则表达式 $r$ 对应一个正则语言 $L(r)$

    \vspace{0.30cm}
    \fig{width = 0.50\textwidth}{figs/syntax-semantics}

    \vspace{0.30cm}
    正则表达式是\red{\bf 语法}, 正则语言是\red{\bf 语义}
  \end{center}
\end{frame}
%%%%%%%%%%%%%%%%%%%%

%%%%%%%%%%%%%%%%%%%%
\begin{frame}{}
  \begin{definition}[正则表达式]
    给定字母表 $\Sigma$, $\Sigma$ 上的正则表达式\red{\bf 由且仅由}以下规则定义:
    \begin{enumerate}[(1)]
      \setlength{\itemsep}{8pt}
      \item $\epsilon$ 是正则表达式;
      \item $\forall a \in \Sigma$, $a$ 是正则表达式;
      \item 如果 $r$ 是正则表达式, 则 $(r)$ 是正则表达式;
      \item 如果 $r$ 与 $s$ 是正则表达式, 则 $r|s$, $rs$, $r^{\ast}$ 也是正则表达式。
    \end{enumerate}

    \vspace{0.30cm}
    \begin{center}
      运算优先级: $()\; \purple{\succ} \ast\; \purple{\succ} \text{ 连接 }\; \purple{\succ}\; |$
    \end{center}
    \[
      (a) | ((b)^{\ast}(c)) \equiv a | b^{\ast} c
    \]
  \end{definition}
\end{frame}
%%%%%%%%%%%%%%%%%%%%

%%%%%%%%%%%%%%%%%%%%
\begin{frame}{}
  \begin{center}
    每个正则表达式 $r$ 对应一个正则语言 $L(r)$
  \end{center}

  \begin{definition}[正则表达式对应的正则语言]
    \begin{gather}
      L(\epsilon) = \set{\epsilon} \\[8pt]
      L(a) = \set{a}, \forall a \in \Sigma \\[8pt]
      L((r)) = L(r) \\[8pt]
      \red{L(r|s) = L(r) \cup L(s) \quad L(rs) = L(r)L(s)
      \quad L(r^{\ast}) = (L(r))^{\ast}}
    \end{gather}
  \end{definition}
\end{frame}
%%%%%%%%%%%%%%%%%%%%

%%%%%%%%%%%%%%%%%%%%
\begin{frame}{}
  \[
    \Sigma = \set{a, b}
  \]

  \[
    L(a | b) = \set{a, b}
  \]

  \pause
  \[
    L((a | b) (a | b))
  \]

  \pause
  \[
    L(a^{\ast})
  \]

  \pause
  \[
    L((a | b)^{\ast})
  \]

  \pause
  \[
    L(a | a^{\ast}b)
  \]
\end{frame}
%%%%%%%%%%%%%%%%%%%%

%%%%%%%%%%%%%%%%%%%%
\begin{frame}{}
  \fig{width = 0.50\textwidth}{figs/so-easy}
\end{frame}
%%%%%%%%%%%%%%%%%%%%

%%%%%%%%%%%%%%%%%%%%
\begin{frame}{}
  \fig{width = 0.75\textwidth}{figs/regex-extended}
  \pause
  \[
    [0-9] \quad [a-zA-Z] \qquad \red{\hat\qquad \$}
  \]
\end{frame}
%%%%%%%%%%%%%%%%%%%%

%%%%%%%%%%%%%%%%%%%%
\begin{frame}{}
  \begin{center}
    正则表达式\red{\bf 简记法}
  \end{center}

  \fig{width = 0.92\textwidth}{figs/regex-shorthand}
\end{frame}
%%%%%%%%%%%%%%%%%%%%

%%%%%%%%%%%%%%%%%%%%
\begin{frame}{}
  \fig{width = 0.40\textwidth}{figs/too-easy}
\end{frame}
%%%%%%%%%%%%%%%%%%%%

%%%%%%%%%%%%%%%%%%%%
\begin{frame}{}
  \[
    \Big(0|\big(1(01^{\ast}0)^{\ast}1\big)\Big)^{\ast}
  \]
  \fig{width = 0.40\textwidth}{figs/sha}
\end{frame}
%%%%%%%%%%%%%%%%%%%%

%%%%%%%%%%%%%%%%%%%%
\begin{frame}{}
  \begin{columns}
    \column{0.50\textwidth}
      \fig{width = 0.50\textwidth}{figs/regex101-multiple-3}
    \column{0.50\textwidth}
      \begin{center}
        {\small \teal{\url{https://regex101.com/r/ED4qgC/1}}}
      \end{center}
      \fig{width = 0.70\textwidth}{figs/regex101}
      \pause
      \begin{center}
        3的倍数 (二进制表示)
      \end{center}
  \end{columns}
\end{frame}
%%%%%%%%%%%%%%%%%%%%