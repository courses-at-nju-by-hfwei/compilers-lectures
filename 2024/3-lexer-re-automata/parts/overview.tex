% overview.tex

%%%%%%%%%%%%%%%%%%%%
\begin{frame}{}
  \fig{width = 0.70\textwidth}{figs/dfa-lexer}
  \begin{center}
    自动化词法分析器生成器
  \end{center}
\end{frame}
%%%%%%%%%%%%%%%%%%%%

%%%%%%%%%%%%%%%%%%%%
\begin{frame}{}
  \begin{columns}
    \column{0.40\textwidth}
      \fig{width = 0.85\textwidth}{figs/antlr-logo}
    \column{0.60\textwidth}
      \fig{width = 0.80\textwidth}{figs/simpleexpr-lexer-g4}
  \end{columns}
\end{frame}
%%%%%%%%%%%%%%%%%%%%

%%%%%%%%%%%%%%%%%%%%
\begin{frame}{}
  \begin{columns}
    \column{0.50\textwidth}
      \begin{center}
        \fig{width = 0.80\textwidth}{figs/ws}
        \fig{width = 1.00\textwidth}{figs/number}
        \fig{width = 0.80\textwidth}{figs/id}
      \end{center}
    \column{0.50\textwidth}
      \begin{center}
        \fig{width = 0.70\textwidth}{figs/relop}
        \fig{width = 0.60\textwidth}{figs/error-code}
      \end{center}
  \end{columns}

  \vspace{0.30cm}
  \begin{center}
    \red{\bf 关键点:} 合并$22, 12, 9, 0$, 根据\purple{\bf 下一个字符}即可判定词法单元的类型 \\[4pt]
    否则, 调用错误处理模块(对应\texttt{other}), 报告\purple{\bf 该字符有误}, 忽略该字符。 \\[4pt]
    注意, 在 \floatnum{} 与 \scinum{} 中, 有时需要\purple{\bf 回退}, 寻找最长匹配。
  \end{center}
\end{frame}
%%%%%%%%%%%%%%%%%%%%

%%%%%%%%%%%%%%%%%%%%
% \begin{frame}{}
%   \begin{center}
%     \blue{\large 自动机} \\[5pt]
%     (Automaton; Automata)

%     \fig{width = 0.50\textwidth}{figs/off-on}
%     \vspace{-0.30cm}
%     ``开关''自动机

%     \vspace{0.80cm}
%     两大要素: {\bf 状态集} $S$ 以及{\bf 状态转移函数} $\delta$
%   \end{center}
% \end{frame}
%%%%%%%%%%%%%%%%%%%%

%%%%%%%%%%%%%%%%%%%%
\begin{frame}{}
  \begin{center}
    \red{\bf 目标: 正则表达式 RE $\implies$ 词法分析器}

    \vspace{0.30cm}
    \fig{width = 0.70\textwidth}{figs/re-dfa-lexer}

    \vspace{0.50cm}
    \purple{终点固然令人向往, 这一路上的风景更是美不胜收}
  \end{center}
\end{frame}
%%%%%%%%%%%%%%%%%%%%