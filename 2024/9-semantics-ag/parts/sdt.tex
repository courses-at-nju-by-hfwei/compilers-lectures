% sdt.tex

%%%%%%%%%%%%%%%%%%%%
\begin{frame}{}
  \begin{definition}[语法制导的翻译方案 (Syntax-Directed Translation Scheme; SDT)]
    \purple{\bf SDT} 是在其产生式体中嵌入\red{\bf 语义动作}的上下文无关文法。
  \end{definition}

  \vspace{0.30cm}
  \begin{columns}
    \column{0.50\textwidth}
      \fig{width = 1.00\textwidth}{figs/SDD-expr-left-recursion-rule}
    \column{0.50\textwidth}
      \fig{width = 1.00\textwidth}{figs/SDT-expr-left-recursion}
  \end{columns}

  \pause
  \vspace{0.30cm}
  \begin{center}
    \red{$Q:$} 如何将带有\blue{\bf 语义规则}的 SDD 转换为带有\blue{\bf 语义动作}的 SDT
  \end{center}
\end{frame}
%%%%%%%%%%%%%%%%%%%%

%%%%%%%%%%%%%%%%%%%%
\begin{frame}{}
  \begin{center}
    \blue{\bf 时机} (Timing; タイミング)

    \fig{width = 0.60\textwidth}{figs/timing-japanese}

    \vspace{0.30cm}
    \red{\bf 语义动作嵌入在什么地方? 这决定了何时执行语义动作。}
  \end{center}
\end{frame}
%%%%%%%%%%%%%%%%%%%%

%%%%%%%%%%%%%%%%%%%%
\begin{frame}{}
  \begin{center}
    \begin{columns}
      \column{0.50\textwidth}
        \begin{center}
          \blue{\bf $S$ 属性定义}
        \end{center}
        \fig{width = 1.00\textwidth}{figs/SDD-expr-left-recursion-rule}
      \column{0.50\textwidth}
        \begin{center}
          \red{\bf 后缀翻译方案}
        \end{center}
        \fig{width = 1.00\textwidth}{figs/SDT-expr-left-recursion}
    \end{columns}

    \vspace{0.80cm}
    \red{\bf 后缀翻译方案:} 所有动作都在产生式的最后
  \end{center}
\end{frame}
%%%%%%%%%%%%%%%%%%%%

%%%%%%%%%%%%%%%%%%%%
\begin{frame}{}
  \begin{center}
    \red{\bf $L$ 属性定义} 与 \blue{\bf $LL$ 语法分析}

    \vspace{0.30cm}
    \fig{width = 0.80\textwidth}{figs/dep-expr-no-left-recursion}
    \vspace{-0.20cm}
    \[
      \teal{3 \ast 5}
    \]

    \vspace{-0.50cm}
    \[
      \blue{A \to X_{1} \cdots X_{i} \cdots X_{n}}
    \]

    \red{\bf 原则:} \teal{\bf 从左到右}处理各个 $X_{i}$ 符号

    \vspace{0.10cm}
    对每个 $X_{i}$, 先计算\teal{\bf 继承属性}, 后计算\teal{\bf 综合属性}
  \end{center}
\end{frame}
%%%%%%%%%%%%%%%%%%%%

%%%%%%%%%%%%%%%%%%%%
\begin{frame}{}
  \begin{center}
    \red{\bf 递归下降子过程 $A \to X_{1} \cdots X_{i} \cdots X_{n}$}

    \vspace{0.80cm}
    \begin{itemize}
      \centering
      \setlength{\itemsep}{15pt}
      \item 在调用 $X_{i}$ 子过程之前, 计算 $X_{i}$ 的\red{\bf 继承属性}
      \item 以 $X_{i}$ 的继承属性为\blue{\bf 参数}调用 $X_{i}$ 子过程
      \vspace{10pt}
      \item 在 $X_{i}$ 子过程返回之前, 计算 $X_{i}$ 的\red{\bf 综合属性}
      \item 在 $X_{i}$ 子过程结束时\blue{\bf 返回} $X_{i}$ 的综合属性
    \end{itemize}
  \end{center}
\end{frame}
%%%%%%%%%%%%%%%%%%%%

%%%%%%%%%%%%%%%%%%%%
\begin{frame}{}
  \begin{center}
    \fig{width = 0.90\textwidth}{figs/dep-expr-left-right-recursion}
  \end{center}

  \vspace{-1.00cm}
  \[
    \teal{\boxed{X Y^{\ast}}}
  \]

  \begin{columns}
    \column{0.35\textwidth}
      \centerline{(左递归) $S$ 属性定义}
      \vspace{-0.80cm}
      % left-recursion-S-SDD.tex

\begin{alignat*}{3}
  A &\to A_{1} Y \quad && A.a = g(A_{1}.a, Y.y) \\[8pt]
  A &\to X && A.a = f(X.x)
\end{alignat*}
    \column{0.65\textwidth}
      \pause
      \centerline{(右递归) $L$ 属性定义}
      \vspace{-0.80cm}
      {% right-recursion-L-SDD.tex

\begin{alignat*}{3}
  A &\to XR \quad && \red{R.i} = f(X.x);\; A.a = R.s \\[8pt]
  R &\to YR_{1} && \red{R_{1}.i} = g(R.i, Y.y);\; R.s = R_{1}.s \\[8pt]
  R &\to \epsilon && \blue{R.s = R.i}
\end{alignat*}}
  \end{columns}
\end{frame}
%%%%%%%%%%%%%%%%%%%%

%%%%%%%%%%%%%%%%%%%%
\begin{frame}{}
  \begin{center}
    (右递归) $L$属性定义
    \vspace{-0.50cm}
    % right-recursion-L-SDD.tex

\begin{alignat*}{3}
  A &\to XR \quad && \red{R.i} = f(X.x);\; A.a = R.s \\[8pt]
  R &\to YR_{1} && \red{R_{1}.i} = g(R.i, Y.y);\; R.s = R_{1}.s \\[8pt]
  R &\to \epsilon && \blue{R.s = R.i}
\end{alignat*}

    \vspace{0.30cm}
    \red{\bf 原则: 继承属性在处理文法符号之前, 综合属性在处理文法符号之后}

    \pause
    \vspace{0.50cm}
    $L$ 属性定义的 SDT
    \vspace{-0.50cm}
    % SDT-right-recursion-L-SDD.tex

\begin{align*}
  A &\to X \quad \set{\red{R.i} = f(X.x)} \quad R \quad \set{A.a = R.s} \\[8pt]
  R &\to Y \quad \set{\red{R_{1}.i} = g(R.i, Y.y)} \quad R_{1} \quad \set{R.s = R_{1}.s} \\[8pt]
  R &\to \epsilon \quad \set{\blue{R.s = R.i}}
\end{align*}
  \end{center}
\end{frame}
%%%%%%%%%%%%%%%%%%%%

%%%%%%%%%%%%%%%%%%%%
\begin{frame}{}
  \begin{center}
    % SDT-right-recursion-L-SDD.tex

\begin{align*}
  A &\to X \quad \set{\red{R.i} = f(X.x)} \quad R \quad \set{A.a = R.s} \\[8pt]
  R &\to Y \quad \set{\red{R_{1}.i} = g(R.i, Y.y)} \quad R_{1} \quad \set{R.s = R_{1}.s} \\[8pt]
  R &\to \epsilon \quad \set{\blue{R.s = R.i}}
\end{align*}

    % L-SDT-LL-A.tex

\begin{algorithm}[H]
% \caption{}
% \label{alg:L-SDT-LL-A}
\begin{algorithmic}[1]
  \Procedure{\purple{$A$}}{\null} \Comment{$A$ 是开始符号, 无需继承属性做参数}
    \If{\texttt{token} = ?} \Comment{假设选择 $A \to XR$ 产生式}
      \State $X.x \gets \Call{match}{X}$ \Comment{假设 $X$ 是终结符, 返回综合属性}
      \State $\red{R.i} \gets f(X.x)$ \Comment{先计算 \red{$R.i$} 继承属性}
      \State $\blue{R.s} \gets R(R.i)$ \Comment{递归调用子过程 $R(R.i)$}
      \State \purple{\Return $R.s$} \Comment{返回 $A.a \gets R.s$ 综合属性}
    \EndIf
  \EndProcedure
\end{algorithmic}
\end{algorithm}
  \end{center}
\end{frame}
%%%%%%%%%%%%%%%%%%%%

%%%%%%%%%%%%%%%%%%%%
\begin{frame}{}
  \begin{center}
    % SDT-right-recursion-L-SDD.tex

\begin{align*}
  A &\to X \quad \set{\red{R.i} = f(X.x)} \quad R \quad \set{A.a = R.s} \\[8pt]
  R &\to Y \quad \set{\red{R_{1}.i} = g(R.i, Y.y)} \quad R_{1} \quad \set{R.s = R_{1}.s} \\[8pt]
  R &\to \epsilon \quad \set{\blue{R.s = R.i}}
\end{align*}

    % L-SDT-LL-R.tex

\begin{algorithm}[H]
% \caption{}
% \label{alg:L-SDT-LL-R}
\begin{algorithmic}[1]
  \Procedure{\purple{$R$}}{\blue{$R.i$}} \Comment{$R$ 使用继承属性 \blue{$R.i$} 做参数}
    \If{\texttt{token} = ?} \Comment{假设选择 $R \to YR$ 产生式}
      \State $Y.y \gets \Call{match}{Y}$ \Comment{假设 $Y$ 是终结符, 返回综合属性}
      \State $\red{R.i} \gets g(R.i, Y.y)$ \Comment{先计算 \red{$R.i$} 继承属性}
      \State $\blue{R.s} \gets R(R.i)$ \Comment{递归调用子过程 $R(R.i)$}
      \State \purple{\Return $R.s$} \Comment{返回综合属性}
    \ElsIf{\texttt{token} = ?} \Comment{假设选择 $R \to \epsilon$ 产生式}
      \State \purple{\Return $R.i$} \Comment{返回 $R.s \gets R.i$ 综合属性}
    \EndIf
  \EndProcedure
\end{algorithmic}
\end{algorithm}
  \end{center}
\end{frame}
%%%%%%%%%%%%%%%%%%%%

%%%%%%%%%%%%%%%%%%%%
% csv.tex

%%%%%%%%%%%%%%%%%%%%
\begin{frame}{}
  \begin{center}
    \fig{width = 0.50\textwidth}{figs/csv}
    \blue{\bf Comma-Separated Values}
  \end{center}
\end{frame}
%%%%%%%%%%%%%%%%%%%%

%%%%%%%%%%%%%%%%%%%%
\begin{frame}{}
  \begin{columns}
    \column{0.40\textwidth}
      \fig{width = 0.80\textwidth}{figs/csv-scores-input}
    \column{0.60\textwidth}
      \uncover<2->{\fig{width = 0.90\textwidth}{figs/csv-scores-output}}
  \end{columns}

  \vspace{0.30cm}
  \fig{width = 1.00\textwidth}{figs/csv-tree}

  \fig{width = 0.40\textwidth}{figs/csv-g4}
\end{frame}
%%%%%%%%%%%%%%%%%%%%

%%%%%%%%%%%%%%%%%%%%
\begin{frame}{}
  \fig{width = 1.00\textwidth}{figs/CSVAG-g4}

	\begin{center}
		\teal{\texttt{CSVAG.g4}}
	\end{center}
\end{frame}
%%%%%%%%%%%%%%%%%%%%

%%%%%%%%%%%%%%%%%%%%
% \begin{frame}{}
%   \fig{width = 1.00\textwidth}{figs/csv-file-rule}
% \end{frame}
%%%%%%%%%%%%%%%%%%%%

%%%%%%%%%%%%%%%%%%%%
% \begin{frame}{}
%   \fig{width = 0.85\textwidth}{figs/csv-hdr-rule}
% \end{frame}
%%%%%%%%%%%%%%%%%%%%

%%%%%%%%%%%%%%%%%%%%
% \begin{frame}{}
%   \fig{width = 0.85\textwidth}{figs/csv-row-rule}
% \end{frame}
%%%%%%%%%%%%%%%%%%%%  % left as hw?
%%%%%%%%%%%%%%%%%%%%