% overview.tex

%%%%%%%%%%%%%%%%%%%%
% \begin{frame}{}
%   \begin{center}
%     \blue{\Large 第 5 章: 语法制导的翻译} \\[10pt]
%     {\large (Syntax-Directed Translation)}

%     \fig{width = 0.50\textwidth}{figs/directed}

%     % \pause
%     % \vspace{0.30cm}
%     % {\Large 语法``\red{\bf 指导}''的翻译}

%     % \pause
%     \vspace{0.30cm}
%     {\large 依据语法结构进行翻译, 或在语法分析过程中进行翻译}
%   \end{center}
% \end{frame}
%%%%%%%%%%%%%%%%%%%%

%%%%%%%%%%%%%%%%%%%%
% \begin{frame}{}
%   \fig{width = 0.55\textwidth}{figs/dragon-book-2}
% \end{frame}
%%%%%%%%%%%%%%%%%%%%

%%%%%%%%%%%%%%%%%%%%
\begin{frame}{}
  \begin{center}
    Regular Expression (\purple{词法分析})

    \vspace{0.30cm}
    Context-Free Grammar (\purple{语法分析})

    \fig{width = 0.40\textwidth}{figs/recall}

    \pause
    \vspace{0.20cm}
    {用什么样的文法刻画语言的\red{\bf 语义}?}

    \vspace{0.30cm}
    \blue{(类型检查、符号检查)}
  \end{center}
\end{frame}
%%%%%%%%%%%%%%%%%%%%

%%%%%%%%%%%%%%%%%%%%
\begin{frame}{}
  \fig{width = 0.60\textwidth}{figs/knuth}

  \begin{center}
    Donald Knuth ($1938 \sim$)

    \vspace{0.30cm}
    \blue{(Turing Award, 1974)}
  \end{center}
\end{frame}
%%%%%%%%%%%%%%%%%%%%

%%%%%%%%%%%%%%%%%%%%
\begin{frame}{}
  \fig{width = 0.70\textwidth}{figs/knuth-paper-1967}

  \begin{center}
    \red{\bf 属性文法 (Attribute Grammar):} 为上下文无关文法赋予\blue{\bf 语义}
  \end{center}
\end{frame}
%%%%%%%%%%%%%%%%%%%%

%%%%%%%%%%%%%%%%%%%%
\begin{frame}{}
  \fig{width = 0.60\textwidth}{figs/talk-cheap}
\end{frame}
%%%%%%%%%%%%%%%%%%%%

%%%%%%%%%%%%%%%%%%%%
\begin{frame}{}
  \fig{width = 0.45\textwidth}{figs/antlr4-book-ch}
  \begin{center}
    \blue{\bf 第 10 章: 属性和动作}

    \vspace{0.30cm}
    \purple{\bf (第 15.4 节: 动作和属性)}
  \end{center}
\end{frame}
%%%%%%%%%%%%%%%%%%%%

%%%%%%%%%%%%%%%%%%%%
\begin{frame}{}
  \fig{width = 0.40\textwidth}{figs/calculator}
  \begin{center}
    \blue{\bf (交互式) 迷你计算器}

    \vspace{0.50cm}
    \teal{\texttt{Expr.g4}}
  \end{center}
\end{frame}
%%%%%%%%%%%%%%%%%%%%

%%%%%%%%%%%%%%%%%%%%
\begin{frame}{}
  \begin{columns}
    \column{0.50\textwidth}
      \fig{width = 0.20\textwidth}{figs/calc-input}
    \column{0.50\textwidth}
      \fig{width = 0.15\textwidth}{figs/calc-output}
  \end{columns}

  \fig{width = 1.00\textwidth}{figs/parsetree-calc}
\end{frame}
%%%%%%%%%%%%%%%%%%%%

%%%%%%%%%%%%%%%%%%%%
\begin{frame}{}
  \begin{center}
    \fig{width = 0.40\textwidth}{figs/mma-logo}

    \vspace{0.50cm}
    \blue{\bf 交互式}
  \end{center}
\end{frame}
%%%%%%%%%%%%%%%%%%%%

%%%%%%%%%%%%%%%%%%%%
\begin{frame}{}
  \begin{center}
    \blue{\bf Offline 方式计算属性值:} 已有语法分析树 (\teal{\texttt{calc}})

    \vspace{0.50cm}
    \fig{width = 0.50\textwidth}{figs/dfs}

    \pause
    \vspace{0.50cm}
    按照\red{\bf 从左到右}的\red{\bf 深度优先}顺序遍历语法分析树

    \vspace{0.50cm}
    \red{\bf 关键:} 在合适的时机执行合适的动作,计算相应的属性值
  \end{center}
\end{frame}
%%%%%%%%%%%%%%%%%%%%

%%%%%%%%%%%%%%%%%%%%
\begin{frame}{}
  \begin{center}
    在\blue{\bf 语法分析过程中}实现\blue{\bf 属性文法}
  \end{center}

  \[
    B \to \blue{X} \red{\set{a}} Y
  \]

  \pause
  \vspace{0.80cm}
  \begin{center}
    语义动作嵌入的位置决定了\red{\bf 何时}执行该动作

    \vspace{0.60cm}
    \red{\bf 基本思想:} 一个动作在它\cyan{\bf 左边的}所有文法符号都\cyan{\bf 处理}过之后立刻执行
  \end{center}
\end{frame}
%%%%%%%%%%%%%%%%%%%%

%%%%%%%%%%%%%%%%%%%%
\begin{frame}{}
  \begin{center}
    \fig{width = 1.00\textwidth}{figs/ExprAG-g4}

    \vspace{0.30cm}
    \teal{\texttt{ExprAG.g4}}
  \end{center}
\end{frame}
%%%%%%%%%%%%%%%%%%%%

%%%%%%%%%%%%%%%%%%%%
\begin{frame}{}
  \fig{width = 0.40\textwidth}{figs/calculator}
  \begin{center}
    \blue{\bf \red{(交互式)} 迷你计算器}

    \vspace{0.50cm}
    \teal{\texttt{ExprAGMain.java}}
  \end{center}
\end{frame}
%%%%%%%%%%%%%%%%%%%%

%%%%%%%%%%%%%%%%%%%%
\begin{frame}{}
  \begin{center}
    ``源码面前, 了无秘密''

    \fig{width = 0.50\textwidth}{figs/hj}

    \teal{\texttt{ExprAGParser.java}}
  \end{center}
\end{frame}
%%%%%%%%%%%%%%%%%%%%

% vars-decl.tex

%%%%%%%%%%%%%%%%%%%%
\begin{frame}{}
  \begin{center}
    \red{\bf 类型声明}文法举例

    \fig{width = 0.70\textwidth}{figs/SDD-type-decl}
    \[
      \teal{\floatkw\; \id_{1}, \id_{2}, \id_{3}}
    \]
  \end{center}
\end{frame}
%%%%%%%%%%%%%%%%%%%%

%%%%%%%%%%%%%%%%%%%%
\begin{frame}{}
  \begin{center}
    \red{$L.inh$} 将声明的类型沿着标识符列表向下传递

    \fig{width = 0.70\textwidth}{figs/anno-type-decl}
    \[
        \teal{\floatkw\; \id_{1}, \id_{2}, \id_{3}}
    \]
  \end{center}
\end{frame}
%%%%%%%%%%%%%%%%%%%%

%%%%%%%%%%%%%%%%%%%%
\begin{frame}{}
  \begin{center}
    \url{https://stackoverflow.com/q/76062088/1833118}
  \end{center}
  \fig{width = 0.95\textwidth}{figs/VarsDeclAG-g4}

  \pause
  \begin{center}
    Fortunately, you can rewrite it as \blue{\texttt{vars : ID (',' ID)*}}
  \end{center}
\end{frame}
%%%%%%%%%%%%%%%%%%%%

%%%%%%%%%%%%%%%%%%%%
\begin{frame}{}
  \begin{center}
    Fortunately, you can rewrite it as \blue{\texttt{vars : ID (',' ID)*}}

    \vspace{0.80cm}
    \fig{width = 1.00\textwidth}{figs/VarsDeclStarAG-g4}
    \teal{\texttt{VarsDeclStarAG.g4}}
  \end{center}
\end{frame}
%%%%%%%%%%%%%%%%%%%%

%%%%%%%%%%%%%%%%%%%%
\begin{frame}{}
  \begin{center}
    ``源码面前, 了无秘密''

    \fig{width = 0.50\textwidth}{figs/hj}

    \teal{\texttt{VarsDeclStarParser.java}}
  \end{center}
\end{frame}
%%%%%%%%%%%%%%%%%%%%

%%%%%%%%%%%%%%%%%%%%
% \begin{frame}{}
%   \begin{center}
%     \fig{width = 0.60\textwidth}{figs/VarsDecl-g4}
%     \[
%       \teal{\intkw\; \texttt{d}, \texttt{f}, \texttt{g}}
%     \]
%   \end{center}
% \end{frame}
%%%%%%%%%%%%%%%%%%%%

%%%%%%%%%%%%%%%%%%%%
% \begin{frame}{}
%   \begin{center}
%     \texttt{vars} 借助\red{\bf 继承属性}将声明的类型沿着标识符列表向下传递

%     \begin{columns}
%       \column{0.40\textwidth}
%         \fig{width = 1.00\textwidth}{figs/VarsDecl-g4}
%       \column{0.60\textwidth}
%         \fig{width = 1.00\textwidth}{figs/VarsDecl-tree}
%         \[
%           \teal{\intkw\; \texttt{d}, \texttt{f}, \texttt{g}}
%         \]
%     \end{columns}
%   \end{center}
% \end{frame}
%%%%%%%%%%%%%%%%%%%%