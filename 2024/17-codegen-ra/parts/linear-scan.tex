% linear-scan.tex

%%%%%%%%%%%%%%%%%%%%
\begin{frame}{}
	\begin{columns}
		\column{0.50\textwidth}
		  \fig{width = 0.70\textwidth}{figs/sum}
		\column{0.50\textwidth}
		  \fig{width = 0.70\textwidth}{figs/ir}
	\end{columns}

	\pause
	\begin{center}
		以\red{非 SSA 形式}的\blue{中间代码}为例
	\end{center}
\end{frame}
%%%%%%%%%%%%%%%%%%%%

%%%%%%%%%%%%%%%%%%%%
\begin{frame}{}
	\begin{center}
		\uncover<2->{\blue{问题 1: 变量 $n, s, i$} 的\red{活跃区间} (live interval) 分别是什么?}

		\begin{columns}
			\column{0.50\textwidth}
				\fig{width = 0.70\textwidth}{figs/ir}
			\column{0.50\textwidth}
			  \uncover<3->{\fig{width = 0.50\textwidth}{figs/cfg}}
		\end{columns}

		\uncover<4->{\blue{问题 2: 在第 3 行后, 有哪些变量是}\red{活跃}的?}
	\end{center}
\end{frame}
%%%%%%%%%%%%%%%%%%%%

%%%%%%%%%%%%%%%%%%%%
\begin{frame}{}
	\begin{definition}[活跃 (Live)]
		对于给定的变量 $x$, 考虑从其一个定义点 $p$ 到使用点 $q$ 的路径 $l$。\\[5pt]
		对于该路径 $l$ 上的任意点 $r$,
		如果 \purple{$r$ 和 $q$ 之间没有对变量 $x$ 的其它定义},\\[5pt]
		则称 $x$ 在程序点 $r$ 上是\blue{\it 活跃}的。
	\end{definition}

	\pause
	\vspace{0.80cm}
	\begin{center}
		\fbox{\red{在同一个程序点上活跃的变量是有冲突的, 不能分配到同一个寄存器。}}
	\end{center}

	\pause
	\vspace{0.50cm}
	\begin{definition}[活跃分析 (Liveness Analysis)]
		分析变量的活跃点的程序分析被称为 \blue{\it 活跃分析}。
	\end{definition}
\end{frame}
%%%%%%%%%%%%%%%%%%%%

%%%%%%%%%%%%%%%%%%%%
\begin{frame}{}
	\[
		\livein(s): s\; \text{\blue{执行前}的活跃变量集合}
	\]
	\vspace{-0.60cm}
	\begin{columns}
		\column{0.40\textwidth}
		  \fig{width = 0.60\textwidth}{figs/cfg}
		\column{0.60\textwidth}
		  \uncover<2->{
				\begin{gather*}
					\liveout(s) = \bigcup_{p \in \mathit{succ}(s)} \livein(p) \\[15pt]
					\livein(s) = (\liveout(s) \setminus \textsf{\violet{def}}(s)) \cup \textsf{\teal{use}}(s)
				\end{gather*}
			}
			\uncover<3->{
				\begin{center}
					\red{如何求解这个``数据流''方程组?}
				\end{center}
			}
	\end{columns}
	\[
		\liveout(s): s\; \text{\blue{执行后}的活跃变量集合}
	\]
\end{frame}
%%%%%%%%%%%%%%%%%%%%

%%%%%%%%%%%%%%%%%%%%
\begin{frame}{}
	\begin{columns}
		\column{0.40\textwidth}
		  \fig{width = 0.60\textwidth}{figs/cfg}
		\column{0.60\textwidth}
		  \fig{width = 1.00\textwidth}{figs/live-var-alg}
			\begin{center}
				\blue{不动点算法}
			\end{center}
	\end{columns}
\end{frame}
%%%%%%%%%%%%%%%%%%%%

%%%%%%%%%%%%%%%%%%%%
\begin{frame}{}
	\begin{columns}
		\column{0.40\textwidth}
		  \fig{width = 0.70\textwidth}{figs/cfg}
		\column{0.60\textwidth}
		  \fig{width = 1.00\textwidth}{figs/live-var}
	\end{columns}
\end{frame}
%%%%%%%%%%%%%%%%%%%%

%%%%%%%%%%%%%%%%%%%%
\begin{frame}{}
	\begin{columns}
		\column{0.40\textwidth}
		  \fig{width = 0.70\textwidth}{figs/cfg}
		\column{0.60\textwidth}
		  \fig{width = 0.40\textwidth}{figs/live-interval}
	\end{columns}
\end{frame}
%%%%%%%%%%%%%%%%%%%%

%%%%%%%%%%%%%%%%%%%%
\begin{frame}{}
	\begin{columns}
		\column{0.40\textwidth}
		  \fig{width = 0.60\textwidth}{figs/live-interval}
		\column{0.60\textwidth}
		  \centerline{$s.\text{index}:$ 语句 $s$ 的\blue{行号}}
		  \fig{width = 1.00\textwidth}{figs/live-interval-alg}
			\pause
			\[
				n: [0, 10] \quad s: [2, 10] \quad i: [3, 11]
			\]
	\end{columns}
\end{frame}
%%%%%%%%%%%%%%%%%%%%

%%%%%%%%%%%%%%%%%%%%
\begin{frame}{}
	\begin{columns}
		\column{0.40\textwidth}
		  \fig{width = 0.60\textwidth}{figs/live-interval}
		\column{0.60\textwidth}
			\[
				n: [0, 10] \quad s: [2, 10] \quad i: [3, 11]
			\]

			\vspace{0.60cm}
			\begin{center}
				\blue{\small 线性扫描分配算法 @ TOPLAS1999}
				\fig{width = 0.70\textwidth}{figs/linear-scan-paper}

				\vspace{0.30cm}
				三大关键操作: \violet{\bf 占用、释放、溢出}
			\end{center}
	\end{columns}
\end{frame}
%%%%%%%%%%%%%%%%%%%%

%%%%%%%%%%%%%%%%%%%%
\begin{frame}{}
	\begin{columns}
		\column{0.50\textwidth}
			\begin{align*}
				x_{1}&: [2, 16] \\[3pt]
				x_{2}&: [2, 20] \\[3pt]
				x_{3}&: [7, 8] \\[3pt]
				x_{4}&: [9, 10] \\[3pt]
				x_{5}&: [11, 12] \\[3pt]
				x_{6}&: [15, 19] \\[3pt]
				x_{7}&: [17, 19]
			\end{align*}
		\column{0.50\textwidth}
			\pause
			\[
				\green{|R| = 3 \quad (R_{1}, R_{2}, R_{3})}
			\]

			\vspace{6.00cm}
			\[
				\red{|R| = 2 \quad (R_{1}, R_{2})}
			\]
	\end{columns}
\end{frame}
%%%%%%%%%%%%%%%%%%%%

%%%%%%%%%%%%%%%%%%%%
\begin{frame}{}
	如何溢出 (两种方法, 避免迭代)
\end{frame}
%%%%%%%%%%%%%%%%%%%%

%%%%%%%%%%%%%%%%%%%%
\begin{frame}{}
	伪线性序问题
\end{frame}
%%%%%%%%%%%%%%%%%%%%

%%%%%%%%%%%%%%%%%%%%
\begin{frame}{}
	$\phi$ 消除问题
\end{frame}
%%%%%%%%%%%%%%%%%%%%