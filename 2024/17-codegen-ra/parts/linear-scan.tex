% linear-scan.tex

%%%%%%%%%%%%%%%%%%%%
\begin{frame}{}
	\begin{columns}
		\column{0.50\textwidth}
		  \fig{width = 0.70\textwidth}{figs/sum}
		\column{0.50\textwidth}
		  \pause
		  \fig{width = 0.70\textwidth}{figs/ir}
	\end{columns}

	\pause
	\begin{center}
		以\red{非 SSA 形式}的\blue{中间代码}为例
	\end{center}
\end{frame}
%%%%%%%%%%%%%%%%%%%%

%%%%%%%%%%%%%%%%%%%%
\begin{frame}{}
	\begin{center}
		\uncover<2->{\blue{问题 1: 变量 $n, s, i$} 的\red{活跃区间} (live interval) 分别是什么?}

		\begin{columns}
			\column{0.50\textwidth}
				\fig{width = 0.70\textwidth}{figs/ir}
			\column{0.50\textwidth}
			  \uncover<3->{\fig{width = 0.50\textwidth}{figs/cfg}}
		\end{columns}

		\uncover<4->{\blue{问题 2: 在第 3 行后, 有哪些变量是}\red{活跃}的?}
	\end{center}
\end{frame}
%%%%%%%%%%%%%%%%%%%%

%%%%%%%%%%%%%%%%%%%%
\begin{frame}{}
	\begin{definition}[活跃 (Live)]
		对于给定的变量 $x$, 考虑从其一个定义点 $p$ 到使用点 $q$ 的路径 $l$。\\[5pt]
		对于该路径 $l$ 上的任意点 $r$,
		如果 \purple{$r$ 和 $q$ 之间没有对变量 $x$ 的其它定义},\\[5pt]
		则称 $x$ 在程序点 $r$ 上是\blue{\it 活跃}的。
	\end{definition}

	\pause
	\vspace{0.80cm}
	\begin{center}
		\fbox{\red{在同一个程序点上活跃的变量是有冲突的, 不能分配到同一个寄存器。}}
	\end{center}

	\pause
	\vspace{0.50cm}
	\begin{definition}[活跃分析 (Liveness Analysis)]
		分析变量的活跃点的程序分析被称为 \blue{\it 活跃分析}。
	\end{definition}
\end{frame}
%%%%%%%%%%%%%%%%%%%%

%%%%%%%%%%%%%%%%%%%%
\begin{frame}{}
	\[
		\livein(s): s\; \text{\blue{执行前}的活跃变量集合}
	\]
	\vspace{-0.60cm}
	\begin{columns}
		\column{0.40\textwidth}
		  \fig{width = 0.60\textwidth}{figs/cfg}
		\column{0.60\textwidth}
		  \uncover<2->{
				\begin{gather*}
					\liveout(s) = \bigcup_{p \in \mathit{succ}(s)} \livein(p) \\[15pt]
					\livein(s) = (\liveout(s) \setminus \textsf{\violet{def}}(s)) \cup \textsf{\teal{use}}(s)
				\end{gather*}
			}
			\uncover<3->{
				\begin{center}
					\red{如何求解这个``数据流''方程组?}
				\end{center}
			}
	\end{columns}
	\[
		\liveout(s): s\; \text{\blue{执行后}的活跃变量集合}
	\]
\end{frame}
%%%%%%%%%%%%%%%%%%%%

%%%%%%%%%%%%%%%%%%%%
\begin{frame}{}
	\begin{columns}
		\column{0.40\textwidth}
		  \fig{width = 0.60\textwidth}{figs/cfg}
		\column{0.60\textwidth}
		  \fig{width = 1.00\textwidth}{figs/live-var-alg}
			\begin{center}
				\blue{不动点算法}
			\end{center}
	\end{columns}
\end{frame}
%%%%%%%%%%%%%%%%%%%%

%%%%%%%%%%%%%%%%%%%%
\begin{frame}{}
	\begin{columns}
		\column{0.40\textwidth}
		  \fig{width = 0.70\textwidth}{figs/cfg}
		\column{0.60\textwidth}
		  \fig{width = 1.00\textwidth}{figs/live-var}
	\end{columns}
\end{frame}
%%%%%%%%%%%%%%%%%%%%

%%%%%%%%%%%%%%%%%%%%
\begin{frame}{}
	\begin{columns}
		\column{0.40\textwidth}
		  \fig{width = 0.70\textwidth}{figs/cfg}
		\column{0.60\textwidth}
		  \fig{width = 0.40\textwidth}{figs/live-interval}
	\end{columns}
\end{frame}
%%%%%%%%%%%%%%%%%%%%

%%%%%%%%%%%%%%%%%%%%
\begin{frame}{}
	\begin{columns}
		\column{0.30\textwidth}
		  \fig{width = 0.80\textwidth}{figs/live-interval}
		\column{0.70\textwidth}
		  \centerline{$s.\text{index}:$ 语句 $s$ 的\blue{行号}}
			\vspace{-0.20cm}
		  \fig{width = 1.00\textwidth}{figs/live-interval-alg}
			\pause
			\vspace{-0.20cm}
			\centerline{\blue{不计最后一次``使用'' (use)}}
			\vspace{0.20cm}
			\pause
			\[
				n: [0, 10] \quad s: [2, 10] \quad i: [3, 11]
			\]
	\end{columns}
\end{frame}
%%%%%%%%%%%%%%%%%%%%

%%%%%%%%%%%%%%%%%%%%
\begin{frame}{}
	\begin{columns}
		\column{0.40\textwidth}
		  \fig{width = 0.60\textwidth}{figs/live-interval}
		\column{0.60\textwidth}
			\[
				n: [0, 10] \quad s: [2, 10] \quad i: [3, 11]
			\]

			\vspace{0.60cm}
			\begin{center}
				\blue{\small 线性扫描分配算法 @ TOPLAS1999}
				\fig{width = 0.70\textwidth}{figs/linear-scan-paper}

				\vspace{0.30cm}
				三大关键操作: \violet{\bf 占用、释放、溢出}
			\end{center}
	\end{columns}
\end{frame}
%%%%%%%%%%%%%%%%%%%%

%%%%%%%%%%%%%%%%%%%%
\begin{frame}{}
	\begin{columns}
		\column{0.50\textwidth}
			\begin{align*}
				x_{1}&: [2, 16] \\[3pt]
				x_{2}&: [2, 20] \\[3pt]
				x_{3}&: [7, 8] \\[3pt]
				x_{4}&: [9, 10] \\[3pt]
				x_{5}&: [11, 12] \\[3pt]
				x_{6}&: [15, 19] \\[3pt]
				x_{7}&: [17, 19]
			\end{align*}
		\column{0.50\textwidth}
			\pause
			\[
				\green{|R| = 3 \quad (R_{1}, R_{2}, R_{3})}
			\]

			\vspace{6.00cm}
			\[
				\red{|R| = 2 \quad (R_{1}, R_{2})}
			\]
	\end{columns}
\end{frame}
%%%%%%%%%%%%%%%%%%%%

%%%%%%%%%%%%%%%%%%%%
\begin{frame}{}
	\uncover<2->{
		\centerline{\red{缺点:} 引入了新的临时变量, 需要进行迭代 }
	}
	\uncover<3->{
    \centerline{\footnotesize (稍感欣慰的是, 新临时变量的活跃区间都很短)}
	}
	\begin{columns}
		\column{0.50\textwidth}
			\fig{width = 0.45\textwidth}{figs/cfg}
		\column{0.50\textwidth}
			\fig{width = 0.45\textwidth}{figs/spill-i}
	\end{columns}

	\vspace{0.20cm}
	\centerline{溢出变量 $i$: \blue{使用 \texttt{store/load} 存/取内存}}
\end{frame}
%%%%%%%%%%%%%%%%%%%%

%%%%%%%%%%%%%%%%%%%%
\begin{frame}{}
	\begin{center}
		\blue{解决方案一:} 生成代码时, 使用\violet{临时物理寄存器}实现临时变量
	\end{center}

	\begin{columns}
		\column{0.50\textwidth}
			\fig{width = 0.45\textwidth}{figs/spill-i}
		\column{0.50\textwidth}
		  \begin{center}
				\red{但是, 物理寄存器本来就不够用}

				\vspace{0.20cm}
				\pause
				\blue{使用某寄存器前将其保存到内存中}
				\vspace{0.30cm}

				\pause
				\texttt{addi sp, sp, -4}

				\vspace{0.10cm}
				\texttt{store x2, sp, 0}

				\vspace{0.10cm}
				\cyan{使用 \texttt{x2} 作为临时寄存器}

				\vspace{0.10cm}
				\texttt{load x2, sp, 0}

				\vspace{0.10cm}
				\texttt{addi sp, sp, 4}
			\end{center}
	\end{columns}
	\pause
	\centerline{\red{缺点:} 可能带来大量内存操作}
\end{frame}
%%%%%%%%%%%%%%%%%%%%

%%%%%%%%%%%%%%%%%%%%
\begin{frame}{}
	\begin{center}
		\blue{解决方案二:} \purple{预留}若干物理寄存器, 作为溢出处理时所需的\violet{临时寄存器}
	\end{center}

	\begin{columns}
		\column{0.50\textwidth}
		  \begin{gather*}
				\cdots = x \\[8pt]
				\cdots \\[8pt]
				x = \cdots
			\end{gather*}
		\column{0.50\textwidth}
		  \begin{gather*}
				r_{i} = \cdots \\[8pt]
				[l_x] = r_{i} \\[12pt]
				\cdots \\[12pt]
				r_{i} = [l_x] \\[8pt]
				\cdots = r_{i}
			\end{gather*}
	\end{columns}

	\pause
	\begin{center}
		\red{预留多少 ($K$) 个物理寄存器?}

		\pause
		\vspace{0.30cm}
		$K$ 为程序中所有语句\textsf{def}集合或\textsf{use}集合的最大元素个数

		\pause
		\vspace{0.30cm}
		\teal{\small (对于 RISC-V 典型程序, $K = 2$)}
	\end{center}
\end{frame}
%%%%%%%%%%%%%%%%%%%%

%%%%%%%%%%%%%%%%%%%%
\begin{frame}{}
	\begin{center}
		\blue{解决方案二:} \purple{预留}若干物理寄存器, 作为溢出处理时所需的\violet{临时寄存器}
	\end{center}

	\begin{columns}
		\column{0.50\textwidth}
		  \begin{gather*}
				y = \tau(x_{1}, x_{2}, \cdots, x_{K})
			\end{gather*}
			假设 $x_{1}, \cdots, x_{K}, \red{y}$ 均发生溢出
		\column{0.50\textwidth}
		  \begin{gather*}
				r_{1} = [l_{x_{1}}] \\[8pt]
				\cdots \\[8pt]
				r_{K} = [l_{x_{K}}] \\[8pt]
				\red{r_{1}} = \tau(r_{1}, \cdots, r_{K}) \\[8pt]
				[l_{y}] = \red{r_{1}}
			\end{gather*}
	\end{columns}
\end{frame}
%%%%%%%%%%%%%%%%%%%%

%%%%%%%%%%%%%%%%%%%%
\begin{frame}{}
	\begin{center}
		还有\red{两个重要问题}要解决

		\vspace{0.20cm}
		\fig{width = 0.50\textwidth}{figs/questions}
		\vspace{0.20cm}

		\pause
		\blue{代码编号问题 (index) \qquad $\phi$-指令消除问题}
	\end{center}
\end{frame}
%%%%%%%%%%%%%%%%%%%%

%%%%%%%%%%%%%%%%%%%%
\begin{frame}{}
	\begin{center}
		\uncover<5->{\red{影响``活跃变量''分析结果吗?}}
		\uncover<6->{\blue{\quad (不影响)}}
	\end{center}

	\vspace{-0.30cm}
	\begin{columns}
		\column{0.33\textwidth}
		  \uncover<3->{\fig{width = 0.75\textwidth}{figs/cfg}}
		\column{0.33\textwidth}
		  \only<1>{\fig{width = 0.75\textwidth}{figs/ir-no-lineno}}
		  \only<2->{\fig{width = 0.75\textwidth}{figs/cfg-no-lineno}}
		\column{0.33\textwidth}
		  \uncover<4->{\fig{width = 0.75\textwidth}{figs/cfg-2}}
	\end{columns}

	\begin{center}
		\uncover<7->{\purple{影响``活跃区间''分析结果吗?}}
		\uncover<8->{\teal{\quad (影响)}}

		\vspace{0.10cm}
		\uncover<9->{\violet{影响``线性扫描算法''正确性吗?}}
		\uncover<10->{\cyan{\quad (不影响)}}
	\end{center}
\end{frame}
%%%%%%%%%%%%%%%%%%%%

%%%%%%%%%%%%%%%%%%%%
\begin{frame}{}
	\begin{center}
		常用编号策略: \purple{深度优先搜索顺序}

		\fig{width = 0.40\textwidth}{figs/dfs}

		\blue{逆后序遍历序} (reverse post-order traversal ordering)
	\end{center}
\end{frame}
%%%%%%%%%%%%%%%%%%%%

%%%%%%%%%%%%%%%%%%%%
\begin{frame}{}
	\fig{width = 0.50\textwidth}{figs/phi}
\end{frame}
%%%%%%%%%%%%%%%%%%%%

%%%%%%%%%%%%%%%%%%%%
\begin{frame}{}
	\begin{center}
		基本思想: \blue{将``拷贝''操作上推到前驱基本块的末尾}

		\vspace{0.30cm}
		\fig{width = 0.80\textwidth}{figs/phi-elim-idea}

		\pause
		\vspace{0.30cm}
		注意\red{``循环''}情况
	\end{center}
\end{frame}
%%%%%%%%%%%%%%%%%%%%

%%%%%%%%%%%%%%%%%%%%
\begin{frame}{}
	\begin{center}
		问题一: \blue{备份丢失} (lost-copy) 问题
	\end{center}

	\begin{columns}
		\column{0.50\textwidth}
			\fig{width = 1.00\textwidth}{figs/phi-elim-lost-copy}
			\begin{center}
				\red{``关键路径''}导致的\blue{备份丢失}问题
			\end{center}
		\column{0.50\textwidth}
			\fig{width = 0.80\textwidth}{figs/phi-elim-lost-copy-solution}
			\begin{center}
				\purple{引入临时变量消除备份丢失}
			\end{center}
	\end{columns}
\end{frame}
%%%%%%%%%%%%%%%%%%%%

%%%%%%%%%%%%%%%%%%%%
\begin{frame}{}
	\begin{center}
		问题二: \blue{交换} (swap) 问题
	\end{center}

	\begin{columns}
		\column{0.20\textwidth}
		  \fig{width = 0.80\textwidth}{figs/swap-code}
		\column{0.20\textwidth}
			\pause
		  \fig{width = 1.00\textwidth}{figs/swap-phi}
			\begin{center}
				\red{$\phi$指令的并行语义}
			\end{center}
		\column{0.30\textwidth}
		  \pause
		  \fig{width = 0.50\textwidth}{figs/swap-phi-wrong}
			\begin{center}
				\blue{交换}问题
			\end{center}
		\column{0.33\textwidth}
		  \pause
		  \begin{gather*}
				\teal{t_{x}} \gets x_{0} \\[3pt]
				\cyan{t_{y}} \gets y_{0}
			\end{gather*}
			\begin{center}
				\purple{\small 引入临时变量}
			\end{center}
		  \begin{gather*}
				x_{1} \gets \teal{t_{x}} \\[3pt]
				y_{1} \gets \cyan{t_{y}} \\[3pt]
				\cdots \\[3pt]
				\teal{t_{x}} \gets y_{1} \\[3pt]
				\cyan{t_{y}} \gets x_{1} \\[3pt]
			\end{gather*}
	\end{columns}
\end{frame}
%%%%%%%%%%%%%%%%%%%%

% %%%%%%%%%%%%%%%%%%%%
% \begin{frame}{}
% 	\begin{center}
% 		问题二: \blue{交换} (swap) 问题
% 	\end{center}

% 	\begin{columns}
% 		\column{0.50\textwidth}
% 		  \begin{gather*}
% 				\teal{x} = \phi(\cyan{y}, u) \\[3pt]
% 				\cyan{y} = \phi(\teal{x}, v)
% 			\end{gather*}
% 			\begin{center}
% 				\red{$\phi$指令的并行语义}导致的\blue{交换}问题
% 			\end{center}
% 			\begin{gather*}
% 				\teal{x} = \cyan{y}	\qquad \teal{x} = u \\[3pt]
% 				\cyan{y} = \teal{x} \qquad \cyan{y} = v
% 			\end{gather*}
% 		\column{0.50\textwidth}
% 		  \begin{gather*}
% 				t_x = \cyan{y}	\qquad t_x = u \\[3pt]
% 				t_y = \teal{x} \qquad t_y = v
% 			\end{gather*}
% 			\begin{center}
% 				\purple{引入临时变量消除交换问题}
% 			\end{center}
% 		  \begin{gather*}
% 				\teal{x} = t_x	\\[3pt]
% 				\cyan{y} = t_y
% 			\end{gather*}
% 	\end{columns}
% \end{frame}
% %%%%%%%%%%%%%%%%%%%%