% overview.tex

%%%%%%%%%%%%%%%%%%%%
\begin{frame}{}
  \begin{center}
    \teal{\texttt{Control.g4}}
  \end{center}
  \begin{columns}
    \column{0.50\textwidth}
      \fig{width = 0.45\textwidth}{figs/cfg-S-grammar}
    \column{0.50\textwidth}
      \fig{width = 0.35\textwidth}{figs/cfg-boolexpr-grammar}
  \end{columns}
\end{frame}
%%%%%%%%%%%%%%%%%%%%

%%%%%%%%%%%%%%%%%%%%
\begin{frame}{}
  \begin{columns}
    \column{0.50\textwidth}
      \fig{width = 0.80\textwidth}{figs/dragon-book}
    \column{0.50\textwidth}
      \fig{width = 0.85\textwidth}{figs/Engineering-A-Compiler-book}
  \end{columns}
  \begin{center}
    虽然有\blue{\bf 多种}讲法, 教材偏偏采用了让初学者\red{\bf 望而生畏}的那一种。
  \end{center}
\end{frame}
%%%%%%%%%%%%%%%%%%%%

%%%%%%%%%%%%%%%%%%%%
\begin{frame}{}
  \begin{center}
    {\Large 分工 \qquad 合作}

    \vspace{0.20cm}
    \fig{width = 0.60\textwidth}{figs/team-work-fly}
  \end{center}

  \begin{center}
    为布尔表达式 $B$ 计算逻辑值 (假设保存在临时变量 \texttt{t1} 中) \\[5pt]
    \texttt{\bf if}、\texttt{\bf while} 等语句
    根据 $B$ 的结果改变控制流
  \end{center}

  \vspace{-0.30cm}
  \[
    \texttt{\bf if}\; (B)\; S_{1}
  \]
\end{frame}
%%%%%%%%%%%%%%%%%%%%

%%%%%%%%%%%%%%%%%%%%
\begin{frame}{}
  \begin{center}
    \red{\bf 如何生成更短、\textcolor{lightgray}{更高效}的代码?}
  \end{center}

  \begin{columns}
    \column{0.50\textwidth}
      \fig{width = 0.60\textwidth}{figs/while-if-II}
    \column{0.50\textwidth}
      \fig{width = 0.85\textwidth}{figs/while-if-II-code}
  \end{columns}
\end{frame}
%%%%%%%%%%%%%%%%%%%%

%%%%%%%%%%%%%%%%%%%%
\begin{frame}{}
  \begin{center}
    \red{\bf 如何生成更短、\textcolor{lightgray}{更高效}的代码?}
  \end{center}

  \begin{columns}
    \column{0.50\textwidth}
      \fig{width = 0.80\textwidth}{figs/bool-short-circuit-II}
    \column{0.50\textwidth}
      \fig{width = 0.85\textwidth}{figs/bool-short-circuit-II-code}
  \end{columns}
\end{frame}
%%%%%%%%%%%%%%%%%%%%