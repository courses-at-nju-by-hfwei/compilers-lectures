% llvm-ir.tex

%%%%%%%%%%%%%%%%%%%%
\begin{frame}{}
	\begin{center}
		{\href{https://llvm.org/docs/LangRef.html}{\bf LLVM Language Reference Manual}}

		\fig{width = 0.50\textwidth}{figs/llvm-ir}

		\vspace{0.50cm}
		IR: Intermediate Representation

		\vspace{0.50cm}
		LLVM IR: 带类型的、介于高级程序设计语言与汇编语言之间 \\ [3pt]
		(LLVM Assembly Language)
	\end{center}
\end{frame}
%%%%%%%%%%%%%%%%%%%%

%%%%%%%%%%%%%%%%%%%%
\begin{frame}{}
	\fig{width = 0.60\textwidth}{figs/talk-cheap}
\end{frame}
%%%%%%%%%%%%%%%%%%%%

%%%%%%%%%%%%%%%%%%%%
\begin{frame}{}
	\begin{center}
		\fig{width = 0.60\textwidth}{figs/factorial0}
		\vspace{0.30cm}
		\texttt{factorial0.c}
	\end{center}
\end{frame}
%%%%%%%%%%%%%%%%%%%%

%%%%%%%%%%%%%%%%%%%%
\begin{frame}{}
	\begin{center}
		\uncover<2->{\blue {Three Address Code (TAC)}}
		\uncover<3->{\qquad \red{Static Single Assignment (SSA)}}

		\vspace{0.30cm}
		\fig{width = 0.95\textwidth}{figs/f0-opt0}

		\teal{\texttt{clang}} \red{\texttt{-S -emit-llvm}} \teal{\texttt{factorial0.c -o f0-opt0.ll}}
	\end{center}
\end{frame}
%%%%%%%%%%%%%%%%%%%%

%%%%%%%%%%%%%%%%%%%%
\begin{frame}{}
	\begin{center}
		\blue{mem2reg}
		\vspace{0.30cm}

		\fig{width = 0.95\textwidth}{figs/f0-opt1}

		\vspace{0.30cm}
		\teal{\texttt{clang -S -emit-llvm factorial0.c -o f0-opt1.ll \red{-O1} -g0}}
	\end{center}
\end{frame}
%%%%%%%%%%%%%%%%%%%%

%%%%%%%%%%%%%%%%%%%%
\begin{frame}{}
	\begin{columns}
		\column{0.60\textwidth}
			\fig{width = 0.80\textwidth}{figs/f1}
		\column{0.40\textwidth}
			\fig{width = 0.80\textwidth}{figs/f1-callgraph}
	\end{columns}
	\begin{center}
		\teal{\texttt{factorial1.c}}
	\end{center}
\end{frame}
%%%%%%%%%%%%%%%%%%%%

%%%%%%%%%%%%%%%%%%%%
\begin{frame}{}
	\begin{columns}
		\column{0.50\textwidth}
		  \fig{width = 0.80\textwidth}{figs/Allen}
		  \begin{center}
			  \href{https://en.wikipedia.org/wiki/Frances_Allen}{Frances Elizabeth Allen} ($1932 \sim 2020$; \blue{2006 Turing Award})
		  \end{center}
		\column{0.50\textwidth}
		  \fig{width = 0.60\textwidth}{figs/cfg-wiki}
	\end{columns}

	\vspace{0.30cm}
	\begin{center}
		\red{(Intra-procedure)} \href{https://en.wikipedia.org/wiki/Control-flow_graph}{Control Flow Graph (CFG)}
	\end{center}
\end{frame}
%%%%%%%%%%%%%%%%%%%%

%%%%%%%%%%%%%%%%%%%%
\begin{frame}{}
	\begin{center}
		\href{https://en.wikipedia.org/wiki/Control-flow_graph}{Control Flow Graph (CFG)}

		\vspace{0.50cm}
		\begin{definition}[CFG]
			Each \teal{node} represents a \emph{\blue{basic block}},
			i.e. a straight-line code sequence
			with no \cyan{\bf branches/jumps} in except to the \purple{entry point} \\
			and no \cyan{\bf branches/jumps} out except at the \purple{exit point}.

			\vspace{0.80cm}
			\pause
			Jump targets start a block, and jumps end a block.

			\vspace{0.80cm}
			\pause
			Directed \teal{edges} are used to represent jumps in the control flow.
		\end{definition}
	\end{center}
\end{frame}
%%%%%%%%%%%%%%%%%%%%

%%%%%%%%%%%%%%%%%%%%
% \begin{frame}{}
% 	\begin{columns}
% 		\column{0.50\textwidth}
% 			\fig{width = 0.80\textwidth}{figs/f1-opt0-main-cfg-only}
% 		\column{0.50\textwidth}
% 			\fig{width = 0.90\textwidth}{figs/f1-opt0-main-cfg}
% 	\end{columns}
% 	\begin{center}
% 		\texttt{factorial1.c (opt0)}
% 	\end{center}
% \end{frame}
%%%%%%%%%%%%%%%%%%%%

%%%%%%%%%%%%%%%%%%%%
\begin{frame}{}
	\fig{width = 0.60\textwidth}{figs/factorial1}
\end{frame}
%%%%%%%%%%%%%%%%%%%%

%%%%%%%%%%%%%%%%%%%%
\begin{frame}{}
	\begin{center}
		\uncover<2->{\blue{\texttt{\%2}: store the return value (in different branches)}}
		\fig{width = 0.70\textwidth}{figs/f1-opt0-factorial-cfg}
	\end{center}
\end{frame}
%%%%%%%%%%%%%%%%%%%%

%%%%%%%%%%%%%%%%%%%%
\begin{frame}{}
	\fig{width = 0.50\textwidth}{figs/terminator}
\end{frame}
%%%%%%%%%%%%%%%%%%%%

%%%%%%%%%%%%%%%%%%%%
% \begin{frame}{}
% 	\begin{columns}
% 		\column{0.50\textwidth}
% 			\fig{width = 0.50\textwidth}{figs/f1-opt1-main-cfg-only}
% 		\column{0.50\textwidth}
% 			\fig{width = 0.90\textwidth}{figs/f1-opt1-main-cfg}
% 	\end{columns}
% 	\begin{center}
% 		\texttt{factorial1.c (opt1)}
% 	\end{center}
% \end{frame}
%%%%%%%%%%%%%%%%%%%%

%%%%%%%%%%%%%%%%%%%%
\begin{frame}{}
	\fig{width = 0.70\textwidth}{figs/f1-opt1-factorial-cfg}
\end{frame}
%%%%%%%%%%%%%%%%%%%%

%%%%%%%%%%%%%%%%%%%%
\begin{frame}{}
	\begin{center}
		$\phi$ 根据控制流决定选择 $y_{1}$ 还是 $y_{2}$
	\end{center}
	\begin{columns}
		\column{0.33\textwidth}
			\fig{width = 0.85\textwidth}{figs/ssa-wiki-example-1}
		\column{0.33\textwidth}
			\fig{width = 0.85\textwidth}{figs/ssa-wiki-example-2}
		\column{0.33\textwidth}
			\fig{width = 0.85\textwidth}{figs/ssa-wiki-example-3}
	\end{columns}

	\pause
	\vspace{0.30cm}
	\begin{center}
		\href{https://mapping-high-level-constructs-to-llvm-ir.readthedocs.io/en/latest/control-structures/ssa-phi.html}{How to implement $\phi$ instruction?}

		\pause
		\vspace{0.30cm}
		基本思想: 将 $\phi$ 指令转换成若干赋值指令, 上推至前驱基本块中
	\end{center}
\end{frame}
%%%%%%%%%%%%%%%%%%%%

%%%%%%%%%%%%%%%%%%%%
\begin{frame}{}
	\begin{center}
		SSA 形式的构建与消去
	\end{center}

	\begin{columns}
		\column{0.33\textwidth}
		  \uncover<1->{\fig{width = 0.90\textwidth}{figs/BiSheng-book}
			\begin{center}
				Section 4.3
			\end{center}}
		\column{0.33\textwidth}
		  \uncover<3->{\fig{width = 1.00\textwidth}{figs/TOPLAS1991}
			\begin{center}
				\href{https://github.com/courses-at-nju-by-hfwei/compilers-papers-we-love/blob/master/ir/TOPLAS1991\%20Efficiently\%20Computing\%20Static\%20Single\%20Assignment\%20Form\%20and\%20the\%20Control\%20Dependence\%20Graph.pdf}{\footnotesize TOPLAS1991 @ compilers-papers-we-love}
			\end{center}}
		\column{0.33\textwidth}
		  \uncover<2->{\fig{width = 0.90\textwidth}{figs/Engineering-A-Compiler-book}
			\begin{center}
				Section 9.3
			\end{center}}
	\end{columns}
\end{frame}
%%%%%%%%%%%%%%%%%%%%

%%%%%%%%%%%%%%%%%%%%
\begin{frame}{}
	\begin{center}
		\fig{width = 0.20\textwidth}{figs/f1-opt3-factorial-cfg-only}

		\vspace{-0.30cm}
		\texttt{factorial1 (opt3)}
	\end{center}
\end{frame}
%%%%%%%%%%%%%%%%%%%%

%%%%%%%%%%%%%%%%%%%%
% \begin{frame}{}
% 	\begin{columns}
% 		\column{0.50\textwidth}
% 			\fig{width = 0.80\textwidth}{figs/f2-opt0-main-cfg-only}
% 		\column{0.50\textwidth}
% 			\fig{width = 0.90\textwidth}{figs/f2-opt0-main-cfg}
% 	\end{columns}
% 	\begin{center}
% 		\texttt{factorial2.c (opt0)}
% 	\end{center}
% \end{frame}
%%%%%%%%%%%%%%%%%%%%

%%%%%%%%%%%%%%%%%%%%
\begin{frame}{}
	\fig{width = 0.60\textwidth}{figs/factorial2}
	\begin{center}
		\texttt{factorial2.c}
	\end{center}
\end{frame}
%%%%%%%%%%%%%%%%%%%%

%%%%%%%%%%%%%%%%%%%%
\begin{frame}{}
	\vspace{-0.80cm}
	\fig{width = 0.70\textwidth}{figs/f2-opt0-factorial-cfg.pdf}

	% \begin{center}
	% 	\texttt{factorial2.c (opt0)}
	% \end{center}
\end{frame}
%%%%%%%%%%%%%%%%%%%%

%%%%%%%%%%%%%%%%%%%%
% \begin{frame}{}
% 	\fig{width = 0.75\textwidth}{figs/f2-opt0-factorial-cfg}
% \end{frame}
%%%%%%%%%%%%%%%%%%%%

%%%%%%%%%%%%%%%%%%%%
% \begin{frame}{}
% 	\begin{columns}
% 		\column{0.50\textwidth}
% 			\fig{width = 0.80\textwidth}{figs/f2-opt1-main-cfg-only}
% 		\column{0.50\textwidth}
% 			\fig{width = 0.80\textwidth}{figs/f2-opt1-main-cfg}
% 	\end{columns}
% 	\begin{center}
% 		\texttt{factorial2.c (opt1)}
% 	\end{center}
% \end{frame}
%%%%%%%%%%%%%%%%%%%%

%%%%%%%%%%%%%%%%%%%%
\begin{frame}{}
	\vspace{-0.50cm}
	\fig{width = 0.70\textwidth}{figs/f2-opt1-factorial-cfg.pdf}
\end{frame}
%%%%%%%%%%%%%%%%%%%%

%%%%%%%%%%%%%%%%%%%%
\begin{frame}{}
	\begin{center}
		\fig{width = 0.20\textwidth}{figs/f2-opt3-factorial-cfg-only}

		\vspace{-0.30cm}
		\texttt{factorial2 (opt3)}
	\end{center}
\end{frame}
%%%%%%%%%%%%%%%%%%%%