% overview.tex

%%%%%%%%%%%%%%%%%%%%
\begin{frame}{}
	\begin{center}
		使用\blue{标签}标记跳转目标
	\end{center}

  \begin{columns}
    \column{0.50\textwidth}
      \fig{width = 0.80\textwidth}{figs/bool-short-circuit-II}
    \column{0.50\textwidth}
      \fig{width = 0.40\textwidth}{figs/bool-short-circuit-II-hard-code}
  \end{columns}
\end{frame}
%%%%%%%%%%%%%%%%%%%%

%%%%%%%%%%%%%%%%%%%%
\begin{frame}{}
	\begin{center}
		Java Bytecode: 使用\red{地址值}作为跳转目标
	\end{center}
  \begin{columns}
    \column{0.50\textwidth}
      \fig{width = 0.80\textwidth}{figs/java-code}
    \column{0.50\textwidth}
      \pause
      \fig{width = 0.45\textwidth}{figs/java-bytecode}
  \end{columns}

	\begin{center}
		\red{$Q:$ 如何在一趟扫描中生成跳转目标的地址?}
	\end{center}
\end{frame}
%%%%%%%%%%%%%%%%%%%%

%%%%%%%%%%%%%%%%%%%%
\begin{frame}{}
  \begin{center}
    \fig{width = 1.00\textwidth}{figs/SDD-if}

    \vspace{0.80cm}
    \uncover<2->{\blue{$B$ 可以自行计算 $B.\mathit{true}$ 对应的指令地址}}

    \vspace{0.30cm}
    \fig{width = 0.40\textwidth}{figs/if-code-block}
    \vspace{0.30cm}

    \uncover<3->{\red{$B$ 计算不出 $B.\mathit{false}$ 对应的指令地址}}
  \end{center}
\end{frame}
%%%%%%%%%%%%%%%%%%%%

%%%%%%%%%%%%%%%%%%%%
% \begin{frame}{}
%   \begin{center}
%     \blue{\bf 回填技术: 在一趟扫描中生成跳转目标地址 (而非目标标签)}
%   \end{center}
% \end{frame}
%%%%%%%%%%%%%%%%%%%%

%%%%%%%%%%%%%%%%%%%%
\begin{frame}{}
  \begin{center}
    \red{回填 (Backpatching) 技术}: {子节点挖坑、祖先节点填坑}

    \fig{width = 0.50\textwidth}{figs/wakeng}

    \pause
    子节点暂时不指定跳转指令的目标地址 \\[5pt]
    待祖先节点能够确定目标地址时回头填充

    \pause
    \vspace{0.80cm}
    父节点通过\red{综合属性}收集子节点中具有相同目标的跳转指令
  \end{center}
\end{frame}
%%%%%%%%%%%%%%%%%%%%

%%%%%%%%%%%%%%%%%%%%
\begin{frame}{}
	\fig{width = 0.65\textwidth}{figs/train-college-film}
\end{frame}
%%%%%%%%%%%%%%%%%%%%

%%%%%%%%%%%%%%%%%%%%
\begin{frame}{}
	\fig{width = 0.65\textwidth}{figs/train-college-cartoon}
\end{frame}
%%%%%%%%%%%%%%%%%%%%