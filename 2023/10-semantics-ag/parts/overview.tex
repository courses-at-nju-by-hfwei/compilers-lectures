% overview.tex

%%%%%%%%%%%%%%%%%%%%
% \begin{frame}{}
%   \begin{center}
%     \blue{\Large 第 5 章: 语法制导的翻译} \\[10pt]
%     {\large (Syntax-Directed Translation)}

%     \fig{width = 0.50\textwidth}{figs/directed}

%     % \pause
%     % \vspace{0.30cm}
%     % {\Large 语法``\red{\bf 指导}''的翻译}

%     % \pause
%     \vspace{0.30cm}
%     {\large 依据语法结构进行翻译, 或在语法分析过程中进行翻译}
%   \end{center}
% \end{frame}
%%%%%%%%%%%%%%%%%%%%

%%%%%%%%%%%%%%%%%%%%
% \begin{frame}{}
%   \fig{width = 0.55\textwidth}{figs/dragon-book-2}
% \end{frame}
%%%%%%%%%%%%%%%%%%%%

%%%%%%%%%%%%%%%%%%%%
\begin{frame}{}
  \begin{center}
    Regular Expression (词法分析)

    \vspace{1.00cm}
    Context-Free Grammar (语法分析)

    \pause
    \vspace{1.00cm}
    {什么样的文法刻画语言的\red{\bf 语义}?}
  \end{center}
\end{frame}
%%%%%%%%%%%%%%%%%%%%

%%%%%%%%%%%%%%%%%%%%
\begin{frame}{}
  \fig{width = 0.60\textwidth}{figs/knuth}

  \begin{center}
    Donald Knuth ($1938 \sim$)
  \end{center}
\end{frame}
%%%%%%%%%%%%%%%%%%%%

%%%%%%%%%%%%%%%%%%%%
\begin{frame}{}
  \fig{width = 0.70\textwidth}{figs/knuth-paper-1967}

  \begin{center}
    \red{\bf 属性文法 (Attribute Grammar):} 为上下文无关文法赋予\blue{\bf 语义}
  \end{center}
\end{frame}
%%%%%%%%%%%%%%%%%%%%

%%%%%%%%%%%%%%%%%%%%
% \begin{frame}{}
%     \begin{center}
%       \red{\bf DFS (around $1972$)}

%       \fig{width = 0.35\textwidth}{figs/Tarjan}

%       Robert Tarjan ($1948 \sim$) \\[6pt]
%     \end{center}
% \end{frame}
%%%%%%%%%%%%%%%%%%%%

%%%%%%%%%%%%%%%%%%%%
\begin{frame}{}
  \fig{width = 0.60\textwidth}{figs/talk-cheap}
\end{frame}
%%%%%%%%%%%%%%%%%%%%

%%%%%%%%%%%%%%%%%%%%
\begin{frame}{}
  \fig{width = 0.45\textwidth}{figs/antlr4-book-ch}
  \begin{center}
    \blue{\bf 第 10 章: 属性和动作}
  \end{center}
\end{frame}
%%%%%%%%%%%%%%%%%%%%

%%%%%%%%%%%%%%%%%%%%
\begin{frame}{}
  \fig{width = 0.40\textwidth}{figs/calculator}
  \begin{center}
    \blue{\bf (交互式) 迷你计算器}
  \end{center}
\end{frame}
%%%%%%%%%%%%%%%%%%%%

%%%%%%%%%%%%%%%%%%%%
\begin{frame}{}
  \begin{columns}
    \column{0.50\textwidth}
      \fig{width = 0.20\textwidth}{figs/calc-input}
    \column{0.50\textwidth}
      \fig{width = 0.15\textwidth}{figs/calc-output}
  \end{columns}

  \fig{width = 1.00\textwidth}{figs/parsetree-calc}
\end{frame}
%%%%%%%%%%%%%%%%%%%%

%%%%%%%%%%%%%%%%%%%%
\begin{frame}{}
  \begin{center}
    \blue{\bf Offline 方式计算属性值:} 已有语法分析树 (\teal{\texttt{calc}})

    \vspace{0.50cm}
    \fig{width = 0.50\textwidth}{figs/dfs}

    \pause
    \vspace{0.50cm}
    按照\red{\bf 从左到右}的\red{\bf 深度优先}顺序遍历语法分析树

    \vspace{0.50cm}
    \red{\bf 关键:} 在合适的时机执行合适的动作,计算相应的属性值
  \end{center}
\end{frame}
%%%%%%%%%%%%%%%%%%%%

%%%%%%%%%%%%%%%%%%%%
\begin{frame}{}
  \begin{center}
    在\blue{\bf 语法分析过程中}实现\blue{\bf 属性文法}
  \end{center}

  \[
    B \to \blue{X} \red{\set{a}} Y
  \]

  \pause
  \vspace{0.80cm}
  \begin{center}
    语义动作嵌入的位置决定了\red{\bf 何时}执行该动作

    \vspace{0.60cm}
    \red{\bf 基本思想:} 一个动作在它\teal{\bf 左边的}所有文法符号都\teal{\bf 处理}过之后立刻执行
  \end{center}
\end{frame}
%%%%%%%%%%%%%%%%%%%%

%%%%%%%%%%%%%%%%%%%%
\begin{frame}{}
  \fig{width = 0.40\textwidth}{figs/calculator}
  \begin{center}
    \blue{\bf \red{(交互式)} 迷你计算器}

    \vspace{0.50cm}
    \teal{\texttt{ExprAGMain.java}}
  \end{center}
\end{frame}
%%%%%%%%%%%%%%%%%%%%

%%%%%%%%%%%%%%%%%%%%
\begin{frame}{}
  \fig{width = 0.50\textwidth}{figs/csv}
  \begin{center}
    \blue{\bf Comma-Separated Values}
  \end{center}
\end{frame}
%%%%%%%%%%%%%%%%%%%%

%%%%%%%%%%%%%%%%%%%%
\begin{frame}{}
  \begin{columns}
    \column{0.40\textwidth}
      \fig{width = 0.80\textwidth}{figs/csv-scores-input}
    \column{0.60\textwidth}
      \fig{width = 0.90\textwidth}{figs/csv-scores-output}
  \end{columns}

  \vspace{0.30cm}
  \fig{width = 1.00\textwidth}{figs/parsetree-csv-scores}

  \fig{width = 0.50\textwidth}{figs/csv-g4}
\end{frame}
%%%%%%%%%%%%%%%%%%%%

%%%%%%%%%%%%%%%%%%%%
% \begin{frame}{}
%   \begin{center}
%     \begin{columns}[b]
%       \column{0.25\textwidth}
%         \fig{width = 0.80\textwidth}{figs/1}
%         \vspace{-0.10cm}
%         \begin{center}
%           \blue{一对概念}
%         \end{center}
%       \column{0.25\textwidth}
%         \fig{width = 0.85\textwidth}{figs/2}
%         \vspace{-0.60cm}
%         \begin{center}
%           \blue{两类属性定义}
%         \end{center}
%       \column{0.25\textwidth}
%         \fig{width = 0.65\textwidth}{figs/3}
%         \begin{center}
%           \blue{三种实现方式}
%         \end{center}
%       \column{0.25\textwidth}
%         \fig{width = 0.65\textwidth}{figs/4}
%         \begin{center}
%           \blue{四大应用}
%         \end{center}
%     \end{columns}
%   \end{center}
% \end{frame}
%%%%%%%%%%%%%%%%%%%%

%%%%%%%%%%%%%%%%%%%%
% \begin{frame}{}
%   \begin{center}
%     \fig{width = 0.30\textwidth}{figs/4}
%     \vspace{0.30cm}
%     \teal{\bf 表达式求值}

%     \vspace{0.30cm}
%     \red{\bf 类型系统} (语义分析)

%     \vspace{0.30cm}
%     抽象语法树

%     \vspace{0.30cm}
%     \blue{\bf 后缀表达式} (中间代码生成)
%   \end{center}
% \end{frame}
%%%%%%%%%%%%%%%%%%%%

%%%%%%%%%%%%%%%%%%%%
\begin{frame}{}
  \begin{center}
    % \red{\bf 关键问题: 如何基于上下文无关文法做上下文相关分析?}

    % \vspace{0.50cm}
    \fig{width = 0.70\textwidth}{figs/info-flow}
    \vspace{0.30cm}

    \blue{\bf 语法分析树}上的\red{\bf 有序}信息流动
  \end{center}
\end{frame}
%%%%%%%%%%%%%%%%%%%%

%%%%%%%%%%%%%%%%%%%%
\begin{frame}{}
  \begin{center}
    \blue{\bf 时机} (Timing; タイミング)

    \fig{width = 0.60\textwidth}{figs/timing-japanese}

    \vspace{0.30cm}
    \red{\bf 语义动作嵌入在什么地方? 这决定了何时执行语义动作。}
  \end{center}
\end{frame}
%%%%%%%%%%%%%%%%%%%%