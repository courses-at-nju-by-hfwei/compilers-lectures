% sdd.tex

%%%%%%%%%%%%%%%%%%%%
\begin{frame}{}
  \begin{center}
    \begin{definition}[语法制导定义 (Syntax-Directed Definition; SDD)]
      SDD 是一个上下文无关文法和\red{\bf 属性}及\blue{\bf 规则}的结合。
    \end{definition}

    \vspace{0.50cm}
    \red{每个文法符号都可以关联多个属性}

    \fig{width = 0.60\textwidth}{figs/SDD-expr-left-recursion}

    \blue{每个产生式都可以关联一组规则}
  \end{center}
\end{frame}
%%%%%%%%%%%%%%%%%%%%

%%%%%%%%%%%%%%%%%%%%
\begin{frame}{}
  \begin{center}
    \begin{definition}[语法制导定义 (Syntax-Directed Definition; SDD)]
      SDD 是一个上下文无关文法和\red{\bf 属性}及\blue{\bf 规则}的结合。
    \end{definition}

    \vspace{0.50cm}
    SDD \red{\bf 唯一确定}了语法分析树上每个\red{非终结符}节点的属性值

    \fig{width = 0.60\textwidth}{figs/SDD-expr-left-recursion}

    SDD \red{\bf 没有}规定以什么方式、什么顺序计算这些属性值
  \end{center}
\end{frame}
%%%%%%%%%%%%%%%%%%%%

%%%%%%%%%%%%%%%%%%%%
\begin{frame}{}
  \begin{center}
    \red{\bf 注释(annotated)语法分析树:} 显示了各个属性值的语法分析树
    \vspace{0.30cm}
    \cyan{\texttt{ParseTreeProperty<Integer>} \qquad \texttt{put(ctx, ...)}, \texttt{get(ctx, ...)}}

    \fig{width = 0.60\textwidth}{figs/anno-expr-left-recursion}
    \[
        \teal{3 \ast 5 + 4}
    \]

    % \begin{columns}
    %   \column{0.45\textwidth}
    %     \fig{width = 0.90\textwidth}{figs/SDD-expr-left-recursion}
    %   \column{0.55\textwidth}
    %     \fig{width = 1.00\textwidth}{figs/anno-expr-left-recursion}
    %     \[
    %       \teal{3 \ast 5 + 4}
    %     \]
    % \end{columns}
  \end{center}
\end{frame}
%%%%%%%%%%%%%%%%%%%%

%%%%%%%%%%%%%%%%%%%%
\begin{frame}{}
  \begin{center}
    \begin{definition}[综合属性 (Synthesized Attribute)]
      节点$N$上的\red{\bf 综合属性}只能通过
      \blue{$N$的子节点或$N$本身}的属性来定义。
    \end{definition}

    \vspace{0.50cm}
    \fig{width = 0.60\textwidth}{figs/SDD-expr-left-recursion}

    \pause
    \vspace{0.50cm}
    \begin{definition}[$S$ 属性定义 ($S$-Attributed Definition)]
      如果一个SDD的每个属性都是综合属性, 则它是 $S$ 属性定义。
    \end{definition}
  \end{center}
\end{frame}
%%%%%%%%%%%%%%%%%%%%

%%%%%%%%%%%%%%%%%%%%
\begin{frame}{}
  \begin{center}
    \red{\bf 依赖图}用于确定一棵\blue{给定的语法分析树}中\purple{各个属性实例}之间的依赖关系

    \begin{columns}
      \column{0.45\textwidth}
        \fig{width = 0.90\textwidth}{figs/SDD-expr-left-recursion}

        \fig{width = 0.90\textwidth}{figs/dep-expr-left-recursion}
      \column{0.55\textwidth}
        \fig{width = 1.00\textwidth}{figs/anno-expr-left-recursion}
    \end{columns}

    \vspace{0.80cm}
    \teal{\bf $S$ 属性定义}的依赖图刻画了属性实例之间\teal{\bf 自底向上}的信息流动
  \end{center}
\end{frame}
%%%%%%%%%%%%%%%%%%%%

%%%%%%%%%%%%%%%%%%%%
\begin{frame}{}
  \begin{center}
    \teal{\bf $S$ 属性定义}的依赖图描述了属性实例之间\teal{\bf 自底向上}的信息流

    \vspace{0.80cm}
    此类属性值的计算可以在\purple{自顶向下}的 $LL$ 语法分析\red{\bf 过程中}实现

    \pause
    \vspace{0.80cm}
    在 $LL$ 语法分析器中, 递归下降函数 $A$ \red{\bf 返回} 时, \\[3pt]
    计算相应节点 $A$ 的综合属性值
  \end{center}
\end{frame}
%%%%%%%%%%%%%%%%%%%%

%%%%%%%%%%%%%%%%%%%%
\begin{frame}{}
  \begin{center}
    $T'$有一个综合属性$syn$与一个继承属性$inh$

    \fig{width = 0.60\textwidth}{figs/SDD-expr-no-left-recursion}

    \begin{definition}[继承属性 (Inherited Attribute)]
      节点$N$上的\red{\bf 继承属性}只能通过\blue{$N$的父节点、$N$本身和$N$的兄弟节点}上的属性来定义。
    \end{definition}
  \end{center}
\end{frame}
%%%%%%%%%%%%%%%%%%%%

%%%%%%%%%%%%%%%%%%%%
\begin{frame}{}
  \begin{center}
    继承属性 \red{$T'.inh$} 用于在表达式中从左向右传递中间计算结果

    \vspace{0.50cm}
    \begin{columns}
      \column{0.45\textwidth}
        \fig{width = 1.00\textwidth}{figs/SDD-expr-no-left-recursion}
      \column{0.55\textwidth}
        \fig{width = 1.00\textwidth}{figs/anno-expr-no-left-recursion}

        \[
            \teal{3 \ast 5}
        \]
    \end{columns}

    \pause
    \vspace{0.40cm}
    \blue{在右递归文法下实现了左结合}
  \end{center}
\end{frame}
%%%%%%%%%%%%%%%%%%%%

%%%%%%%%%%%%%%%%%%%%
\begin{frame}{}
  \begin{center}
    \red{\bf 依赖图}用于确定一棵\blue{给定的语法分析树}中\purple{各个属性实例}之间的依赖关系

    \vspace{0.50cm}
    \fig{width = 0.80\textwidth}{figs/dep-expr-no-left-recursion}

    % \begin{columns}
    %   \column{0.45\textwidth}
    %     \fig{width = 1.00\textwidth}{figs/anno-expr-no-left-recursion}
    %   \column{0.55\textwidth}
    %     \fig{width = 1.00\textwidth}{figs/dep-expr-no-left-recursion}
    % \end{columns}

    \vspace{0.50cm}
    \red{\bf 信息流向:} 先从左向右、从上到下传递信息, 再从下到上传递信息
  \end{center}
\end{frame}
%%%%%%%%%%%%%%%%%%%%

%%%%%%%%%%%%%%%%%%%%
\begin{frame}{}
  \begin{center}
    \begin{definition}[$L$ 属性定义 (\red{$L$}-Attributed Definition)]
      如果一个SDD的每个属性
      \begin{enumerate}[(1)]
        \setlength{\itemsep}{8pt}
        \item 要么是综合属性,
        \item 要么是继承属性, 但是它的规则满足如下限制: \\
          对于产生式 $A \to X_{1}X_{2} \dots X_{n}$ 及其对应规则定义的继承属性 $X_{i}.a$,
          则这个规则只能使用
          \begin{enumerate}[(a)]
            \setlength{\itemsep}{6pt}
            \item 和\red{\bf 产生式头 $A$}关联的\blue{\bf 继承}属性;
            \item 位于\red{\bf $X_{i}$左边}的文法符号实例 $X_{1}$、$X_{2}$、$\dots$、$X_{i-1}$
              相关的\blue{\bf 继承}属性或\blue{\bf 综合}属性;
            \item 和\red{\bf 这个$X_{i}$的实例本身}相关的继承属性或综合属性,
              但是在由这个$X_{i}$的全部属性组成的依赖图中\blue{\bf 不存在环}。
          \end{enumerate}
      \end{enumerate}
      则它是$L$属性定义。
    \end{definition}
  \end{center}
\end{frame}
%%%%%%%%%%%%%%%%%%%%

%%%%%%%%%%%%%%%%%%%%
\begin{frame}{}
  \begin{center}
    \fig{width = 0.70\textwidth}{figs/info-flow}
    \vspace{0.30cm}

    \blue{\bf 语法分析树}上的\red{\bf 有序}信息流动
  \end{center}
\end{frame}
%%%%%%%%%%%%%%%%%%%%
\begin{frame}{}
  \fig{width = 0.50\textwidth}{figs/learn-by-examples}
\end{frame}
%%%%%%%%%%%%%%%%%%%%
% postfix.tex

%%%%%%%%%%%%%%%%%%%%
\begin{frame}{}
  \begin{definition}[后缀表示 (Postfix Notation)]
    \begin{enumerate}[(1)]
      \setlength{\itemsep}{8pt}
      \item 如果$E$是一个 \red{\bf 变量或常量}, 则$E$的后缀表示是$E$本身;
      \item 如果$E$是形如 \red{$E_{1} \;\text{\bf op}\; E_{2}$} 的表达式,
        则$E$的后缀表示是$E'_{1} E'_{2} \text{\bf op}$,
        这里 $E'_{1}$ 和 $E'_{2}$ 分别是 $E_{1}$ 与 $E_{2}$ 的后缀表达式;
      \item 如果$E$是形如 \red{$(E_{1})$} 的表达式, 则$E$的后缀表示是$E_{1}$的后缀表示。
    \end{enumerate}
  \end{definition}

  \begin{center}
    \[
      (9 * 5)+2 \implies 95*2+
    \]
    \[
      9 * (5+2) \implies 952+*
    \]
  \end{center}
\end{frame}
%%%%%%%%%%%%%%%%%%%%

%%%%%%%%%%%%%%%%%%%%
\begin{frame}{}
  \begin{center}
	  \fig{width = 0.35\textwidth}{figs/PostfixExprAG-tree}
	\end{center}

	\vspace{-0.30cm}
	\[
		9 * (5+2) \implies 952+*
	\]
\end{frame}
%%%%%%%%%%%%%%%%%%%%

%%%%%%%%%%%%%%%%%%%%
\begin{frame}{}
  \begin{center}
    \red{\bf 后缀表达式} $S$ 属性定义

		\vspace{0.50cm}
	  \fig{width = 1.00\textwidth}{figs/PostfixExprAG-g4}
	\end{center}
\end{frame}
%%%%%%%%%%%%%%%%%%%%
% binary-number.tex

%%%%%%%%%%%%%%%%%%%%
\begin{frame}{}
  \begin{center}
    \red{\bf 有符号二进制数}文法

    \fig{width = 0.90\textwidth}{figs/cfg-signed-binary}
    \[
      \teal{-101}_{2} = \blue{-5}_{10}
    \]
  \end{center}
\end{frame}
%%%%%%%%%%%%%%%%%%%%

%%%%%%%%%%%%%%%%%%%%
\begin{frame}{}
  \begin{center}
    \red{\bf 有符号二进制数} $L$ 属性定义

    \fig{width = 0.80\textwidth}{figs/SDD-signed-binary}
  \end{center}
\end{frame}
%%%%%%%%%%%%%%%%%%%%

%%%%%%%%%%%%%%%%%%%%
\begin{frame}{}
  \begin{center}
    \fig{width = 0.70\textwidth}{figs/anno-signed-binary}
    \vspace{-0.60cm}
    \[
      \teal{-101}_{2} = \blue{-5}_{10}
    \]
  \end{center}
\end{frame}
%%%%%%%%%%%%%%%%%%%%

% %%%%%%%%%%%%%%%%%%%%
% \begin{frame}{}
%   \begin{center}
%     (左递归) $S$属性文法 $\implies$ (右递归) $L$属性文法

%     \begin{columns}
%       \column{0.50\textwidth}
%         \fig{width = 1.00\textwidth}{figs/SDD-expr-left-recursion}
%       \column{0.50\textwidth}
%         \fig{width = 1.00\textwidth}{figs/SDD-expr-no-left-recursion}
%     \end{columns}

%     \vspace{0.60cm}
%     仍保持了操作的左结合性
%   \end{center}
% \end{frame}
% %%%%%%%%%%%%%%%%%%%%

% %%%%%%%%%%%%%%%%%%%%
% \begin{frame}{}
%   \begin{center}
%     (左递归) $S$属性定义
%     \vspace{-0.50cm}
%     % left-recursion-S-SDD.tex

\begin{alignat*}{3}
  A &\to A_{1} Y \qquad && A.a = g(A_{1}.a, Y.y) \\[8pt]
  A &\to X && A.a = f(X.x)
\end{alignat*}

%     \pause
%     \vspace{0.80cm}
%     (右递归) $L$属性定义
%     \vspace{-0.50cm}
%     % right-recursion-L-SDD.tex

\begin{alignat*}{3}
  A &\to XR \quad && \red{R.i} = f(X.x);\; A.a = R.s \\[8pt]
  R &\to YR_{1} && \red{R_{1}.i} = g(R.i, Y.y);\; R.s = R_{1}.s \\[8pt]
  R &\to \epsilon && \blue{R.s = R.i}
\end{alignat*}
%   \end{center}
% \end{frame}
% %%%%%%%%%%%%%%%%%%%%

% array-type.tex

%%%%%%%%%%%%%%%%%%%%
\begin{frame}{}
  \begin{center}
    \red{\bf 数组类型}文法举例

    \fig{width = 0.30\textwidth}{figs/SDD-type-array}
    \[
      \teal{\intkw[2][3]}
    \]

    \red{\bf 类型表达式} \blue{$(2, (3, \textrm{int}))$}
  \end{center}
\end{frame}
%%%%%%%%%%%%%%%%%%%%

%%%%%%%%%%%%%%%%%%%%
\begin{frame}{}
  \begin{center}
    继承属性 \red{$C.b$} 将一个基本类型沿着树向下传播

    \fig{width = 0.80\textwidth}{figs/anno-type-array}
    \vspace{-0.30cm}
    \[
        \teal{\intkw[2][3]}
    \]

    \vspace{0.30cm}
    综合属性 \blue{$C.t$} 收集最终得到的类型表达式
  \end{center}
\end{frame}
%%%%%%%%%%%%%%%%%%%%

%%%%%%%%%%%%%%%%%%%%
\begin{frame}{}
  \fig{width = 1.00\textwidth}{figs/ArrayType-g4}
  \begin{center}
    \teal{\texttt{ArrayType.g4}}
  \end{center}
\end{frame}
%%%%%%%%%%%%%%%%%%%%

%%%%%%%%%%%%%%%%%%%%
\begin{frame}{}
  \fig{width = 1.00\textwidth}{figs/ArrayType-tree}
  \[
      \teal{\intkw[2][3]}
  \]
\end{frame}
%%%%%%%%%%%%%%%%%%%%

%%%%%%%%%%%%%%%%%%%%
\begin{frame}{}
  \fig{width = 1.00\textwidth}{figs/ArrayTypeAG-g4}
  \begin{center}
    \teal{\texttt{ArrayTypeAG.g4}}
  \end{center}
\end{frame}
%%%%%%%%%%%%%%%%%%%%

%%%%%%%%%%%%%%%%%%%%
\begin{frame}{}
  \begin{center}
    数组声明、数组引用、类型检查
    \fig{width = 0.85\textwidth}{figs/Array-g4}
    \vspace{-0.30cm}
    \teal{\texttt{Array.g4}}
  \end{center}
\end{frame}
%%%%%%%%%%%%%%%%%%%%

%%%%%%%%%%%%%%%%%%%%
\begin{frame}{}
  \begin{columns}
    \column{0.50\textwidth}
      \fig{width = 0.50\textwidth}{figs/Array-g4-input}
    \column{0.50\textwidth}
      \fig{width = 0.70\textwidth}{figs/Array-g4-output}
  \end{columns}

  \vspace{0.80cm}
  \fig{width = 1.00\textwidth}{figs/Array-tree}
\end{frame}
%%%%%%%%%%%%%%%%%%%%

%%%%%%%%%%%%%%%%%%%%
\begin{frame}{}
  \fig{width = 1.00\textwidth}{figs/ArrayAG-stat}
  \begin{center}
    \teal{\texttt{ArrayAG.g4}}
  \end{center}
\end{frame}
%%%%%%%%%%%%%%%%%%%%

%%%%%%%%%%%%%%%%%%%%
\begin{frame}{}
  \fig{width = 1.00\textwidth}{figs/ArrayAG-varDecl-arrDecl}
  \begin{center}
    \teal{\texttt{ArrayAG.g4}}
  \end{center}
\end{frame}
%%%%%%%%%%%%%%%%%%%%

%%%%%%%%%%%%%%%%%%%%
\begin{frame}{}
  \fig{width = 1.00\textwidth}{figs/ArrayAG-expr}
  \begin{center}
    \teal{\texttt{ArrayAG.g4}}
  \end{center}
\end{frame}
%%%%%%%%%%%%%%%%%%%%

%%%%%%%%%%%%%%%%%%%%
\begin{frame}{}
  \begin{columns}
    \column{0.50\textwidth}
      \begin{center}
        \fig{width = 0.70\textwidth}{figs/patterns-book-ch}
        \vspace{0.30cm}
        第八章: 静态类型检查
      \end{center}
    \column{0.50\textwidth}
      \fig{width = 0.80\textwidth}{figs/pfpl-book}
  \end{columns}
\end{frame}
%%%%%%%%%%%%%%%%%%%%

%%%%%%%%%%%%%%%%%%%%
\begin{frame}{}
  \fig{width = 0.80\textwidth}{figs/TypeCheckingListener-Fields}
  \begin{center}
    \teal{\texttt{TypeCheckingListener.java}}
  \end{center}
\end{frame}
%%%%%%%%%%%%%%%%%%%%
% % ast.tex

%%%%%%%%%%%%%%%%%%%%
% \begin{frame}{}
%   \begin{center}
%     表达式的\red{\bf 抽象语法树} $S$属性定义

%     \fig{width = 0.90\textwidth}{figs/SDD-expr-ast-S}
%   \end{center}
% \end{frame}
%%%%%%%%%%%%%%%%%%%%

%%%%%%%%%%%%%%%%%%%%
% \begin{frame}{}
%   \begin{center}
%     \red{\bf 抽象语法树:} 丢弃非本质的东西, 仅保留重要结构信息

%     \fig{width = 0.70\textwidth}{figs/ast-S}
%     \vspace{-0.50cm}
%     \[
%       \teal{a - 4 + c}
%     \]
%   \end{center}
% \end{frame}
%%%%%%%%%%%%%%%%%%%%

%%%%%%%%%%%%%%%%%%%%
% \begin{frame}{}
%   \begin{center}
%     表达式的\red{\bf 抽象语法树} $L$属性定义

%     \fig{width = 0.80\textwidth}{figs/SDD-expr-ast-L}
%   \end{center}
% \end{frame}
%%%%%%%%%%%%%%%%%%%%

%%%%%%%%%%%%%%%%%%%%
% \begin{frame}{}
%   \begin{center}
%     \fig{width = 0.90\textwidth}{figs/ast-L}
%     \[
%       \teal{a - 4 + c}
%     \]
%     \pause
%     通过\red{继承属性}重新实现\red{左结合}
%   \end{center}
% \end{frame}
%%%%%%%%%%%%%%%%%%%%
% csv.tex

%%%%%%%%%%%%%%%%%%%%
\begin{frame}{}
  \begin{center}
    \fig{width = 0.50\textwidth}{figs/csv}
    \blue{\bf Comma-Separated Values}
  \end{center}
\end{frame}
%%%%%%%%%%%%%%%%%%%%

%%%%%%%%%%%%%%%%%%%%
\begin{frame}{}
  \begin{columns}
    \column{0.40\textwidth}
      \fig{width = 0.80\textwidth}{figs/csv-scores-input}
    \column{0.60\textwidth}
      \uncover<2->{\fig{width = 0.90\textwidth}{figs/csv-scores-output}}
  \end{columns}

  \vspace{0.30cm}
  \fig{width = 1.00\textwidth}{figs/csv-tree}

  \fig{width = 0.40\textwidth}{figs/csv-g4}
\end{frame}
%%%%%%%%%%%%%%%%%%%%

%%%%%%%%%%%%%%%%%%%%
\begin{frame}{}
  \fig{width = 1.00\textwidth}{figs/CSVAG-g4}

	\begin{center}
		\teal{\texttt{CSVAG.g4}}
	\end{center}
\end{frame}
%%%%%%%%%%%%%%%%%%%%

%%%%%%%%%%%%%%%%%%%%
% \begin{frame}{}
%   \fig{width = 1.00\textwidth}{figs/csv-file-rule}
% \end{frame}
%%%%%%%%%%%%%%%%%%%%

%%%%%%%%%%%%%%%%%%%%
% \begin{frame}{}
%   \fig{width = 0.85\textwidth}{figs/csv-hdr-rule}
% \end{frame}
%%%%%%%%%%%%%%%%%%%%

%%%%%%%%%%%%%%%%%%%%
% \begin{frame}{}
%   \fig{width = 0.85\textwidth}{figs/csv-row-rule}
% \end{frame}
%%%%%%%%%%%%%%%%%%%%  % left as hw?
%%%%%%%%%%%%%%%%%%%%