% lr1.tex

%%%%%%%%%%%%%%%%%%%%
\begin{frame}{}
  \begin{center}
    \red{$SLR(1)$ 分析表}

    \fig{width = 0.60\textwidth}{figs/lr0-table-expr-add-mul}

    \blue{\bf 归约:}
    \begin{enumerate}[(3)]
      \centering
      \item $[k: A \to \alpha \cdot] \in I_{i} \land \blue{A \neq S'} \implies
        \forall t \in \red{\follow(A)}.\; \action[i, t] = rk$
    \end{enumerate}
  \end{center}
\end{frame}
%%%%%%%%%%%%%%%%%%%%

%%%%%%%%%%%%%%%%%%%%
\begin{frame}{}
  \begin{definition}[$SLR(1)$文法]
    如果文法 $G$ 的\red{\bf $SLR(1)$分析表}是\blue{\bf 无冲突}的,
    则 $G$ 是 $SLR(1)$ 文法。
  \end{definition}

  \vspace{0.30cm}
  \begin{center}
    \blue{\bf 无冲突:} \action{}表中每个单元格最多只有一种动作 \\[8pt]

    \fig{width = 0.50\textwidth}{figs/lr0-table-expr-add-mul}

    \pause
    \blue{\bf 两类可能的冲突:} ``移入/归约''冲突、``归约/归约''冲突
  \end{center}
\end{frame}
%%%%%%%%%%%%%%%%%%%%

%%%%%%%%%%%%%%%%%%%%
\begin{frame}{}
  \begin{center}
    非 $SLR(1)$ 文法举例
  \end{center}

  \begin{columns}
    \column{0.50\textwidth}
      \fig{width = 0.60\textwidth}{figs/cfg-LR}
    \column{0.50\textwidth}
      \fig{width = 1.00\textwidth}{figs/lr0-automaton-LR-2}
  \end{columns}

  \vspace{0.30cm}
  \[
    [S \to L \cdot = R] \in I_{2} \implies \action(I_{2}, =) \gets s6
  \]
  \[
    \red{=\; \in \follow(R)} \implies \action(I_{2}, =) \gets r5
  \]
\end{frame}
%%%%%%%%%%%%%%%%%%%%

%%%%%%%%%%%%%%%%%%%%
\begin{frame}{}
  \begin{center}
    即使考虑了\red{$=\; \in \follow(R)$}, 对该文法来说仍然不够

    \vspace{0.20cm}
    因为, 这仅仅说明 $=$ 可能跟在 $R$ 后面, 是一种``粗糙''的估计

    \fig{width = 0.80\textwidth}{figs/lr0-automaton-SLR}

    \pause
    实际上, 该文法没有\blue{\bf 以 $R = \cdots$ 开头}的最右句型
  \end{center}
\end{frame}
%%%%%%%%%%%%%%%%%%%%

%%%%%%%%%%%%%%%%%%%%
\begin{frame}{}
  \begin{center}
    希望$LR$语法分析器的每个状态能\blue{\bf 尽可能精确}地

    \vspace{0.30cm}
    指明\red{\bf 哪些输入符号可以跟在句柄$A \to \alpha$的后面}

    \pause
    \vspace{1.00cm}
    在 $LR(0)$ 自动机中, 某个项集 \purple{$I_{j}$} 中包含 \purple{$[A \to \alpha \cdot]$}

    \vspace{0.50cm}
    则在之前的某个项集 \teal{$I_{i}$} 中包含 \teal{$[B \to \beta \cdot A \gamma]$}
    与 \teal{$[A \to \cdot \alpha]$}

    \vspace{0.50cm}
    \red{\fbox{这表明只有$a \in \first(\gamma)$时, 才可以进行 $A \to \alpha$ 归约}}

    \pause
    \vspace{0.80cm}
    但是, 对$I_{i}$求闭包时, 仅得到 $[A \to \cdot \alpha]$, 丢失了 $\first(\gamma)$ 信息
  \end{center}
\end{frame}
%%%%%%%%%%%%%%%%%%%%

%%%%%%%%%%%%%%%%%%%%
\begin{frame}{}
  \begin{center}
    \begin{definition}[$LR(1)$项 (Item)]
      \[
        [A \to \alpha \cdot \beta, \red{a}] \qquad (a \in T \cup \set{\$})
      \]

      此处, $a$ 是\red{\bf 向前看符号}, 数量为 \red{\bf 1}.
    \end{definition}

    \pause
    \vspace{0.80cm}
    思想: $\alpha$ 在栈顶, 期望剩余输入中\red{\bf 开头}的是可以从$\beta \red{a}$推导出的符号串

    \pause
    \vspace{0.50cm}
    \[
      \blue{[A \to \alpha \cdot, a]}
    \]
    只有下一个输入符号为 $a$ 时, 才可以按照 $A \to \alpha$ 进行归约
  \end{center}
\end{frame}
%%%%%%%%%%%%%%%%%%%%

%%%%%%%%%%%%%%%%%%%%
\begin{frame}{}
  \begin{center}
    \[
      [A \to \alpha \cdot B \beta, \red{a}] \in I \qquad (a \in T \cup \set{\$})
    \]

    % \fig{width = 0.80\textwidth}{figs/lr1-closure}

    \pause
    \[
      \forall \blue{b \in \first(\beta a)}.\; [B \to \cdot \gamma, b] \in I
    \]
  \end{center}
\end{frame}
%%%%%%%%%%%%%%%%%%%%

%%%%%%%%%%%%%%%%%%%%
\begin{frame}{}
  \begin{center}
    % \fig{width = 0.80\textwidth}{figs/lr1-goto}
    \begin{align*}
      J = \textsc{goto}(I, \red{X}) &= \closure\Big(\Big\{
            \blue{[A \to \alpha X \cdot \beta, a]}
            \Big\lvert \red{[A \to \alpha \cdot X \beta, a]} \in I \Big\}\Big)
    \end{align*}
    \[
      (X \in N \cup T)
    \]
  \end{center}
\end{frame}
%%%%%%%%%%%%%%%%%%%%

%%%%%%%%%%%%%%%%%%%%
\begin{frame}{}
  \begin{center}
    % \fig{width = 0.80\textwidth}{figs/lr1-items}

    % \vspace{0.60cm}
    初始状态: $\closure({[S' \to \cdot S, \red{\$}]})$
  \end{center}
\end{frame}
%%%%%%%%%%%%%%%%%%%%

%%%%%%%%%%%%%%%%%%%%
\begin{frame}{}
  \begin{center}
    \begin{columns}
      \column{0.25\textwidth}
        \fig{width = 0.70\textwidth}{figs/cfg-SCC}

        \uncover<2->{
        \begin{table}
          \footnotesize
          \begin{tabular}{c|c|c}
            \hline
              & \first & \follow \\ \hline \hline
            $S$ & $\set{c, d}$ & $\$$ \\ \hline
            $C$ & $\set{c, d}$ & $\set{c, d, \$}$ \\ \hline
          \end{tabular}}
        \end{table}
      \column{0.75\textwidth}
        \fig{width = 1.00\textwidth}{figs/lr1-automaton-SCC}
    \end{columns}
  \end{center}
\end{frame}
%%%%%%%%%%%%%%%%%%%%

%%%%%%%%%%%%%%%%%%%%
\begin{frame}{}
  \begin{center}
    \red{\bf $LR(1)$分析表构造规则}

    \vspace{0.60cm}
    \begin{enumerate}[(1)]
      \setlength{\itemsep}{25pt}
      \item $\goto(I_{i}, a) = I_{j} \land a \in T \implies \action[i, a] \gets sj$
      \item $\goto(I_{i}, A) = I_{j} \land A \in N \implies \goto[i, A] \gets gj$
      \item $[k: A \to \alpha \cdot, \red{a}] \in I_{i} \land \blue{A \neq S'} \implies
        \action[i, \red{a}] = rk$
      \item $[S' \to S \cdot, \purple{\$}] \in I_{i} \implies \action[i, \$] \gets acc$
    \end{enumerate}

    \pause
    \vspace{0.20cm}
    \begin{definition}[$LR(1)$文法]
      如果文法 $G$ 的\red{\bf $LR(1)$分析表}是\blue{\bf 无冲突}的,
      则 $G$ 是 $LR(1)$ 文法。
    \end{definition}
  \end{center}
\end{frame}
%%%%%%%%%%%%%%%%%%%%

%%%%%%%%%%%%%%%%%%%%
\begin{frame}{}
  \begin{center}
    \begin{columns}
      \column{0.70\textwidth}
        \fig{width = 1.00\textwidth}{figs/lr1-automaton-SCC}
      \column{0.35\textwidth}
        \fig{width = 1.00\textwidth}{figs/lr1-table-SCC}
    \end{columns}

    \vspace{0.60cm}
    $LR(1)$ 通过\red{\bf 不同的向前看符号}, 区分了状态对 $(3, 6)$, $(4, 7)$ 与 $(8, 9)$
  \end{center}
\end{frame}
%%%%%%%%%%%%%%%%%%%%

%%%%%%%%%%%%%%%%%%%%
\begin{frame}{}
  \begin{center}
    \begin{columns}
      \column{0.30\textwidth}
        \uncover<3->{
          \[
            w = ccdcd\$
          \]}

        \fig{width = 0.70\textwidth}{figs/cfg-SCC}

        \[
          \blue{L(G) = \uncover<2->{c^{\ast}dc^{\ast}d}}
        \]
      \column{0.80\textwidth}
        \fig{width = 1.00\textwidth}{figs/lr1-automaton-SCC}
    \end{columns}
  \end{center}
\end{frame}
%%%%%%%%%%%%%%%%%%%%

%%%%%%%%%%%%%%%%%%%%
\begin{frame}{}
  \begin{center}
    总结: $LR(0)$、$SLR(1)$、$LR(1)$ 的\red{\bf 归约}条件

    \[
      [k: A \to \alpha \cdot] \in I_{i} \land A \neq S' \implies
        \forall t \in \red{T \cup \set{\$}}.\; \action[i, t] = rk
    \]

    \[
      [k: A \to \alpha \cdot] \in I_{i} \land A \neq S' \implies
        \forall t \in \red{\follow(A)}.\; \action[i, t] = rk
    \]

    \[
      [k: A \to \alpha \cdot, \red{a}] \in I_{i} \land A \neq S' \implies
        \action[i, \red{a}] = rk
    \]
  \end{center}
\end{frame}
%%%%%%%%%%%%%%%%%%%%

%%%%%%%%%%%%%%%%%%%%
\begin{frame}{}
  \begin{center}
    $LR(1)$虽然强大, 但是生成的$LR(1)$分析表可能过大, 状态过多
    \vspace{-0.30cm}

    \begin{columns}
      \column{0.70\textwidth}
        \fig{width = 1.00\textwidth}{figs/lr1-automaton-SCC}
      \column{0.35\textwidth}
        \fig{width = 1.00\textwidth}{figs/lr1-table-SCC}
    \end{columns}

    \pause
    \vspace{0.10cm}
    \red{$LALR(1):$} 合并具有相同\blue{\bf 核心$LR(0)$}项的状态 (忽略不同的向前看符号)
  \end{center}
\end{frame}
%%%%%%%%%%%%%%%%%%%%

%%%%%%%%%%%%%%%%%%%%
\begin{frame}{}
  \begin{center}
    \begin{columns}
      \column{0.30\textwidth}
        % \[
        %   w = ccdcd\$
        % \]

        \fig{width = 0.60\textwidth}{figs/cfg-SCC}

        \[
          \blue{L(G) = c^{\ast}dc^{\ast}d}
        \]
      \column{0.75\textwidth}
        \fig{width = 1.00\textwidth}{figs/lr1-automaton-SCC}
    \end{columns}

    \vspace{0.20cm}
    \blue{合并 $I_{4}$ 与 $I_{7}$ 为 $I_{47}$
      $\set{[C \to d \cdot, c/d/\$]}$},
    \pause \blue{继续合并 $(I_{8}, I_{9})$ 以及 $(I_{3}, I_{6})$}
  \end{center}
\end{frame}
%%%%%%%%%%%%%%%%%%%%

%%%%%%%%%%%%%%%%%%%%
% \begin{frame}{}
%   \begin{center}
%     % \teal{$w = ccd\$ \qquad w = cdcdc\$$}

%     \begin{columns}
%       \column{0.30\textwidth}
%         \fig{width = 0.60\textwidth}{figs/cfg-SCC}

%         \[
%           \blue{L(G) = c^{\ast}dc^{\ast}d}
%         \]
%       \column{0.75\textwidth}
%         \fig{width = 1.00\textwidth}{figs/lr1-automaton-SCC}
%     \end{columns}

%     \pause
%     \vspace{0.20cm}
%     \blue{\bf 继续合并 $(I_{8}, I_{9})$ 以及 $(I_{3}, I_{6})$}
%   \end{center}
% \end{frame}
%%%%%%%%%%%%%%%%%%%%

%%%%%%%%%%%%%%%%%%%%
\begin{frame}{}
  \begin{columns}
    \column{0.50\textwidth}
      \fig{width = 0.80\textwidth}{figs/lr1-table-SCC}
    \column{0.50\textwidth}
      \fig{width = 0.95\textwidth}{figs/lalr-table-SCC}
  \end{columns}

  % \pause
  % \begin{center}
  %   \vspace{0.60cm}
  %   \red{$Q:$ \goto{}函数怎么办?}

  %   \pause
  %   \vspace{0.50cm}
  %   \teal{$A:$ 可以合并的状态 $I, J$ 的\goto{}目标(状态)一定也是可以合并的}

  %   \pause
  %   \vspace{0.50cm}
  %   ($I, J$ 出边集合相同, 且 $\goto(I, X)$ 与 $\goto(J, X)$ 具有相同的核心项)
  % \end{center}
\end{frame}
%%%%%%%%%%%%%%%%%%%%

%%%%%%%%%%%%%%%%%%%%
\begin{frame}{}
  \begin{center}
    \red{$Q:$ 对于 $LR(1)$ 文法, 合并得到的$LALR(1)$ 分析表是否会引入冲突?}

    \pause
    \begin{theorem}
      $LALR(1)$ 分析表\red{\bf 不会}引入移入/归约冲突。
    \end{theorem}

    \pause
    \vspace{0.50cm}
    \purple{\bf 反证法}

    \vspace{0.30cm}
    假设合并后出现 $[A \to \alpha \cdot, a]$ 与 $[B \to \beta \cdot a \gamma, b]$

    \pause
    \vspace{0.60cm}
    则在$LR(1)$自动机中,

    \vspace{0.20cm}
    存在某状态同时包含 $[A \to \alpha \cdot, a]$ 与 $[B \to \beta \cdot a \gamma, c]$ ($c \neq b$)

    \pause
    \vspace{0.60cm}
    矛盾!
  \end{center}
\end{frame}
%%%%%%%%%%%%%%%%%%%%

%%%%%%%%%%%%%%%%%%%%
\begin{frame}{}
  \begin{center}
    \red{$Q:$ 对于 $LR(1)$ 文法, 合并得到的$LALR(1)$ 分析表是否会引入冲突?}

    \begin{theorem}
      $LALR(1)$ 分析表\blue{\bf 可能会}引入归约/归约冲突。
    \end{theorem}

    \pause
    \[
      L(G) = \set{acd, ace, bcd, bce}
    \]
    \fig{width = 0.50\textwidth}{figs/lalr-rr-conflict}

    \pause
    \vspace{-0.30cm}
    \begin{gather*}
      \set{[A \to c \cdot, d], [B \to c \cdot, e]} \\
      \set{[A \to c \cdot, e], [B \to c \cdot, d]} \\[5pt]
      \set{[A \to c \cdot, \red{d/e}], [B \to c \cdot, \red{d/e}]} \\
    \end{gather*}
  \end{center}
\end{frame}
%%%%%%%%%%%%%%%%%%%%

%%%%%%%%%%%%%%%%%%%%
% \begin{frame}{}
%   \begin{center}
%     \begin{theorem}
%       如果合并后的语法分析器\blue{\bf 无冲突}, 则它的行为与原分析器本质上\red{\bf 一致}。
%     \end{theorem}

%     \vspace{0.80cm}
%     \begin{enumerate}[(1)]
%       \setlength{\itemsep}{15pt}
%       \item \green{\bf 接受}原分析器所接受的句子, 且状态转移相同
%       \item \purple{\bf 拒绝}原分析器所拒绝的句子, 但可能多一些不必要的\teal{\bf 归约}动作

%         \begin{center}
%           \pause
%           \vspace{0.50cm}
%           (``实际上, 这个错误会在移入任何新的输入符号之前就被发现'')

%           \pause
%           \vspace{0.50cm}
%           (``两个分析器有相同的\red{\bf 移入}动作'')
%         \end{center}
%     \end{enumerate}
%   \end{center}
% \end{frame}
%%%%%%%%%%%%%%%%%%%%

%%%%%%%%%%%%%%%%%%%%
\begin{frame}{}
  \begin{center}
    \blue{\bf $LALR(1)$语法分析器的优点}

    \vspace{0.80cm}
    状态数量与$SLR(1)$语法分析器的状态\purple{\bf 数量相同} \\[10pt]
    \uncover<2->{($LALR(1)$ 与 $SLR(1)$ 都使用相同的 $LR(0)$ 核心项)}

    \vspace{0.80cm}
    对于$LR(1)$文法, \red{\bf 不会}产生移入/归约冲突
  \end{center}
\end{frame}
%%%%%%%%%%%%%%%%%%%%

%%%%%%%%%%%%%%%%%%%%
\begin{frame}{}
  \begin{center}
    \red{Q: 但是你却通过 $LR(1)$ 自动机构造$LALR(1)$ 项集族?}

    \pause
    \vspace{1.20cm}
    第 \blue{\bf 4.7.5} 节 (本科教学版): 高效构造 $LALR(1)$ 语法分析表的方法
  \end{center}
\end{frame}
%%%%%%%%%%%%%%%%%%%%

%%%%%%%%%%%%%%%%%%%%
\begin{frame}{}
  \fig{width = 0.60\textwidth}{figs/cfg-hierarchy}

  \pause
  \vspace{0.20cm}
  \begin{center}
    ``除 $LR(0)$ 外, 以上各种 $LR$ 类\blue{\bf 文法}对应的\red{\bf 语言}是等价的''
  \end{center}
\end{frame}
%%%%%%%%%%%%%%%%%%%%

%%%%%%%%%%%%%%%%%%%%
\begin{frame}{}
  \begin{center}
    \href{https://en.wikipedia.org/wiki/Comparison_of_parser_generators\#Deterministic_context-free_languages}{List of Parser Generators}

    \vspace{0.50cm}
    \fig{width = 1.00\textwidth}{figs/parser-generator-wiki}

    \vspace{0.50cm}
    Bison/Yacc/Lark
  \end{center}
\end{frame}
%%%%%%%%%%%%%%%%%%%%

%%%%%%%%%%%%%%%%%%%%
\begin{frame}{}
  \fig{width = 0.90\textwidth}{figs/python-peg}
\end{frame}
%%%%%%%%%%%%%%%%%%%%

%%%%%%%%%%%%%%%%%%%%
\begin{frame}{}
  \fig{width = 1.00\textwidth}{figs/python-bison-yacc}
\end{frame}
%%%%%%%%%%%%%%%%%%%%

%%%%%%%%%%%%%%%%%%%%
\begin{frame}{}
  \begin{center}
    \green{\bf 好消息:} 善用 $LR$ 语法分析器, 处理\blue{\bf 二义性}文法

    \pause
    \vspace{1.00cm}
    \red{\bf 坏消息:} 二义性表现为``冲突'', 需要根据某种规则进行消解
  \end{center}
\end{frame}
%%%%%%%%%%%%%%%%%%%%

%%%%%%%%%%%%%%%%%%%%
\begin{frame}{}
  \begin{center}
    \red{\bf 表达式文法}

    \begin{columns}
      \column{0.50\textwidth}
        % cfg-expr-add-mul.tex

\begin{empheq}[box=\widefbox]{align*}
  E \to E + E \mid E \ast E \mid (E) \mid -E \mid \id
\end{empheq}
      \column{0.50\textwidth}
        % cfg-expr-add-mul-mul-first.tex

\begin{empheq}[box=\widefbox]{align*}
  E &\to E + \red{T} \mid \red{T} \\[8pt]
  T &\to \red{T \ast F} \mid F \\[8pt]
  F &\to \purple{(E)} \mid \id
\end{empheq}
    \end{columns}

    % cfg-expr-add-mul-no-left-recursion.tex

\begin{empheq}[box=\widefbox]{align*}
  E &\to T E' \\[8pt]
  E' &\to +\; T E' \mid \epsilon \\[8pt]
  T &\to F T' \\[8pt]
  T' &\to \ast\; F T' \mid \epsilon \\[8pt]
  F &\to (E) \mid \id
\end{empheq}
  \end{center}
\end{frame}
%%%%%%%%%%%%%%%%%%%%

%%%%%%%%%%%%%%%%%%%%
\begin{frame}{}
  \begin{center}
    \red{\bf 表达式文法:} 使用 $SLR(1)$ 语法分析方法

    \vspace{-0.50cm}
    % cfg-expr-add-mul.tex

\begin{empheq}[box=\widefbox]{align*}
  E \to E + E \mid E \ast E \mid (E) \mid -E \mid \id
\end{empheq}

    \vspace{-0.30cm}
    \begin{columns}
      \column{0.60\textwidth}
        \fig{width = 0.65\textwidth}{figs/lr0-automaton-expr-E}
      \column{0.45\textwidth}
        \pause
        \[
          \set{+, \ast} \subseteq \follow(E)
        \]

        \pause
        \begin{center}
          考虑到\blue{\bf 结合性与优先级}:
          \fig{width = 1.00\textwidth}{figs/slr-table-expr-E}
        \end{center}
    \end{columns}

    \vspace{-0.30cm}
    \pause
    \[
      \id + \id \red{\;\ast\; \id} \qquad \id + \id \red{\;+\; \id}
    \]
  \end{center}
\end{frame}
%%%%%%%%%%%%%%%%%%%%

%%%%%%%%%%%%%%%%%%%%
% \begin{frame}{}
%   \fig{width = 0.80\textwidth}{figs/LRExprPrec}
% \end{frame}
%%%%%%%%%%%%%%%%%%%%

%%%%%%%%%%%%%%%%%%%%
\begin{frame}{}
  \begin{center}
    \red{\bf 条件语句文法}

    \begin{columns}
      \column{0.50\textwidth}
        \fig{width = 1.00\textwidth}{figs/cfg-if-then-else}
      \column{0.50\textwidth}
        \fig{width = 0.80\textwidth}{figs/cfg-if-then-else-short-2}
    \end{columns}
  \end{center}
\end{frame}
%%%%%%%%%%%%%%%%%%%%

%%%%%%%%%%%%%%%%%%%%
\begin{frame}{}
  \begin{center}
    \red{\bf 条件语句文法:} 使用$SLR(1)$语法分析方法
    \fig{width = 0.30\textwidth}{figs/cfg-if-then-else-short-2}

    \begin{columns}
      \column{0.55\textwidth}
        \fig{width = 1.00\textwidth}{figs/lr0-items-if-then-else}
        \[
          e \in \follow(S)
        \]
      \column{0.45\textwidth}
        \fig{width = 1.00\textwidth}{figs/slr-table-if-then-else-short-2}
        \[
          \action[4, e] = s5
        \]
    \end{columns}
  \end{center}
\end{frame}
%%%%%%%%%%%%%%%%%%%%

%%%%%%%%%%%%%%%%%%%%
% \begin{frame}{}
%   \fig{width = 0.30\textwidth}{figs/cfg-if-then-else-short-left-factor}

%   \fig{width = 0.90\textwidth}{figs/ll1-table-if-then-else-short}

%   \[
%     \ifkw\; b \;\thenkw\; \ifkw\; b \;\thenkw\; a \;\elsekw\; a
%   \]

%   \begin{center}
%     \red{\bf 解决二义性:} 选择 $S' \to eS$, 将 $\elsekw$ 与前面最近的 $\thenkw$ 关联起来
%   \end{center}
% \end{frame}
%%%%%%%%%%%%%%%%%%%%

%%%%%%%%%%%%%%%%%%%%
% \begin{frame}{}
%   \fig{width = 0.95\textwidth}{figs/IfStat}
% \end{frame}
%%%%%%%%%%%%%%%%%%%%