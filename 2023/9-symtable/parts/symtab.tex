% symtab.tex

%%%%%%%%%%%%%%%%%%%%
\begin{frame}{}
  \begin{columns}
    \column{0.50\textwidth}
      \begin{center}
        \fig{width = 0.95\textwidth}{figs/front-end}
      \end{center}
    \column{0.50\textwidth}
      \begin{center}
        \fig{width = 0.50\textwidth}{figs/back-end}
      \end{center}
  \end{columns}
\end{frame}
%%%%%%%%%%%%%%%%%%%%

%%%%%%%%%%%%%%%%%%%%
\begin{frame}{}
  \begin{center}
    \begin{definition}[符号表 (Symbol Table)]
      \red{\bf 符号表}是用于保存\blue{\bf 各种信息}的\purple{\bf 数据结构}。
    \end{definition}
  \end{center}

  \pause
  % symtable.tex

\begin{table}[]
  \centering
  \renewcommand\arraystretch{1.6}
  \resizebox{0.95\textwidth}{!}{%
  \begin{tabular}{|c|c|c|c|c|c|c|c|}
    \hline
    \textbf{Name}  & \textbf{Type} & \textbf{Size} & \textbf{Dimension} & \textbf{Line of Declaration} & \textbf{Line of Usage} & \textbf{Address} & \textbf{$\cdots$} \\ \hline
    \textit{count} & int  & 4 & 0 & $\cdots$ & \textbf{$\cdots$} & $\cdots$ & $\cdots$ \\ \hline
    \textit{str}  & char[] & 5  & 1 & $\cdots$ & $\cdots$ & $\cdots$ & $\cdots$ \\ \hline
  \end{tabular}%
}
\end{table}
\end{frame}
%%%%%%%%%%%%%%%%%%%%

%%%%%%%%%%%%%%%%%%%%
\begin{frame}{}
  \begin{center}
    ``领域特定语言'' (DSL) 通常只有\blue{\bf 单作用域} (全局作用域)
  \end{center}
  \fig{width = 0.60\textwidth}{figs/config-file}
\end{frame}
%%%%%%%%%%%%%%%%%%%%

%%%%%%%%%%%%%%%%%%%%
\begin{frame}{}
  \begin{center}
    ``通用程序设计语言'' (GPL) 通常需要\blue{\bf 嵌套作用域}
  \end{center}

  \fig{width = 0.60\textwidth}{figs/nested-symbol-tables}
\end{frame}
%%%%%%%%%%%%%%%%%%%%

%%%%%%%%%%%%%%%%%%%%
\begin{frame}{}
  \begin{columns}
    \column{0.50\textwidth}
      \fig{width = 0.65\textwidth}{figs/vars}
    \column{0.50\textwidth}
      \pause
      \fig{width = 1.00\textwidth}{figs/vars-symtable}
  \end{columns}
\end{frame}
%%%%%%%%%%%%%%%%%%%%

%%%%%%%%%%%%%%%%%%%%
\begin{frame}{}
  \begin{center}
    \fig{width = 0.90\textwidth}{figs/scope-interface}

    \vspace{0.60cm}
    全局作用域、函数/方法作用域、局部作用域
  \end{center}
\end{frame}
%%%%%%%%%%%%%%%%%%%%

%%%%%%%%%%%%%%%%%%%%
\begin{frame}{}
  \fig{width = 0.50\textwidth}{figs/talk-cheap}
\end{frame}
%%%%%%%%%%%%%%%%%%%%

%%%%%%%%%%%%%%%%%%%%
\begin{frame}{}
  \begin{columns}
    \column{0.30\textwidth}
      \fig{width = 0.80\textwidth}{figs/vars}
    \column{0.35\textwidth}
      \fig{width = 1.00\textwidth}{figs/vars-symtable}
    \column{0.35\textwidth}
      \fig{width = 1.00\textwidth}{figs/nested-dot}
  \end{columns}
\end{frame}
%%%%%%%%%%%%%%%%%%%%

%%%%%%%%%%%%%%%%%%%%
\begin{frame}{}
  \fig{width = 1.00\textwidth}{figs/symtable-classes}
\end{frame}
%%%%%%%%%%%%%%%%%%%%

%%%%%%%%%%%%%%%%%%%%
\begin{frame}{}
  \begin{columns}
    \column{0.50\textwidth}
      \fig{width = 1.00\textwidth}{figs/scope-classes}
    \column{0.50\textwidth}
      \fig{width = 0.80\textwidth}{figs/scope-class}
  \end{columns}
\end{frame}
%%%%%%%%%%%%%%%%%%%%

%%%%%%%%%%%%%%%%%%%%
\begin{frame}{}
  \begin{columns}
    \column{0.40\textwidth}
      \fig{width = 1.00\textwidth}{figs/symboltablelistener}
    \column{0.60\textwidth}
      \pause
      \fig{width = 0.80\textwidth}{figs/symboltablelistener-field}
  \end{columns}
\end{frame}
%%%%%%%%%%%%%%%%%%%%

%%%%%%%%%%%%%%%%%%%%
\begin{frame}{}
  \fig{width = 0.70\textwidth}{figs/symboltablelistener-class}
\end{frame}
%%%%%%%%%%%%%%%%%%%%

%%%%%%%%%%%%%%%%%%%%
\begin{frame}{}
  \begin{center}
    \red{\bf \texttt{struct}}: 类型作用域
  \end{center}
  \begin{columns}
    \column{0.40\textwidth}
      \fig{width = 0.90\textwidth}{figs/struct-code}
    \column{0.60\textwidth}
      \fig{width = 1.00\textwidth}{figs/struct-tree}
  \end{columns}
  \begin{center}
    \blue{\bf $d.i$ \qquad $a.b.y$}
  \end{center}
\end{frame}
%%%%%%%%%%%%%%%%%%%%

%%%%%%%%%%%%%%%%%%%%
\begin{frame}{}
  \begin{center}
    \href{https://github.com/antlr/symtab}{symtab @ antlr by parrt}

    \vspace{0.80cm}
    \href{https://github.com/parrt/cs652/tree/master/lectures/code/symtab/src}{symtab @ cs652 by parrt}
  \end{center}
\end{frame}
%%%%%%%%%%%%%%%%%%%%