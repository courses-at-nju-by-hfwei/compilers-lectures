% lexer-handwritten.tex

%%%%%%%%%%%%%%%%%%%%
\begin{frame}{}
  \begin{center}
    \fig{width = 0.80\textwidth}{figs/tokens}

    \vspace{0.60cm}
    \texttt{\teal{DragonLexerGrammar.g4}}
  \end{center}
\end{frame}
%%%%%%%%%%%%%%%%%%%%

%%%%%%%%%%%%%%%%%%%%
\begin{frame}{}
  \begin{center}
    \red{\bf 向前看、向前走、调整状态、记录关键点、侍机回头}
  \end{center}

  \fig{width = 0.80\textwidth}{figs/jiangwen}

  \begin{center}
    妹妹\quad 你大胆地往前走\quad 往前走\quad 莫回呀头
  \end{center}
\end{frame}
%%%%%%%%%%%%%%%%%%%%

%%%%%%%%%%%%%%%%%%%%
\begin{frame}{}
  \begin{center}
    \red{\bf \texttt{nextToken()}}

    \vspace{0.60cm}
    \blue{\bf \texttt{while (nextToken())}}
  \end{center}
\end{frame}
%%%%%%%%%%%%%%%%%%%%

%%%%%%%%%%%%%%%%%%%%
\begin{frame}{}
  \begin{columns}
    \column{0.45\textwidth}
      \begin{center}
        \fig{width = 0.90\textwidth}{figs/ws}
        \ws: 空白符

        \vspace{0.40cm}
        \uncover<3->{\fig{width = 0.70\textwidth}{figs/intnum}
        \intnum: 整数}
      \end{center}
    \column{0.55\textwidth}
      \begin{center}
        \uncover<2->{
        \fig{width = 0.80\textwidth}{figs/id}
        \id: 标识符}

        \vspace{0.40cm}
        \uncover<4->{
        \fig{width = 0.60\textwidth}{figs/error-code}
        错误处理模块}
      \end{center}
  \end{columns}

  \vspace{0.30cm}
  \begin{center}
    \uncover<5->{
    \red{\bf 关键点:} 合并$22, 12, 9$, 根据\purple{\bf 下一个字符}即可判定词法单元的类型 \\[8pt]
    否则, 调用错误处理模块(对应\texttt{other}), 报告\purple{\bf 该字符有误}, 并忽略该字符}
  \end{center}
\end{frame}
%%%%%%%%%%%%%%%%%%%%

%%%%%%%%%%%%%%%%%%%%
\begin{frame}{}
  \begin{center}
    用于识别 \blue{\relop} 的状态转移图
    \fig{width = 0.50\textwidth}{figs/relop}
  \end{center}
\end{frame}
%%%%%%%%%%%%%%%%%%%%

%%%%%%%%%%%%%%%%%%%%
\begin{frame}{}
  \begin{center}
    \red{{\bf 难点:} 如何区分\; \intnum{}、\floatnum{} 与 \scinum?}
  \end{center}
\end{frame}
%%%%%%%%%%%%%%%%%%%%

%%%%%%%%%%%%%%%%%%%%
\begin{frame}{}
  \begin{columns}
    \column{0.50\textwidth}
      \fig{width = 0.70\textwidth}{figs/intnum}
      \begin{center}
        \intnum: 整数
      \end{center}
    \column{0.50\textwidth}
      \pause
      \fig{width = 1.00\textwidth}{figs/realnum}
      \begin{center}
        \floatnum: 浮点数 (无科学计数法)\\
        (不识别 \texttt{2.})
      \end{center}
  \end{columns}

  \pause
  \vspace{1.00cm}
  \fig{width = 0.80\textwidth}{figs/scinum}
  \begin{center}
    \scinum: 带科学计数法的浮点数 \\
    (\texttt{2.99792458E8 \quad 3E8})
  \end{center}
\end{frame}
%%%%%%%%%%%%%%%%%%%%

%%%%%%%%%%%%%%%%%%%%
\begin{frame}{}
  \begin{center}
    \num:
    整数部分\blue{[\texttt{.}可选的小数部分]}\purple{[\texttt{E}[可选的\texttt{+-}]可选的指数部分]}

    \vspace{0.60cm}
    \fig{width = 0.80\textwidth}{figs/number}
    \red{$19, 20, 21:$} 代表了不同的数字类型

    \pause
    \vspace{0.80cm}
    \red{$14, 16, 17:$} 碰到\texttt{other}怎么办?
    \pause
    {\hfill (回退, 寻找\red{\bf 最长匹配})}
  \end{center}
\end{frame}
%%%%%%%%%%%%%%%%%%%%

%%%%%%%%%%%%%%%%%%%%
\begin{frame}{}
  \begin{columns}
    \column{0.50\textwidth}
      \begin{center}
        \fig{width = 0.80\textwidth}{figs/ws}
        \fig{width = 1.00\textwidth}{figs/number}
        \fig{width = 0.80\textwidth}{figs/id}
      \end{center}
    \column{0.50\textwidth}
      \begin{center}
        \fig{width = 0.70\textwidth}{figs/relop}
        \fig{width = 0.60\textwidth}{figs/error-code}
      \end{center}
  \end{columns}

  \pause
  \vspace{0.30cm}
  \begin{center}
    \red{\bf 关键点:} 合并$22, 12, 9, 0$, 根据\purple{\bf 下一个字符}即可判定词法单元的类型 \\[4pt]
    否则, 调用错误处理模块(对应\texttt{other}), 报告\purple{\bf 该字符有误}, 忽略该字符。 \\[4pt]
    注意, 在 \floatnum{} 与 \scinum{}中, 有时需要\purple{\bf 回退}, 寻找最长匹配。
  \end{center}
\end{frame}
%%%%%%%%%%%%%%%%%%%%

%%%%%%%%%%%%%%%%%%%%
\begin{frame}{}
  \begin{center}
    \red{\bf 向前看、向前走、调整状态、记录关键点、侍机回头}
  \end{center}

  \fig{width = 0.80\textwidth}{figs/jiangwen}

  \begin{center}
    妹妹\quad 你大胆地往前走\quad 往前走\quad 莫回呀头
  \end{center}
\end{frame}
%%%%%%%%%%%%%%%%%%%%