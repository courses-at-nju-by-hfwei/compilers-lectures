% overview.tex

%%%%%%%%%%%%%%%%%%%%
\begin{frame}{}
  \fig{width = 0.70\textwidth}{figs/grammar-control}
\end{frame}
%%%%%%%%%%%%%%%%%%%%

%%%%%%%%%%%%%%%%%%%%
\begin{frame}{}
  \fig{width = 0.70\textwidth}{figs/grammar-boolexpr}
\end{frame}
%%%%%%%%%%%%%%%%%%%%

%%%%%%%%%%%%%%%%%%%%
\begin{frame}{}
  \begin{center}
    \fig{width = 0.50\textwidth}{figs/cat-question-mark}
    \begin{center}
      本讲内容颇有难度, 需要多多思考
    \end{center}
  \end{center}
\end{frame}
%%%%%%%%%%%%%%%%%%%%

%%%%%%%%%%%%%%%%%%%%
\begin{frame}{}
  \begin{columns}
    \column{0.50\textwidth}
      \fig{width = 0.80\textwidth}{figs/dragon-book}
    \column{0.50\textwidth}
      \fig{width = 0.85\textwidth}{figs/Engineering-A-Compiler-book}
  \end{columns}
  \begin{center}
    虽然有\blue{\bf 非常简单}的讲法, 教材偏偏采用了让初学者\red{\bf 望而生畏}的讲法。
  \end{center}
\end{frame}
%%%%%%%%%%%%%%%%%%%%

%%%%%%%%%%%%%%%%%%%%
\begin{frame}{}
  \begin{center}
    ``We will overcome all difficulties.'' (我们有困难要上)
    \fig{width = 0.48\textwidth}{figs/difficulties}
    ``If we don't have enough difficulties, we'll create them!'' \\
    (没困难, 我们创造困难还要上)
  \end{center}
\end{frame}
%%%%%%%%%%%%%%%%%%%%

%%%%%%%%%%%%%%%%%%%%
\begin{frame}{}
  \begin{center}
    \red{\bf 关键问题: 为什么要``创造困难''?}

    \pause
    \vspace{1.00cm}
    \blue{\bf 生成更短、更高效的代码}

    \pause
    \vspace{1.00cm}
    \teal{\bf 我们先介绍一种非常简单的做法}
  \end{center}
\end{frame}
%%%%%%%%%%%%%%%%%%%%

%%%%%%%%%%%%%%%%%%%%
\begin{frame}
  \fig{width = 0.80\textwidth}{figs/daju}
  \begin{center}
    心中有``树'' (语法分析树)
  \end{center}
\end{frame}
%%%%%%%%%%%%%%%%%%%%