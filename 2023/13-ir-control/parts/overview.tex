% overview.tex

%%%%%%%%%%%%%%%%%%%%
\begin{frame}{}
  \begin{center}
    \teal{\texttt{Control.g4}}
  \end{center}
  \begin{columns}
    \column{0.50\textwidth}
      \fig{width = 0.45\textwidth}{figs/cfg-S-grammar}
    \column{0.50\textwidth}
      \pause
      \fig{width = 0.35\textwidth}{figs/cfg-boolexpr-grammar}
  \end{columns}
\end{frame}
%%%%%%%%%%%%%%%%%%%%

%%%%%%%%%%%%%%%%%%%%
\begin{frame}{}
  \begin{center}
    \fig{width = 0.50\textwidth}{figs/cat-question-mark}
    \begin{center}
      本讲内容\blue{\bf 非常重要}, 但\red{\bf 颇有难度}, 需要多多思考
    \end{center}
  \end{center}
\end{frame}
%%%%%%%%%%%%%%%%%%%%

%%%%%%%%%%%%%%%%%%%%
\begin{frame}{}
  \begin{columns}
    \column{0.50\textwidth}
      \fig{width = 0.80\textwidth}{figs/dragon-book}
    \column{0.50\textwidth}
      \fig{width = 0.85\textwidth}{figs/Engineering-A-Compiler-book}
  \end{columns}
  \begin{center}
    虽然有\blue{\bf 多种}讲法, 教材偏偏采用了让初学者\red{\bf 望而生畏}的那一种。
  \end{center}
\end{frame}
%%%%%%%%%%%%%%%%%%%%

%%%%%%%%%%%%%%%%%%%%
\begin{frame}{}
  \begin{center}
    ``We will overcome all difficulties.'' (我们有困难要上)
    \fig{width = 0.48\textwidth}{figs/difficulties}
    ``If we don't have enough difficulties, we'll create them!'' \\
    (没困难, 我们创造困难还要上)
  \end{center}
\end{frame}
%%%%%%%%%%%%%%%%%%%%

%%%%%%%%%%%%%%%%%%%%
\begin{frame}{}
  \begin{center}
    \red{\bf 关键问题: 为什么要``创造困难''?}

    \pause
    \vspace{1.00cm}
    \blue{\bf 生成更短、更高效的代码}
  \end{center}
\end{frame}
%%%%%%%%%%%%%%%%%%%%