% array.tex

%%%%%%%%%%%%%%%%%%%%
\begin{frame}{}
  \begin{center}
    \red{\bf 数组引用的中间代码翻译}

    \begin{align*}
      \text{声明}&: \teal{\text{int}\; a[2][3]} \\[12pt]
      \text{数组引用}&: \teal{x = a[1][2];\; a[1][2] = x}
    \end{align*}

    \pause
    \vspace{0.50cm}
    需要计算 $a[1][2]$ 相对于\blue{\bf 数组基地址} $a$ 的\purple{\bf 偏移地址}

    \[
      addr(a[1][2]) = \blue{base} + \purple{1 \times 12 + 2 \times 4}
    \]

    % array-types.tex

\begin{table}
  \centering
  \resizebox{0.60\textwidth}{!}{
  \renewcommand{\arraystretch}{1.3}
  \begin{tabular}{|c||c|c|}
    \hline
    & 类型 & 宽度
    \\ \hline \hline
    $a$ & $array(2, array(3, \text{integer}))$ & 24 \\ \hline
    $a[i]$ & $array(3, \text{integer})$ & 12 \\ \hline
    $a[i][j]$ & $\text{integer}$ & 4 \\ \hline
  \end{tabular}}
\end{table}
  \end{center}
\end{frame}
%%%%%%%%%%%%%%%%%%%%

%%%%%%%%%%%%%%%%%%%%
\begin{frame}{}
  \begin{center}
    \[
      \teal{\text{int}\; a[2][3]}
    \]
    \vspace{-0.50cm}
    \fig{width = 0.85\textwidth}{figs/anno-type-width}
  \end{center}
\end{frame}
%%%%%%%%%%%%%%%%%%%%

%%%%%%%%%%%%%%%%%%%%
\begin{frame}{}
  \begin{center}
    \green{\bf 综合属性 $L.array(.base):$ 数组基地址 (即, 数组名)}

    \fig{width = 0.65\textwidth}{figs/cfg-array}

    \vspace{-0.10cm}
    \purple{\bf 综合属性 $L.addr:$ 偏移地址}
  \end{center}
\end{frame}
%%%%%%%%%%%%%%%%%%%%

%%%%%%%%%%%%%%%%%%%%
% \begin{frame}{}
%   \begin{center}
%     \red{\bf 综合属性 $L.array:$ 数组名 \id 对应的符号表条目}

%     \fig{width = 0.85\textwidth}{figs/SDT-L-L-array}
%   \end{center}
% \end{frame}
%%%%%%%%%%%%%%%%%%%%

%%%%%%%%%%%%%%%%%%%%
% \begin{frame}{}
%   \begin{center}
%     \purple{\bf 综合属性 $L.type:$ (当前)元素类型}

%     \fig{width = 0.85\textwidth}{figs/SDT-L-L-type}
%   \end{center}
% \end{frame}
%%%%%%%%%%%%%%%%%%%%

%%%%%%%%%%%%%%%%%%%%
% \begin{frame}{}
%   \begin{center}
%     \blue{\bf 综合属性 $L.addr:$ (当前)偏移地址}

%     \fig{width = 0.85\textwidth}{figs/SDT-L-L-addr}
%   \end{center}
% \end{frame}
%%%%%%%%%%%%%%%%%%%%

%%%%%%%%%%%%%%%%%%%%
\begin{frame}{}
  \begin{center}
    \fig{width = 0.90\textwidth}{figs/array-code}
    \vspace{-0.65cm}
    \[
      \teal{c + a[i][j]}
    \]
  \end{center}
\end{frame}
%%%%%%%%%%%%%%%%%%%%

%%%%%%%%%%%%%%%%%%%%
\begin{frame}{}
  \uncover<2->{\fig{width = 0.50\textwidth}{figs/array-c-type}}
  \vspace{-0.20cm}
  \fig{width = 0.25\textwidth}{figs/array-c}
  \vspace{-0.20cm}
  \uncover<3->{\fig{width = 0.85\textwidth}{figs/array-c-ll}}
\end{frame}
%%%%%%%%%%%%%%%%%%%%

%%%%%%%%%%%%%%%%%%%%
\begin{frame}{}
  \begin{center}
    GEP provides a way to \red{\bf access arrays and manipulate pointers}.

    \vspace{1.00cm}
    GEP abstract away details like \blue{\bf size of types}.
  \end{center}
\end{frame}
%%%%%%%%%%%%%%%%%%%%

%%%%%%%%%%%%%%%%%%%%
\begin{frame}{}
  \begin{center}
    \texttt{getelementptr \\[5pt] \red{<base-type>}, <base-type>* \blue{<base-addr>}, \teal{[i32 <index>]+}}
  \end{center}

  \pause
  \vspace{0.80cm}
  \red{\texttt{<base-type>}}: base type used for \purple{\bf the first} index
  \begin{itemize}
    \setlength{\itemsep}{5pt}
    \item It does \emph{not} change the pointer type.
    \item It offsets by the \red{\texttt{<base-type>}}.
  \end{itemize}

  \pause
  \vspace{0.80cm}
  \teal{Further indices}:
  \begin{itemize}
    \setlength{\itemsep}{5pt}
    \item Offset \teal{inside} arrays (aggregate types)
    \item \emph{Change} the pointer type by removing one layer of ``aggregation''
  \end{itemize}
\end{frame}
%%%%%%%%%%%%%%%%%%%%

%%%%%%%%%%%%%%%%%%%%
\begin{frame}{}
  \begin{center}
    \href{https://llvm.org/docs/GetElementPtr.html}{The Often Misunderstood GEP Instruction @ LLVM Docs}
  \end{center}

  \pause
  \vspace{0.30cm}
  \begin{center}
    \fig{width = 0.80\textwidth}{figs/llvm-ir-tutorial}

    \vspace{0.30cm}
    \href{https://www.bilibili.com/video/BV1oE411y711/?share_source=copy_web&vd_source=afddc1f6e07c3046ed07519aa34370fd}{LLVM IR Tutorial: Phis, GEPs and Other Things, Oh My! @ bilibili}
  \end{center}
\end{frame}
%%%%%%%%%%%%%%%%%%%%