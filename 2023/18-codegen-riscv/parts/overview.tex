% overview.tex

%%%%%%%%%%%%%%%%%%%%
% \begin{frame}{}
%   \fig{width = 0.50\textwidth}{figs/OPPO-MariSilicon-Website}
% \end{frame}
%%%%%%%%%%%%%%%%%%%%

%%%%%%%%%%%%%%%%%%%%
\begin{frame}{}
  \begin{center}
    \blue{RISC:} Reduced Instruction Set Computer (精简指令集计算机)

    \vspace{0.80cm}
    \fig{width = 0.60\textwidth}{figs/riscv-logo}
    \vspace{0.80cm}

    \blue{V:} Five
  \end{center}
\end{frame}
%%%%%%%%%%%%%%%%%%%%

%%%%%%%%%%%%%%%%%%%%
\begin{frame}{}
  \begin{center}
    \blue{CISC:} Complex Instruction Set Computer (复杂指令集计算机)

    \vspace{0.30cm}
    \fig{width = 0.50\textwidth}{figs/cisc-risc}
    \vspace{0.30cm}

    \red{RISC: Each instruction performs only one function.}
  \end{center}
\end{frame}
%%%%%%%%%%%%%%%%%%%%

%%%%%%%%%%%%%%%%%%%%
\begin{frame}{}
  \begin{center}
    \blue{CISC:} \texttt{SUBL val, \%eax ; \%eax <- \%eax - val)}

    \vspace{0.30cm}
    \fig{width = 0.50\textwidth}{figs/cisc-risc}
    \vspace{0.30cm}
  \end{center}
\end{frame}
%%%%%%%%%%%%%%%%%%%%

%%%%%%%%%%%%%%%%%%%%
\begin{frame}{}
  \begin{center}
    ISA (Instruction Set Architecture) as the Software/Hardware Interface
    \fig{width = 0.70\textwidth}{figs/isa-hardware-software}
  \end{center}
\end{frame}
%%%%%%%%%%%%%%%%%%%%

%%%%%%%%%%%%%%%%%%%%
% \begin{frame}{}
%   \begin{columns}
%     \column{0.50\textwidth}
%       \fig{width = 0.80\textwidth}{figs/UC-Berkeley}
%     \column{0.50\textwidth}
%       \fig{width = 0.80\textwidth}{figs/David-Patterson}
%       \begin{center}
%         David Andrew Patterson ($1947 \sim$)
%       \end{center}
%   \end{columns}
% \end{frame}
%%%%%%%%%%%%%%%%%%%%

%%%%%%%%%%%%%%%%%%%%
\begin{frame}{}
  \begin{center}
    2017 ACM Turing Award
  \end{center}
  \fig{width = 0.60\textwidth}{figs/Patterson-Hennessy}

  \vspace{0.30cm}
  \begin{center}
    % ``for pioneering a systematic, quantitative approach
    % to the design and evaluation of computer architectures
    % with enduring impact on the microprocessor industry. \\[8pt]
    ``Hennessy and Patterson created a systematic and quantitative approach
    to designing faster, lower power,
    and \red{\bf reduced instruction set computer (RISC)} microprocessors.''
  \end{center}
\end{frame}
%%%%%%%%%%%%%%%%%%%%

%%%%%%%%%%%%%%%%%%%%
\begin{frame}{}
  \begin{columns}
    \column{0.50\textwidth}
      \fig{width = 0.90\textwidth}{figs/computer-organization-book}
    \column{0.50\textwidth}
      \fig{width = 0.90\textwidth}{figs/computer-architecture-book}
  \end{columns}
\end{frame}
%%%%%%%%%%%%%%%%%%%%

%%%%%%%%%%%%%%%%%%%%
\begin{frame}{}
  \fig{width = 1.00\textwidth}{figs/riscv-org}
\end{frame}
%%%%%%%%%%%%%%%%%%%%

%%%%%%%%%%%%%%%%%%%%
\begin{frame}{}
  \fig{width = 0.70\textwidth}{figs/riscv-members}
\end{frame}
%%%%%%%%%%%%%%%%%%%%

%%%%%%%%%%%%%%%%%%%%
% \begin{frame}{}
%   \fig{width = 0.50\textwidth}{figs/sifive-website}
% \end{frame}
%%%%%%%%%%%%%%%%%%%%

%%%%%%%%%%%%%%%%%%%%
\begin{frame}{}
  \begin{center}
    \href{https://www.benchcouncil.org/conferences/ficc/2019/chips19/chips19.html}{2019年国际芯片大会}
  \end{center}
  \begin{columns}
    \column{0.50\textwidth}
      \fig{width = 0.90\textwidth}{figs/2019-chip-conf}
    \column{0.50\textwidth}
      \fig{width = 0.90\textwidth}{figs/niguangnan}
  \end{columns}

  \vspace{0.60cm}
  \begin{center}
    ``RISC-V 很可能发展成为世界主流 CPU 之一,
    在 CPU 领域形成 Intel、\\[5pt]
    ARM、RISC-V 三分天下的格局。''
    \hfill --- 中国工程院院士\quad 倪光南
  \end{center}
\end{frame}
%%%%%%%%%%%%%%%%%%%%