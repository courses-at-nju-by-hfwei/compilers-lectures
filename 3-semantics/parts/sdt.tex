% sdt.tex

%%%%%%%%%%%%%%%%%%%%
\begin{frame}{}
  \begin{columns}[b]
    \column{0.25\textwidth}
      \fig{width = 0.80\textwidth}{figs/1}
      \vspace{-0.50cm}
      \begin{center}
        \blue{一对概念}
      \end{center}
    \column{0.25\textwidth}
      \fig{width = 0.85\textwidth}{figs/2}
      \vspace{-0.60cm}
      \begin{center}
        \blue{两类属性定义}
      \end{center}
    \column{0.25\textwidth}
      \fig{width = 0.65\textwidth}{figs/3}
      \begin{center}
        \blue{三种实现方式}
      \end{center}
    \column{0.25\textwidth}
      \fig{width = 0.65\textwidth}{figs/4}
      \begin{center}
        \blue{四大应用}
      \end{center}
  \end{columns}
\end{frame}
%%%%%%%%%%%%%%%%%%%%

%%%%%%%%%%%%%%%%%%%%
\begin{frame}{}
  \begin{definition}[语法制导的翻译方案 (Syntax-Directed Translation Scheme; SDT)]
    SDT 是在其产生式体中嵌入\red{\bf 语义动作}的上下文无关文法。
  \end{definition}

  \vspace{0.30cm}
  \begin{center}
    \blue{\bf 语义动作可以嵌入在产生式体中的任何地方}
  \end{center}

  \begin{columns}
    \column{0.50\textwidth}
      \fig{width = 1.00\textwidth}{figs/SDD-expr-left-recursion}
    \column{0.50\textwidth}
      \fig{width = 1.00\textwidth}{figs/SDT-expr-left-recursion}
  \end{columns}
\end{frame}
%%%%%%%%%%%%%%%%%%%%

%%%%%%%%%%%%%%%%%%%%
\begin{frame}{}
  \begin{center}
    时机 (Timing; タイミング)

    \fig{width = 0.60\textwidth}{figs/timing}
  \end{center}
\end{frame}
%%%%%%%%%%%%%%%%%%%%

%%%%%%%%%%%%%%%%%%%%
\begin{frame}{}
  \begin{center}
    \red{$Q:$} 如何将 SDD 中的\blue{\bf 语义规则}转换为带有\blue{\bf 语义动作}的 SDT

    \vspace{0.60cm}
    % sdt.tex

\begin{table}
  \centering
  \resizebox{0.60\textwidth}{!}{
  \renewcommand{\arraystretch}{1.3}
  \begin{tabular}{|c||c|c|}
    \hline
    & $S$ 属性定义 & $L$ 属性定义
    \\ \hline \hline
    Offline & & \\ \hline
    $LR$ & & \\ \hline
    $LL$ & & \\ \hline
  \end{tabular}}
\end{table}

    \vspace{0.60cm}
    \red{$Q:$} 如何以三种方式实现 SDT?
  \end{center}
\end{frame}
%%%%%%%%%%%%%%%%%%%%

%%%%%%%%%%%%%%%%%%%%
\begin{frame}{}
  \begin{center}
    \fig{width = 0.50\textwidth}{figs/dfs}

    \vspace{0.80cm}
    按照\red{\bf 从左到右}的\red{\bf 深度优先}顺序遍历语法分析树
  \end{center}
\end{frame}
%%%%%%%%%%%%%%%%%%%%

%%%%%%%%%%%%%%%%%%%%
\begin{frame}{}
  \begin{center}
    \fig{width = 0.80\textwidth}{figs/SDT-expr-prefix}
    \vspace{-0.50cm}
    \[
      3 \ast 5 + 4 \implies \pause + \ast 3 5 4
    \]
  \end{center}
\end{frame}
%%%%%%%%%%%%%%%%%%%%

%%%%%%%%%%%%%%%%%%%%
\begin{frame}{}
  \begin{center}
    嵌入语义动作\red{\bf 虚拟节点}的语法分析树

    \fig{width = 0.80\textwidth}{figs/offline-expr-prefix}
    \[
      3 \ast 5 + 4 \implies + \ast 3 5 4
    \]
  \end{center}
\end{frame}
%%%%%%%%%%%%%%%%%%%%

%%%%%%%%%%%%%%%%%%%%
\begin{frame}{}
  \input{tables/sdt-offline}
\end{frame}
%%%%%%%%%%%%%%%%%%%%

%%%%%%%%%%%%%%%%%%%%
\begin{frame}{}
  \begin{center}
    \red{$Q:$ 是否所有的 SDT 都可以在 $LL/LR$ 语法分析过程中实现?}
  \end{center}
\end{frame}
%%%%%%%%%%%%%%%%%%%%

%%%%%%%%%%%%%%%%%%%%
\begin{frame}{}
  \begin{center}
    \red{$Q:$ 如何判断某 SDT 是否可以在 $LL/LR$ 语法分析过程中实现?}
  \end{center}
\end{frame}
%%%%%%%%%%%%%%%%%%%%

%%%%%%%%%%%%%%%%%%%%
\begin{frame}{}
\end{frame}
%%%%%%%%%%%%%%%%%%%%

%%%%%%%%%%%%%%%%%%%%
\begin{frame}{}
\end{frame}
%%%%%%%%%%%%%%%%%%%%

%%%%%%%%%%%%%%%%%%%%
\begin{frame}{}
\end{frame}
%%%%%%%%%%%%%%%%%%%%

%%%%%%%%%%%%%%%%%%%%
\begin{frame}{}
\end{frame}
%%%%%%%%%%%%%%%%%%%%

%%%%%%%%%%%%%%%%%%%%
\begin{frame}{}
\end{frame}
%%%%%%%%%%%%%%%%%%%%

%%%%%%%%%%%%%%%%%%%%
\begin{frame}{}
\end{frame}
%%%%%%%%%%%%%%%%%%%%