% procedure.tex

%%%%%%%%%%%%%%%%%%%%
\begin{frame}{}
  \setcounter{equation}{7}
  \begin{columns}
    \column{0.50\textwidth}
      \begin{align}
        \param\; x \\
        \call\; p, n \\
        y = \call\; p, n \\
        \return\; y
      \end{align}
    \column{0.50\textwidth}
      \fig{width = 0.40\textwidth}{figs/param-call}
      \[
        \teal{p(x_{1}, x_{2}, \dots, x_{n})}
      \]
  \end{columns}
\end{frame}
%%%%%%%%%%%%%%%%%%%%

%%%%%%%%%%%%%%%%%%%%
% \begin{frame}{}
%   \begin{align*}
%     \param\; u \\
%     \param\; v \\
%     \param\; w \\
%     \call\; f \\
%     \call\; g
%   \end{align*}

%   \pause
%   \[
%     g(u, v, f(w)) \qquad g(u, f(v, w))
%   \]
% \end{frame}
%%%%%%%%%%%%%%%%%%%%

%%%%%%%%%%%%%%%%%%%%
\begin{frame}{}
  \begin{center}
    \red{\bf 函数/过程的中间代码翻译}

    \vspace{0.80cm}
    \teal{\texttt{\Large n = f(a[i])}}

    \fig{width = 0.50\textwidth}{figs/call-tac}
  \end{center}
\end{frame}
%%%%%%%%%%%%%%%%%%%%

%%%%%%%%%%%%%%%%%%%%
\begin{frame}{}
  \begin{center}
    \teal{\bf 新增文法以支持函数\red{定义}与\blue{调用}}
    \fig{width = 0.80\textwidth}{figs/cfg-procedure}
  \end{center}
\end{frame}
%%%%%%%%%%%%%%%%%%%%

%%%%%%%%%%%%%%%%%%%%
\begin{frame}{}
  \begin{center}
    \red{\bf 函数定义}

    \vspace{0.80cm}
    \fig{width = 0.60\textwidth}{figs/procedure-declare}

    \vspace{0.60cm}
    \blue{\bf 函数名 \id{} 放入当前符号表, \red{建立新的符号表}, 处理形参 $F$ 与函数体 $S$}
  \end{center}
\end{frame}
%%%%%%%%%%%%%%%%%%%%

%%%%%%%%%%%%%%%%%%%%
\begin{frame}{}
  \begin{center}
    \blue{\bf 函数调用}

    \begin{columns}
      \column{0.50\textwidth}
        \fig{width = 1.00\textwidth}{figs/procedure-call}
      \column{0.50\textwidth}
        \fig{width = 0.60\textwidth}{figs/param-call}
    \end{columns}
  \end{center}
\end{frame}
%%%%%%%%%%%%%%%%%%%%

%%%%%%%%%%%%%%%%%%%%
\begin{frame}{}
  \begin{center}
    \blue{\bf 函数调用}

    \fig{width = 1.00\textwidth}{figs/call-sdt}

    C 语言并未规定参数计算的顺序
  \end{center}
\end{frame}
%%%%%%%%%%%%%%%%%%%%

%%%%%%%%%%%%%%%%%%%%
\begin{frame}{}
  \[
    g(u, v, f(w)) \qquad g(u, f(v, w))
  \]
\end{frame}
%%%%%%%%%%%%%%%%%%%%

%%%%%%%%%%%%%%%%%%%%
\begin{frame}{}
  \begin{columns}
    \column{0.50\textwidth}
      \begin{align*}
        \text{计算实参 $x_{1}$ 的中间代码} \\
        \param\; x_{1} \\
        \text{计算实参 $x_{2}$ 的中间代码} \\
        \param\; x_{2} \\
        \ldots \\
        \text{计算实参 $x_{m}$ 的中间代码} \\
        \param\; x_{n} \\
        \call\; p, n
      \end{align*}
    \column{0.50\textwidth}
      \pause
      \begin{align*}
        \text{计算实参 $x_{1}$ 的中间代码} \\
        \text{计算实参 $x_{2}$ 的中间代码} \\
        \ldots \\
        \text{计算实参 $x_{m}$ 的中间代码} \\
        \param\; x_{1} \\
        \param\; x_{2} \\
        \ldots \\
        \param\; x_{n} \\
        \call\; p, n
      \end{align*}
  \end{columns}
\end{frame}
%%%%%%%%%%%%%%%%%%%%

%%%%%%%%%%%%%%%%%%%%
\begin{frame}{}
  \begin{center}
    \blue{\bf 函数调用}

    \fig{width = 0.70\textwidth}{figs/call-sdt-queue}
    \teal{集中生成 \param{} 指令, 代码更紧凑}
  \end{center}
\end{frame}
%%%%%%%%%%%%%%%%%%%%