% sdt.tex

%%%%%%%%%%%%%%%%%%%%
\begin{frame}{}
  \begin{definition}[语法制导的翻译方案 (Syntax-Directed Translation Scheme; SDT)]
    \purple{\bf SDT} 是在其产生式体中嵌入\red{\bf 语义动作}的上下文无关文法。
  \end{definition}

  \vspace{0.30cm}
  \begin{columns}
    \column{0.50\textwidth}
      \fig{width = 1.00\textwidth}{figs/SDD-expr-left-recursion-rule}
    \column{0.50\textwidth}
      \fig{width = 1.00\textwidth}{figs/SDT-expr-left-recursion}
  \end{columns}
\end{frame}
%%%%%%%%%%%%%%%%%%%%

%%%%%%%%%%%%%%%%%%%%
\begin{frame}{}
  \begin{center}
    \blue{\bf 语义动作可以嵌入在产生式体中的任何位置}

    \fig{width = 0.80\textwidth}{figs/SDT-expr-prefix}
    \teal{前缀表达式 SDT}

  \end{center}
\end{frame}
%%%%%%%%%%%%%%%%%%%%

%%%%%%%%%%%%%%%%%%%%
\begin{frame}{}
  \[
    B \to \blue{X} \red{\set{a}} Y
  \]

  \vspace{0.30cm}
  \begin{center}
    \pause
    \red{$Q$: 如何在{\bf 语法分析过程中}\violet{\bf 自动}实现属性文法?}

    \pause
    \vspace{1.00cm}
    语义动作嵌入的位置决定了\red{\bf 何时}执行该动作

    \vspace{0.60cm}
    \red{\bf 基本思想:} 一个动作在它\blue{\bf 左边的}所有文法符号都\blue{\bf 处理}过之后立刻执行
  \end{center}
\end{frame}
%%%%%%%%%%%%%%%%%%%%

%%%%%%%%%%%%%%%%%%%%
\begin{frame}{}
  \[
    B \to \blue{X} \red{\set{a}} Y
  \]

  \vspace{0.30cm}
  \red{\bf 基本思想:} 一个动作在它\blue{\bf 左边的}所有文法符号都\blue{\bf 处理}过之后立刻执行

  \pause
  \vspace{0.80cm}
  \begin{description}
    \setlength{\itemsep}{20pt}
    \item[自底向上:] 移入 $X$ 或归约为 $X$ \red{之后} (即, $X$ 位于栈顶) 执行动作 $a$
    \pause
    \item[自顶向下:] 对 $Y$ 进行展开或者匹配\red{之前}执行动作 $a$
  \end{description}
\end{frame}
%%%%%%%%%%%%%%%%%%%%

%%%%%%%%%%%%%%%%%%%%
\begin{frame}{}
  \begin{center}
    \blue{\bf 时机} (Timing; タイミング)

    \fig{width = 0.60\textwidth}{figs/timing-japanese}

    \vspace{0.30cm}
    \red{\bf 语义动作嵌入在什么地方? 这决定了何时执行语义动作。}
  \end{center}
\end{frame}
%%%%%%%%%%%%%%%%%%%%

%%%%%%%%%%%%%%%%%%%%
\begin{frame}{}
  \begin{center}
    \red{$Q:$} 如何将带有\blue{\bf 语义规则}的 SDD 转换为带有\blue{\bf 语义动作}的 SDT

    \vspace{0.60cm}
    % sdt.tex

\begin{table}
  \centering
  \resizebox{0.60\textwidth}{!}{
  \renewcommand{\arraystretch}{1.3}
  \begin{tabular}{|c||c|c|}
    \hline
    & $S$ 属性定义 & $L$ 属性定义
    \\ \hline \hline
    Offline & & \\ \hline
    $LR$ & & \\ \hline
    $LL$ & & \\ \hline
  \end{tabular}}
\end{table}

    \vspace{0.60cm}
    \red{$Q:$} 如何以\purple{\bf 三种方式}实现 SDT?
  \end{center}
\end{frame}
%%%%%%%%%%%%%%%%%%%%

%%%%%%%%%%%%%%%%%%%%
\begin{frame}{}
  \begin{center}
    \blue{\bf Offline 方式:} 已有语法分析树

    \vspace{0.50cm}
    \fig{width = 0.50\textwidth}{figs/dfs}

    \vspace{0.50cm}
    按照\red{\bf 从左到右}的\red{\bf 深度优先}顺序遍历语法分析树

    \pause
    \vspace{0.50cm}
    \red{\bf 基本思想:} 一个动作在它\blue{\bf 左边的}所有文法符号都\blue{\bf 处理}过之后立刻执行
  \end{center}
\end{frame}
%%%%%%%%%%%%%%%%%%%%

%%%%%%%%%%%%%%%%%%%%
\begin{frame}{}
  \begin{center}
    \fig{width = 0.80\textwidth}{figs/SDT-expr-prefix}
  \end{center}
\end{frame}
%%%%%%%%%%%%%%%%%%%%

%%%%%%%%%%%%%%%%%%%%
\begin{frame}{}
  \begin{center}
    嵌入语义动作\red{\bf 虚拟节点}的语法分析树

    \fig{width = 0.80\textwidth}{figs/offline-expr-prefix}
    \[
      3 \ast 5 + 4 \implies + \ast 3 5 4
    \]
  \end{center}
\end{frame}
%%%%%%%%%%%%%%%%%%%%

%%%%%%%%%%%%%%%%%%%%
\begin{frame}{}
  \input{tables/sdt-offline}

  \pause
  \vspace{0.60cm}
  \begin{center}
    \blue{ANTLR4} 使用该方法, 将文法与语义分开, 易于开发, 易于维护。
  \end{center}
\end{frame}
%%%%%%%%%%%%%%%%%%%%

%%%%%%%%%%%%%%%%%%%%
\begin{frame}{}
  \begin{center}
    \red{\bf 基本思想:} 一个动作在它\blue{\bf 左边的}所有文法符号都\blue{\bf 处理}过之后立刻执行

    \pause
    \vspace{0.30cm}
    \fig{width = 0.50\textwidth}{figs/cat-question}

    \vspace{0.60cm}
    \red{\bf $Q:$ 是否所有的 SDT 都可以在 \blue{\bf $LL/LR$ 语法分析过程中}实现?}
  \end{center}
\end{frame}
%%%%%%%%%%%%%%%%%%%%

%%%%%%%%%%%%%%%%%%%%
\begin{frame}{}
  \begin{center}
    该 SDT \red{\bf 无法}在 \blue{$\gray{LL(1)}/LR(1)$} 中实现

    \fig{width = 0.80\textwidth}{figs/SDT-expr-prefix}

    \pause
    \vspace{0.50cm}
    它需要在还不知道出现在输入中的运算符是 $\ast$ 还是 $+$时,

    \vspace{0.30cm}
    就执行打印这些运算符的操作
  \end{center}
\end{frame}
%%%%%%%%%%%%%%%%%%%%

%%%%%%%%%%%%%%%%%%%%
\begin{frame}{}
  \begin{center}
    \red{\bf $Q:$ 如何判断某 SDT 是否可以在 \blue{\bf $LL/LR$ 语法分析过程中}实现?}

    \pause
    \vspace{1.20cm}
    将每个内嵌的语义动作 $A$ 替换为一个独有的\purple{\bf 非终结符} $M$

    \vspace{0.60cm}
    添加新产生式 \purple{$M \to \epsilon$}

    \vspace{0.60cm}
    判断新产生的文法是否可用 $LL/LR$ 进行分析
  \end{center}
\end{frame}
%%%%%%%%%%%%%%%%%%%%

%%%%%%%%%%%%%%%%%%%%
\begin{frame}{}
  \begin{center}
    \fig{width = 0.80\textwidth}{figs/SDT-expr-prefix-marker}
    \begin{align*}
      M_{2} &\to \epsilon \\
      M_{4} &\to \epsilon \\
      M_{7} &\to \epsilon
    \end{align*}
  \end{center}
\end{frame}
%%%%%%%%%%%%%%%%%%%%

%%%%%%%%%%%%%%%%%%%%
\begin{frame}{}
  \begin{center}
    \uncover<2->{
      \vspace{-0.50cm}
      \[
        [A \to \alpha \cdot B \beta, a] \in I \implies
          \forall \red{b \in \first(\beta a)}.\; [B \to \cdot \gamma, b] \in I
      \]
      \vspace{-0.50cm}
    }
    \begin{columns}
      \column{0.50\textwidth}
        \fig{width = 1.00\textwidth}{figs/SDT-expr-prefix-marker}
        \begin{align*}
          M_{2} &\to \epsilon \\
          M_{4} &\to \epsilon \\
          M_{7} &\to \epsilon
        \end{align*}
      \column{0.50\textwidth}
        \uncover<3->{
        \begin{align*}
          L &\to \cdot E \;\n, \quad \$ \\[5pt]
          E &\to \cdot \purple{M_{2}}\; E + T, \quad \n \\[5pt]
          E &\to \cdot T, \quad \n \\[5pt]
          \purple{M_{2}} &\to \blue{\cdot, \quad (/\digit} \\[5pt]
          T &\to \cdot \purple{M_{4}}\; T \ast F, \quad \n \\[5pt]
          T &\to \cdot F, \quad \n \\[5pt]
          \purple{M_{4}} &\to \blue{\cdot, \quad (/\digit} \\[5pt]
          F &\to \cdot (E), \quad \n \\[5pt]
          F &\to \blue{\cdot \digit\; M_{7}, \quad \n}
        \end{align*}

        \begin{center}
          遇到 \digit, 产生移入/归约冲突
        \end{center}}
    \end{columns}
  \end{center}
\end{frame}
%%%%%%%%%%%%%%%%%%%%

%%%%%%%%%%%%%%%%%%%%
\begin{frame}{}
  \begin{center}
    \blue{哪些 SDT 可以在 $LL/LR$ 中实现? 如何实现?}
  \end{center}
  % sdt-online.tex

\begin{table}
  \centering
  \resizebox{0.60\textwidth}{!}{
  \renewcommand{\arraystretch}{1.5}
  \begin{tabular}{|c||c|c|}
    \hline
    & $S$ 属性定义 & $L$ 属性定义
    \\ \hline \hline
    Offline & \multicolumn{2}{c|}{嵌入语义动作\red{\bf 虚拟节点}} \\ \hline
    $LR$ & \cmark & \\ \hline
    $LL$ & & \cmark \\ \hline
  \end{tabular}}
\end{table}
\end{frame}
%%%%%%%%%%%%%%%%%%%%

%%%%%%%%%%%%%%%%%%%%
\begin{frame}{}
  \begin{center}
    \begin{columns}
      \column{0.50\textwidth}
        \begin{center}
          \blue{\bf $S$ 属性定义}
        \end{center}
        \fig{width = 1.00\textwidth}{figs/SDD-expr-left-recursion-rule}
      \column{0.50\textwidth}
        \begin{center}
          \red{\bf 后缀翻译方案}
        \end{center}
        \fig{width = 1.00\textwidth}{figs/SDT-expr-left-recursion}
    \end{columns}

    \pause
    \vspace{0.80cm}
    \red{\bf 后缀翻译方案:} 所有动作都在产生式的最后

    \vspace{0.30cm}
    在 $LR$ 中, 按某个产生式\blue{\bf 归约}时, 执行相应动作
  \end{center}
\end{frame}
%%%%%%%%%%%%%%%%%%%%

%%%%%%%%%%%%%%%%%%%%
\begin{frame}{}
  \begin{center}
    \[
      \blue{A \to X Y Z}
    \]

    \fig{width = 0.80\textwidth}{figs/S-SDD-LR}

    \red{\bf 移入}时, 携带终结符的属性

    \vspace{0.30cm}
    \red{\bf 归约}时, 计算 $A$ 的属性值并入栈
  \end{center}
\end{frame}
%%%%%%%%%%%%%%%%%%%%

%%%%%%%%%%%%%%%%%%%%
\begin{frame}{}
  % sdt-S-SDD-LR.tex

\begin{table}
  \centering
  \resizebox{0.60\textwidth}{!}{
  \renewcommand{\arraystretch}{1.5}
  \begin{tabular}{|c||c|c|}
    \hline
    & $S$ 属性定义 & $L$ 属性定义
    \\ \hline \hline
    Offline & \multicolumn{2}{c|}{嵌入语义动作\red{\bf 虚拟节点}} \\ \hline
    $LR$ & \red{\bf 后缀}翻译方案 & \\ \hline
    $LL$ & & \cmark \\ \hline
  \end{tabular}}
\end{table}
\end{frame}
%%%%%%%%%%%%%%%%%%%%

%%%%%%%%%%%%%%%%%%%%
\begin{frame}{}
  \begin{center}
    \red{\bf $L$ 属性定义} 与 \blue{\bf $LL$ 语法分析}

    \vspace{0.30cm}
    \fig{width = 0.80\textwidth}{figs/dep-expr-no-left-recursion}
    \vspace{-0.20cm}
    \[
      \teal{3 \ast 5}
    \]

    \pause
    \vspace{-0.50cm}
    \[
      \blue{A \to X_{1} \cdots X_{i} \cdots X_{n}}
    \]

    \red{\bf 原则:} \teal{\bf 从左到右}处理各个 $X_{i}$ 符号

    \vspace{0.10cm}
    对每个 $X_{i}$, 先计算\teal{\bf 继承属性}, 后计算\teal{\bf 综合属性}
  \end{center}
\end{frame}
%%%%%%%%%%%%%%%%%%%%

%%%%%%%%%%%%%%%%%%%%
\begin{frame}{}
  \begin{center}
    \red{\bf 递归下降子过程 $A \to X_{1} \cdots X_{i} \cdots X_{n}$}

    \vspace{0.80cm}
    \begin{itemize}
      \centering
      \setlength{\itemsep}{15pt}
      \item 在调用 $X_{i}$ 子过程之前, 计算 $X_{i}$ 的\red{\bf 继承属性}
      \item 以 $X_{i}$ 的继承属性为\blue{\bf 参数}调用 $X_{i}$ 子过程
      \item 在 $X_{i}$ 子过程返回之前, 计算 $X_{i}$ 的\red{\bf 综合属性}
      \item 在 $X_{i}$ 子过程中\blue{\bf 返回} $X_{i}$ 的综合属性
    \end{itemize}
  \end{center}
\end{frame}
%%%%%%%%%%%%%%%%%%%%

%%%%%%%%%%%%%%%%%%%%
\begin{frame}{}
  \begin{center}
    (左递归) $S$属性定义
    \vspace{-0.50cm}
    % left-recursion-S-SDD.tex

\begin{alignat*}{3}
  A &\to A_{1} Y \quad && A.a = g(A_{1}.a, Y.y) \\[8pt]
  A &\to X && A.a = f(X.x)
\end{alignat*}

    \vspace{-0.60cm}
    \[
      \teal{\boxed{X Y^{\ast}}}
    \]

    (右递归) $L$属性定义
    \vspace{-0.50cm}
    % right-recursion-L-SDD.tex

\begin{alignat*}{3}
  A &\to XR \quad && \red{R.i} = f(X.x);\; A.a = R.s \\[8pt]
  R &\to YR_{1} && \red{R_{1}.i} = g(R.i, Y.y);\; R.s = R_{1}.s \\[8pt]
  R &\to \epsilon && \blue{R.s = R.i}
\end{alignat*}
  \end{center}
\end{frame}
%%%%%%%%%%%%%%%%%%%%

%%%%%%%%%%%%%%%%%%%%
\begin{frame}{}
  \begin{center}
    继承属性 \blue{$R.i$} 用于计算并传递中间结果

    \vspace{0.50cm}
    \fig{width = 1.00\textwidth}{figs/dep-expr-left-right-recursion}

    \vspace{0.50cm}
    \red{\bf 先计算继承属性, 再计算综合属性}
  \end{center}
\end{frame}
%%%%%%%%%%%%%%%%%%%%

%%%%%%%%%%%%%%%%%%%%
\begin{frame}{}
  \begin{center}
    (右递归) $L$属性定义
    \vspace{-0.50cm}
    % right-recursion-L-SDD.tex

\begin{alignat*}{3}
  A &\to XR \quad && \red{R.i} = f(X.x);\; A.a = R.s \\[8pt]
  R &\to YR_{1} && \red{R_{1}.i} = g(R.i, Y.y);\; R.s = R_{1}.s \\[8pt]
  R &\to \epsilon && \blue{R.s = R.i}
\end{alignat*}

    \vspace{0.30cm}
    \red{\bf 原则: 继承属性在处理文法符号之前, 综合属性在处理文法符号之后}

    \pause
    \vspace{0.50cm}
    $L$ 属性定义的 SDT
    \vspace{-0.50cm}
    % SDT-right-recursion-L-SDD.tex

\begin{align*}
  A &\to X \quad \set{\red{R.i} = f(X.x)} \quad R \quad \set{A.a = R.s} \\[8pt]
  R &\to Y \quad \set{\red{R_{1}.i} = g(R.i, Y.y)} \quad R_{1} \quad \set{R.s = R_{1}.s} \\[8pt]
  R &\to \epsilon \quad \set{\blue{R.s = R.i}}
\end{align*}
  \end{center}
\end{frame}
%%%%%%%%%%%%%%%%%%%%

%%%%%%%%%%%%%%%%%%%%
\begin{frame}{}
  \begin{center}
    % SDT-right-recursion-L-SDD.tex

\begin{align*}
  A &\to X \quad \set{\red{R.i} = f(X.x)} \quad R \quad \set{A.a = R.s} \\[8pt]
  R &\to Y \quad \set{\red{R_{1}.i} = g(R.i, Y.y)} \quad R_{1} \quad \set{R.s = R_{1}.s} \\[8pt]
  R &\to \epsilon \quad \set{\blue{R.s = R.i}}
\end{align*}

    % L-SDT-LL-A.tex

\begin{algorithm}[H]
% \caption{}
% \label{alg:L-SDT-LL-A}
\begin{algorithmic}[1]
  \Procedure{\purple{$A$}}{\null} \Comment{$A$ 是开始符号, 无需继承属性做参数}
    \If{\texttt{token} = ?} \Comment{假设选择 $A \to XR$ 产生式}
      \State $X.x \gets \Call{match}{X}$ \Comment{假设 $X$ 是终结符, 返回综合属性}
      \State $\red{R.i} \gets f(X.x)$ \Comment{先计算 \red{$R.i$} 继承属性}
      \State $\blue{R.s} \gets R(R.i)$ \Comment{递归调用子过程 $R(R.i)$}
      \State \purple{\Return $R.s$} \Comment{返回 $A.a \gets R.s$ 综合属性}
    \EndIf
  \EndProcedure
\end{algorithmic}
\end{algorithm}
  \end{center}
\end{frame}
%%%%%%%%%%%%%%%%%%%%

%%%%%%%%%%%%%%%%%%%%
\begin{frame}{}
  \begin{center}
    % SDT-right-recursion-L-SDD.tex

\begin{align*}
  A &\to X \quad \set{\red{R.i} = f(X.x)} \quad R \quad \set{A.a = R.s} \\[8pt]
  R &\to Y \quad \set{\red{R_{1}.i} = g(R.i, Y.y)} \quad R_{1} \quad \set{R.s = R_{1}.s} \\[8pt]
  R &\to \epsilon \quad \set{\blue{R.s = R.i}}
\end{align*}

    % L-SDT-LL-R.tex

\begin{algorithm}[H]
% \caption{}
% \label{alg:L-SDT-LL-R}
\begin{algorithmic}[1]
  \Procedure{\purple{$R$}}{\blue{$R.i$}} \Comment{$R$ 使用继承属性 \blue{$R.i$} 做参数}
    \If{\texttt{token} = ?} \Comment{假设选择 $R \to YR$ 产生式}
      \State $Y.y \gets \Call{match}{Y}$ \Comment{假设 $Y$ 是终结符, 返回综合属性}
      \State $\red{R.i} \gets g(R.i, Y.y)$ \Comment{先计算 \red{$R.i$} 继承属性}
      \State $\blue{R.s} \gets R(R.i)$ \Comment{递归调用子过程 $R(R.i)$}
      \State \purple{\Return $R.s$} \Comment{返回综合属性}
    \ElsIf{\texttt{token} = ?} \Comment{假设选择 $R \to \epsilon$ 产生式}
      \State \purple{\Return $R.i$} \Comment{返回 $R.s \gets R.i$ 综合属性}
    \EndIf
  \EndProcedure
\end{algorithmic}
\end{algorithm}
  \end{center}
\end{frame}
%%%%%%%%%%%%%%%%%%%%

%%%%%%%%%%%%%%%%%%%%
\begin{frame}{}
  \begin{center}
    \blue{\bf $L$ 属性定义转换为 SDT}

    \[
      A \to X_{1} \cdots X_{i} \cdots X_{n}
    \]

    \vspace{0.80cm}
    计算 $X_{i}$ \red{\bf 继承属性}的动作放在产生式体中 $X_{i}$ 的\blue{\bf 左边}

    \vspace{0.30cm}
    计算产生式头部 $A$ \red{\bf 综合属性}的动作放在产生式体的\blue{\bf 最右边}
  \end{center}
\end{frame}
%%%%%%%%%%%%%%%%%%%%

%%%%%%%%%%%%%%%%%%%%
\begin{frame}{}
  \begin{center}
    \input{tables/sdt-L-SDD-LL}
  \end{center}
\end{frame}
%%%%%%%%%%%%%%%%%%%%

%%%%%%%%%%%%%%%%%%%%
\begin{frame}{}
  \begin{center}
    % sdt-question.tex

\begin{table}
  \centering
  \resizebox{0.60\textwidth}{!}{
  \renewcommand{\arraystretch}{1.3}
  \begin{tabular}{|c||c|c|}
    \hline
    & $S$ 属性定义 & $L$ 属性定义
    \\ \hline \hline
    Offline & \multicolumn{2}{c|}{嵌入语义动作\red{\bf 虚拟节点}} \\ \hline
    $LR$ & \red{\bf 后缀}翻译方案 & \includegraphics[scale = 0.15]{figs/question-mark} \\ \hline
    $LL$ & \includegraphics[scale = 0.15]{figs/question-mark} & 先\red{\bf 继承}, 后\red{\bf 综合} \\ \hline
  \end{tabular}}
\end{table}
  \end{center}
\end{frame}
%%%%%%%%%%%%%%%%%%%%

%%%%%%%%%%%%%%%%%%%%
\begin{frame}{}
  \begin{center}
    \input{tables/sdt-S-SDD-LL}

    \pause
    \vspace{0.60cm}
    \[
      A \to \red{\set{B.i = f(A.i)}} B C
    \]
    \pause
    \red{一般而言, $L$-属性文法无法在 $LR$ 中实现。}
  \end{center}
\end{frame}
%%%%%%%%%%%%%%%%%%%%

%%%%%%%%%%%%%%%%%%%%
\begin{frame}{}
  \begin{center}
    \input{tables/sdt-S-SDD-LL-book}

    \vspace{0.50cm}
    \href{https://en.wikipedia.org/wiki/LR-attributed_grammar}{\blue{\bf LR-attributed Grammar}}

    \pause
    \vspace{0.20cm}
    可以在 $LR$ 中实现的 $L$属性文法

    \pause
    \vspace{0.20cm}
    \begin{align*}
      S\texttt{属性文法} \subseteq LR\texttt{属性文法} \subseteq L\texttt{属性文法}
    \end{align*}
  \end{center}
\end{frame}
%%%%%%%%%%%%%%%%%%%%

%%%%%%%%%%%%%%%%%%%%
\begin{frame}{}
  \begin{columns}
    \column{0.50\textwidth}
      \fig{width = 0.50\textwidth}{figs/bison}
    \column{0.50\textwidth}
      \fig{width = 0.80\textwidth}{figs/antlr-logo}
  \end{columns}

  \vspace{0.60cm}
  \begin{center}
    GNU Bison 与 \green{ANTLR3} 支持 SDT
  \end{center}
\end{frame}
%%%%%%%%%%%%%%%%%%%%

%%%%%%%%%%%%%%%%%%%%
\begin{frame}{}
  \fig{width = 0.26\textwidth}{figs/bison-ag}
\end{frame}
%%%%%%%%%%%%%%%%%%%%

%%%%%%%%%%%%%%%%%%%%
\begin{frame}{}
  \begin{center}
    \href{https://github.com/antlr/antlr4/blob/master/doc/faq/general.md\#what-is-the-difference-between-antlr-3-and-4}{What is the difference between ANTLR 3 and 4?
}

    \vspace{0.30cm}
    \fig{width = 1.00\textwidth}{figs/antlr4-actions}
    \pause
    \fig{width = 1.00\textwidth}{figs/antlr4-design}
  \end{center}
\end{frame}
%%%%%%%%%%%%%%%%%%%%

%%%%%%%%%%%%%%%%%%%%
\begin{frame}{}
  \fig{width = 1.00\textwidth}{figs/yacc-ag}

  \begin{center}
    Yacc (Bison): $LALR$ parsers
  \end{center}
\end{frame}
%%%%%%%%%%%%%%%%%%%%